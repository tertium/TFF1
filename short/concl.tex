\section{Conclusion}
\label{sec_concl}

The TPTP TFF1 format nicely complements the existing TPTP
offerings. %, taking over where TFF0 left off.
For reasoning tools that already
support polymorphism, TFF1 is a portable alternative to the existing ad hoc
syntaxes. But more importantly, the format is a vehicle to foster native
polymorphism support in automatic reasoners.
The time is ripe: After many
years of untyped reasoning, we have recently witnessed the rise of
interpreted arithmetic embedded in monomorphic type systems. TFF1
lifts the most obvious restrictions of such systems.

TFF1 is part of TPTP World. The TPTP library already contains
nearly a thousand TFF1 problems, and although the format is in its
infancy, it is supported by several applications, including the SMT solver
Alt-Ergo (via Why3).
Given that many applications today require polymorphism, it
is likely that other reasoning tools will gradually follow suit.
The annual CADE Automated
System Competition (CASC) will certainly have a
role to play driving adoption of the format.
%But regardless of progress in
%prover technology, equipped with a concrete syntax and suitable middleware,
%users can already turn their favorite automatic theorem prover into a
%fairly efficient polymorphic prover.
Rank-1 polymorphism is, of course, no panacea; dependent types and
other advanced features could be part of a future TFF2.

For SMT (satisfiability modulo theories) solvers, the SMT-LIB 2 format \cite{barrett-et-al-2010} specifies a
classical many-sorted logic with equality and interpreted arithmetic, much in
the style of TFF0 but with parametric symbol declarations (overloading).
Polymorphism would make sense there as well, as witnessed by Alt-Ergo.
However, the SMT community is still recovering from the %major
upgrade to SMT-LIB 2 and busy defining a standard proof format;
%\cite{besson-et-al-2011};
implementers would %likely
not welcome yet another
feature at this point. Moreover, with its support for arithmetic, TFF1 is a
reasonable format to implement in an SMT solver if polymorphism is desired.

\def\ackname{Acknowledgment}
\paragraph{\textbf{\upshape\ackname.}}
%
The present specification is largely the result of consensus among
participants of the {\tt polymorphic-tptp-tff} mailing list, especially
Fran\c{c}ois Bobot, Chad Brown, Florian Rabe, Philipp R\"ummer, Stephan Schulz,
Geoff Sutcliffe, and Josef Urban.
We are grateful to Geoff Sutcliffe, TPTP Master of Ceremonies, for giving TFF1
his benediction and adapting the TPTP BNF and other infrastructure.
He, Mark Summerfield, and three anonymous reviewers suggested many textual
improvements to this paper.
We also thank Viktor Kuncak, Tobias Nipkow, and Nicholas Smallbone for their
support and ideas.
%
The first author's research was supported by the Deutsche
Forschungs\-gemein\-schaft project \relax{Hardening the Hammer} (grant
Ni\,491\slash 14-1).
%
%The authors are listed in alphabetical order regardless of individual
%contributions or seniority.
