\section{Syntax} \label{sec_syntax}

Briefly, the types, terms, and formulas of TFF1 are analogous to those of TFF0,
except that function and predicate symbols can be declared to be polymorphic,
types can contain type variables, and $n$-ary type constructors are allowed.
Type variables in type signatures and formulas are explicitly bound. Instances of
polymorphic symbols are specified by explicit type arguments, rather than
inferred.

\ourparagraph{Types.} The {\em types\/} of TFF1 are built from {\em type
variables\/} and {\em type constructors\/} of fixed arities.
The usual conventions of TPTP
apply: Type variables start with an uppercase letter and type constructors
with a lowercase letter. The types \verb+A+, \verb+list(A)+, \verb+list(bird)+,
and \verb+map(nat,+ \verb+list(B))+ are all examples of well-formed types. A
type is {\em polymorphic} if it contains any type variables; otherwise, it is {\em
monomorphic.}

As in TFF0, the type {\tt \$i} %(also called \verb+$iType+)
of individuals is predefined but has no fixed semantics, whereas the
arithmetic types
{\tt \$int}, {\tt \$rat}, and {\tt \$real} are modeled by $\mathbb{Z}$,
$\mathbb{Q}$, and $\mathbb{R}$ \cite{sutcliffe-et-al-2012-tff0}.
It is perfectly acceptable for a TFF
implementation to restrict itself to ``pure TFF\kern.06ex$k$,''
without arithmetic. %For clarity,
TFF\kern.06ex$k$ with arithmetic is sometimes labeled ``TFA\kern.06ex$k$.''

\ourparagraph{Type Signatures.}
Each function and predicate symbol occurring in a formula must be associated
with a {\em type signature\/} that specifies the types of the arguments and, for
functions, the result type. Type signatures can take any of the following forms:
%
\begin{enumerate}
\item[(a)] a type (predefined or user-defined);
\item[(b)] the Boolean pseudotype {\tt \$o}  (the result ``type'' of predicate symbols);
\item[(c)] {\tt ($\tau_1$\;*\;${\cdots}$\;*\;$\tau_n$)\;>\;$\tilde \tau$}
\ for $n > 0$, where $\tau_1,\dots,\tau_n$ are types and $\tilde \tau$ is
a type or {\tt \$o};
\item[(d)] {\tt !>[$\alpha_1$\;:\;\$tType,}\;{\tt ${\dots}$,}\;{\tt
$\alpha_n$\;:\;\$tType]:\;$\varsigma$}
\ for $n > 0$, where $\alpha_1,\dots,\alpha_n$ are distinct
type variables and $\varsigma$ has one of the previous three forms.
%\footnote{This restriction is necessary for the TPTP World's Prolog-based parser.}
%The converse is not true: Some $\alpha_i$ cannot occur in $\tau$.
\end{enumerate}
%
In accordance with TFF0, the parentheses in form (c) are omitted if $n = 1$.
The binder {\tt !>} in form (d) denotes universal quantification.
If $\varsigma$ is of form (c), it must be enclosed in parentheses.
All type variables must be bound by a {\tt !>}-binder.

Form (a) is used for monomorphic constants; form (b), for
propositional constants, including the predefined symbols {\tt \$true} and
{\tt \$false}; form (c), for monomorphic functions and predicates;
and form (d), for polymorphic functions and predicates.
%A few examples:\enskip
%(a) {\tt \$int}, {\tt monkey}, {\tt banana};\enskip
%(b) {\tt \$o};\enskip
%(c) {\tt monkey\;>\allowbreak\;banana},
%    {\tt (monkey\;*\;banana)\allowbreak\;>\allowbreak\;\$o};\enskip
%(d) {\tt !>[A\;:\;\$tType]:\;((A\;*\;list(A))\;>\;list(A))}.

%It is often
%convenient to regard all forms above as instances of the general syntax
%\begin{center}
%{\tt !>[$\alpha_1$\;:\;\$tType,\;${\dots}$,\;$\alpha_m$\;:\;\$tType]:} {\tt
%(($\tau_1$\;*\;${\cdots}$\;*\;$\tau_n$)\;>\;$\tilde \tau$)}
%\end{center}
%where $m$ and $n$ can be 0. % and $\tilde \tau$ can be any type or {\tt \$o}.

%%%% GEOFF: Add examples here.

Type variables that are bound by {\tt !>} without
occurring in the type signature's body are called \emph{\theghost{} type variables.}
These make it possible to specify operations and relations directly on types and
provide a convenient way to encode type classes.
%For example, we can declare a polymorphic propositional
%constant {\tt is\_linear} with the type signature
%{\tt !>[A} {\tt :} {\tt \$tType]:} {\tt \$o} and use it as a guard to restrict the
%axioms specifying that a binary predicate {\tt less\_eq} with the type signature
%{\tt !>[A} {\tt :} {\tt \$tType]:} {\tt ((A} {\tt *} {\tt A)} {\tt >} {\tt \$o)}
%is a linear order to those types that satisfy the {\tt is\_linear} predicate.

%A type with a quantifier prefix is called {\em polytype},
%otherwise we call it {\em monotype}.
%A monomorphic type is called {\em sort}.
%Notice that a monotype can be a polymorphic type,
%i.e.~contain type variables.

\ourparagraph{Type Declarations.} Type constructors
%, like function and predicate symbols,
can optionally be declared.
The following declarations introduce a nullary type
constructor {\tt bird}, a unary type constructor {\tt list},
and a binary type constructor {\tt map}:
\begin{quote}
\verb+tff(bird, type, bird: $tType).+
\par
\verb+tff(list, type, list: $tType > $tType).+
\par
\verb+tff(map, type, map: ($tType * $tType) > $tType).+
\end{quote}
If a type constructor is used before being declared, its arity is
determined by the first occurrence. Any later declaration
must give it the same arity.

A declaration of a function or predicate symbol specifies its type signature.
Every type variable occurring in a type signature must be bound by a
{\tt !>}-binder.
The following declarations introduce a monomorphic
constant {\tt pi}, a polymorphic predicate
{\tt is\_empty}, and a pair of polymorphic functions {\tt cons} and {\tt lookup}:
\begin{quote}
\verb+tff(pi, type, pi: $real).+
\par
\verb+tff(is_empty, type, is_empty : !>[A : $tType]: (list(A) > $o)).+\kern-10mm
\par
\pagebreak[2] %% TYPESETTING
\verb+tff(cons, type, cons : !>[A : $tType]: ((A * list(A)) > list(A))).+
\par
\pagebreak[2] %% TYPESETTING
\verb+tff(lookup, type,+\\
\verb+    lookup : !>[A : $tType, B : $tType]: ((map(A, B) * A) > B)).+\kern-10mm
\end{quote}
If a function or predicate symbol is used before being declared, a
default type signature is assumed:\ {\tt (\$i\;*\;${\cdots}$\;*\;\$i)\;>\;\$i}
for functions and {\tt (\$i\;*\;${\cdots}$\;*\;\$i)\;>\;\$o} for predicates.
If a symbol is declared after its first use, the declared signature
must agree with the assumed signature.
%
If a type constructor, function symbol, or predicate symbol is declared more
than once, it must be given the same type signature up to renaming of bound
type variables.
All symbols share the same namespace.
%; in particular, a type constructor
% cannot have the same name as a function or predicate symbol.

\ourparagraph{Function and Predicate Application.} To keep the required type
inference to a minimum, every use of a polymorphic symbol must explicitly
specify the type instance. A symbol with a type signature
{{\tt !>[$\alpha_1$\;:\;\$tType,\;${\dots}$,\;$\alpha_m$\;:\;\$tType]:\;%
(($\tau_1$\;*\;${\cdots}$\;*\;$\tau_n$)\;>\;$\tilde \tau$)}}
must be applied to $m$ type arguments and $n$ term arguments. Given the above
type signatures for {\tt is\_empty}, {\tt cons}, and {\tt lookup}, the term
\hbox{\tt lookup(\$int,\;\,list(A),\;\,M,\;\,2)}
and the atom
\hbox{\tt is\_empty(\$i,\;\,cons(\$i,\;X,\;nil(\$i)))}
are well-formed and contain free occurrences of the type variable {\tt A}
and the term variables {\tt M} and {\tt X}.

In keeping with TFF1's rank-1 polymorphic nature, type variables can only be
instantiated with actual types. In particular, \verb+$o+, \verb+$tType+,
and {\tt !>}-binders cannot occur in type arguments of polymorphic symbols.

For systems that implement type inference, the following %nonstandard
extension of TFF1 might be useful. When a type argument of
a polymorphic symbol can be inferred automatically, it may be
replaced with the wildcard {\tt \$\_}. For example:
{\tt is\_empty(\$\_,\allowbreak\;\,cons(\$\_,\allowbreak\;X,\;nil(\$\_)))}.
Although {\tt nil}'s type argument cannot be inferred from the types of
its term arguments (since there are none), it can be
deduced from {\texttt X}'s type.
%The producer of a TFF1 problem must be aware of the type
%inference algorithm implemented in the consumer to omit only redundant type
%arguments.

\ourparagraph{Type and Term Variables.}
Every variable in a TFF1 formula must be bound. The variable's type must be specified
at binding time:
\begin{quote}
\begin{verbatim}
tff(bird_list_not_empty, axiom,
    ![B : bird, Bs : list(bird)]:
        ~ is_empty(bird, cons(bird, B, Bs))).
\end{verbatim}
\end{quote}

\pagebreak[3] %% TYPESETTING

If the type and the preceding colon ({\tt :}) are omitted, the variable is given
type~{\tt\$i}. Every type variable occurring in a TFF1 formula
(whether in a type argument or in the type of a bound variable)
must also be bound, with the pseudotype {\tt\$tType}:
\begin{quote}
\begin{verbatim}
tff(lookup_update_same, axiom,
    ![A : $tType, B : $tType, M : map(A, B), K : A, V : B]:
        lookup(A, B, update(A, B, M, K, V), K) = V).
\end{verbatim}
\end{quote}

A single quantifier cluster can bind both type variables and term variables.
%If the type of a term variable contains a type variable, the latter must be bound
%before the former.
Universal and existential quantifiers over type variables are allowed under the
propositional connectives, including equivalence, as well as under other
quantifiers over type variables, but not in the scope of a quantifier over a
term variable, to avoid dependent types.
%Rationale: A statement of the form ``for every integer $k$, there exists a type
%$\alpha$ such that $\ldots$''\ effectively makes~$\alpha$ a dependent type.
%On such statements, type skolemization (Section~\ref{ssec:skol}) is impossible,
%and there is no easy translation to ML-style polymorphic formalisms.
%Moreover, type handling in an automatic prover would be more difficult were
%such constructions allowed, since they require paramodulation into types.

%On the other hand, all the notions and procedures described in this
%specification---except for type skolemization---are independent of this
%restriction. The rules of type checking and the notion of interpretation are
%directly applicable to unrestricted formulas. The encoding into a monomorphic
%logic (Section~\ref{sec_trans}) is sound and complete on unrestricted formulas,
%and the proofs require
%%% Theorems \ref{thm:mon_sound}~and~\ref{thm:mon_compl} require
%%% (I'm no fan of forward references)
%no adjustments. This prepares the ground for TFF2, which is expected to lift the
%restriction and support more elaborate forms of dependent types. Implementations
%of TFF1 are encouraged to support unrestricted formulas, treating them according
%to the semantics given here, if practicable.

\ourparagraph{Example.} The following problem gives the general flavor of TFF1.
It declares and axiomatizes {\tt lookup} and {\tt update} operations on
maps and conjectures that {\tt update} is idempotent for fixed keys and
values. %Its SZS status \cite{sutcliffe-2008-szs} is Theorem.

\begin{quote}
\verb+tff(map, type, map : ($tType * $tType) > $tType).+
\par
\verb+tff(lookup, type,+\\
\verb+    lookup : !>[A : $tType, B : $tType]: ((map(A, B) * A) > B)).+\kern-10mm
\par
\pagebreak[1] %% TYPESETTING
\verb+tff(update, type,+\\
\verb+    update : !>[A : $tType, B : $tType]:+\\
\verb+                 ((map(A, B) * A * B) > map(A, B))).+
\par%\medskip %% deliberate (between declarations and axioms)
\pagebreak[2] %% TYPESETTING
\verb+tff(lookup_update_same, axiom,+\\
\verb+    ![A : $tType, B : $tType, M : map(A, B), K : A, V : B]:+\\
\verb+        lookup(A, B, update(A, B, M, K, V), K) = V).+
\par
\pagebreak[1] %% TYPESETTING
\verb+tff(lookup_update_diff, axiom,+\\
\verb+    ![A : $tType, B : $tType, M : map(A, B), V : B, K : A, L : A]:+\kern-10mm\\
\verb+        (K != L => lookup(A, B, update(A, B, M, K, V), L) =+\\
\verb+                   lookup(A, B, M, L))).+
\par
\pagebreak[1] %% TYPESETTING
\verb+tff(map_ext, axiom,+\\
\verb+    ![A : $tType, B : $tType, M : map(A, B), N : map(A, B)]:+\\
\verb+        ((![K : A]: lookup(A, B, M, K) = lookup(A, B, N, K)) =>+\kern-10mm\\
\verb+         M = N)).+
\par%\medskip %% deliberate (between axioms and conjecture)
\pagebreak[2] %% TYPESETTING
\verb+tff(update_idem, conjecture,+\\
\verb+    ![A : $tType, B : $tType, M : map(A, B), K : A, V : B]:+\\
\verb+        update(A, B, update(A, B, M, K, V), K, V) =+\\
\verb+        update(A, B, M, K, V)).+
\end{quote}
