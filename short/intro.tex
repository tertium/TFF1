\section{Introduction}
\label{sec_intro}

The TPTP World \cite{sutcliffe-2010-world} is a well-established infrastructure
for supporting research, development, and deployment of automated reasoning
tools. It owes its name to its vast problem library, the Thousands of Problems for Theorem
Provers (TPTP) \cite{sutcliffe-2009-lib}. In addition, it specifies concrete
syntaxes for problems and solutions:
Dozens of reasoning tools implement the TPTP untyped clause normal form
(CNF) and first-order form (FOF) for classical
first-order logic with equality.

It has often been argued that the gap between the features supported by provers
and those needed by applications is too wide, and that rich interchange formats
are needed to address this disconnect \cite{voronkov-2003,kuncak-2011}.
A growing number of reasoners can process the
recently introduced TPTP ``core'' typed first-order form (TFF0) \cite{sutcliffe-et-al-2012-tff0},
with monomorphic types and interpreted arithmetic \cite{SPASS-T,vampire-arith},
or the corresponding higher-order form (THF0)~\cite{benzmueller-et-al-2008-thf0}.
A polymorphic version of THF0, the full THF, is in the works~\cite{sutcliffe-benzmueller-2010}.

Despite the variety of this offering, there is a strong desire in part of the automated
reasoning community for a portable \emph{polymorphic first-order format.} Many applications
require polymorphism, notably interactive theorem provers and program
specification languages; but lacking a suitable syntax, applications
and provers must communicate via monomorphic formats. To make matters worse, there is no entirely
satisfactory way to eliminate polymorphism: Monomorphization algorithms are %generally
necessarily incomplete,
and it is difficult to encode polymorphism in a complete yet
also sound and efficient manner,
especially in the presence of interpreted types
\cite{blanchette-et-al-2013-types,bobot-paskevich-2011,leino-ruemmer-2010}. Tool authors
are reduced to developing their own monomorphizers and type encodings, often
using suboptimal schemes. Polymorphism arguably belongs in
provers, where it can be implemented simply and efficiently, as demonstrated by
Alt-Ergo \cite{bobot-et-al-2008}.

\pagebreak

This paper describes the TFF1
format, an extension of TFF0 with rank-1 polymorphism.
The extension was designed with the participation of members of the TPTP
community, reflecting its needs.
Besides compatibility with TFF0 and conceptual integrity with the upcoming full
THF, an important design goal was to ensure that the format can easily be
processed by existing reasoning tools that support ML-style polymorphism. TFF1
also opens the door to useful middleware, such as monomorphizers and other
translation tools. The complete specification is available online.%
\footnote{\url{http://www21.in.tum.de/~blanchet/tff1spec.pdf}}
%
The parts that TFF1 inherits from TFF0
%, such as the concrete syntax for the logical connectives and the (optional) arithmetic constructs,
are described in the TFF0 specification \cite{sutcliffe-et-al-2012-tff0}.
