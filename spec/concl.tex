\section{Conclusion}
\label{sec_concl}

We described the TPTP TFF1 format, an extension of the monomorphic TFF0 format
with rank-1 polymorphism. The new format nicely complements the existing TPTP
offering. %, taking over where TFF0 left off.
For reasoning tools that already
support polymorphism, TFF1 is a portable alternative to the existing ad hoc
syntaxes. But more importantly, the format is a vehicle to foster native
polymorphism support in automatic theorem provers. The time is ripe: After years
upon years of untyped reasoning, the last decade witnessed the rise of
interpreted arithmetic embedded in restrictive monomorphic type systems. TFF1
lifts some of the most obvious restrictions of these type systems.

TFF1 is part of TPTP World. The TPTP library already contains
nearly a thousand TFF1 problems, and although the format is in its
infancy, it is supported by several applications, including the SMT solver
Alt-Ergo (via Why3). Given that many applications today require polymorphism, we
expect other reasoning tools to gradually follow suit. The annual CADE Automated
System Competition (CASC), starting with the 2013 edition, will certainly have a
role to play driving adoption of the format. But regardless of progress in
prover technology, equipped with a concrete syntax and suitable middleware users
can turn their favorite automatic theorem prover into a polymorphic prover.

Rank-1 polymorphism is, of course, no panacea. More advanced features, such as
type classes and dependent types, are not catered for (although type
classes can be comfortably encoded in TFF1). These are expected to be
part of a future TFF2, with the proviso that there be sufficient interest from
users and implementers.

\def\ackname{Acknowledgment}
\paragraph{\textbf{\upshape\ackname.}}
%
The present specification is largely the result of consensus among
participants of the {\tt polymorphic-tptp-tff} mailing list, especially
Fran\c{c}ois Bobot, Chad Brown, Florian Rabe, Philipp R\"ummer, Stephan Schulz,
Geoff Sutcliffe, and Josef Urban.
We are grateful to Geoff Sutcliffe, TPTP Master of Ceremonies, for giving TFF1
his benediction and adapting the TPTP BNF and other infrastructure.
We also thank Viktor Kuncak, Tobias Nipkow, and Nicholas Smallbone for their
support and ideas.
