\section{Conclusion}
\label{sec_concl}

We described the TPTP TFF1 format, an extension of the monomorphic TFF0 format
with rank-1 polymorphism. The new format nicely complements the existing TPTP
offering and is poised to become a lingua franca for automatic theorem provers
and other automated reasoning tools.

TFF1 is implemented as part of the TPTP World infrastructure. The TPTP library
already contains hundreds of polymorphic problems, and although the format is
still in its infancy, it is supported by several applications, including zero
SMT solvers. Given that many applications today require polymorphism, we expect
other reasoning tools to gradually follow suit. The annual CADE Automated System
Competition (CASC) will surely play an important part here, probably starting
with the 2013 edition.

To facilitate implementation, more advanced features, such as type classes and
dependent types, are not addressed by TFF1 (although type classes can be
comfortably encoded in it). These features are expected to be part of a future
TFF2 if there is enough interest from users.


\def\ackname{Acknowledgment}
\paragraph{\textbf{\upshape\ackname.}}
%
We are grateful to Geoff Sutcliffe, TPTP Master of Ceremonies, for giving TFF1
his benediction and adapting the TPTP BNF and other infrastructure to support
it. The present specification is largely the result of consensus among
participants of the {\tt polymorphic-tptp-tff} mailing list, notably
Fran\c{c}ois Bobot, Chad Brown, Florian Rabe, Philipp R\"ummer, Stephan Schulz,
Geoff Sutcliffe, and Josef Urban. Finally, we thank Koen Claessen, Viktor
Kuncak, Tobias Nipkow, and Nicholas Smallbone for their support and ideas.
