\section{Introduction}
\label{sec_intro}

The TPTP World \cite{sutcliffe-2010-world} is a well-established infrastructure
for supporting research, development, and deployment of automated reasoning
tools. It includes a vast problem library, the Thousands of Problems for Theorem
Provers (TPTP) \cite{sutcliffe-2009-lib}, as well as specifications of concrete
syntaxes to facilitate interchange of problems and solutions: The untyped
conjunctive normal form (CNF) and the first-order form (FOF) are implemented in
dozens of reasoning tools, and a growing number of reasoners can process the
typed first-order form (TFF0) \cite{sutcliffe-et-al-2011-tff0} or the
corresponding higher-order form (THF0) \cite{benzmueller-et-al-2008-thf0}, both
of which provide simple monomorphic types and interpreted arithmetic. A
polymorphic version of THF0, the full THF, is in the works
\cite{sutcliffe-benzmueller-2010}.

Despite the variety of this offering, there is a strong desire in the automated
reasoning community for a polymorphic first-order format. Many applications
require polymorphism, notably interactive theorem provers and program
specification languages, but lacking a suitable interchange format these
must communicate via monomorphic formats. Moreover, there is no entirely
satisfactory way to eliminate polymorphism: Monomorphization is generally
incomplete, and it is difficult to encode polymorphism in a sound, complete, and
efficient manner, especially in the presence of interpreted types. Tool authors
are reduced to developing their own monomorphizers and type encodings, often
using suboptimal schemes. Ultimately, we contend that polymorphism belongs in
provers, where it can be implemented simply and efficiently, as demonstrated by
the SMT solver Alt-Ergo \cite{bobot-et-al-2008}.

This paper attempts to change this state of affairs by introducing the TPTP TFF1
format, an extension of the monomorphic TFF0 with rank-1 polymorphism. This
extension was designed together with the TPTP community, reflecting its needs.
Besides compatibility with TFF0 and conceptual integrity with the upcoming full
THF format, an important design goal was to ensure that the format can easily be
processed by existing reasoning tools that support ML-style polymorphism. TFF1
also opens the door to useful middleware, such as monomorphizers and other
translation tools that encode polymorphism in FOF or TFF0.

This paper is structured as follows:
\begin{itemize}
\item Section 2 specifies the format's syntax.
\item Section 3 specifies its typing rules and semantics.
\item Section 4 presents a translation from TFF1 to TFF0.
\item Section 5 presents two preprocessing steps that address type quantifiers
in TFF1 type signatures and formulas, bridging the gap with ML-style
formalisms. % (where type quantification is implicit).
\item Section 6 reviews applications that implement TFF1.
\item Section 7 considers related work---polymorphic formalisms and translation
schemes to untyped or monomorphic logics.
\end{itemize}

This specification extends, rather than replace, the TFF0 specification
\cite{sutcliffe-et-al-2011-tff0}. We refer to that specification for the parts
that TFF1 inherits directly from TFF0, notably the support for arithmetic and
the concrete syntax for the logical connectives.
