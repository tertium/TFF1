\section{Introduction}
\label{sec_intro}

The TPTP World \cite{sutcliffe-2010-world} is a well-established infrastructure
for supporting research, development, and deployment of automatic theorem
provers and other reasoning tools. It includes a vast problem library, the
Thousands of Problems for Theorem Provers (TPTP) \cite{sutcliffe-2009-lib}, as
well as specifications of concrete syntaxes to facilitate interchange of
problems and solutions: The TPTP untyped conjunctive normal form (CNF) and
first-order form (FOF) are the most used syntaxes and are implemented in dozens
of reasoning tools, but a growing number of reasoners are moving to the core
typed first-order form (TFF0) \cite{TFF0} and the corresponding higher-order
form (THF0) \cite{benzmueller-et-al-2008-thf0}, which provide simple monomorphic
types and interpreted arithmetic symbols.

Despite the variety of this offering, there is a strong desire in the automated
reasoning community for a polymorphic first-order format. Many applications
require polymorphism, notably specification languages for program verification
and bridges between automatic and interactive provers; but lacking an
appropriate interchange format that preserves polymorphism, these can only
communicate via monomorphic formats. Furthermore, there is no satisfactory way
to translate polymorphism away while benefiting from interpreted types:
Monomorphizing the problem is generally incomplete, and it is tricky to encode
polymorphism using functions or predicates in a sound, complete, and efficient
manner, especially in the presence of interpreted types.

Lacking such a standard, everybody writes their own monomorphizers and type
encodings, often using suboptimal solutions. Ultimately, we believe that
polymorphism belongs in provers, where it can be implemented simply and
efficiently, as demonstrated by the SMT solver Alt-Ergo \cite{bobot-et-al-2008}.

Higher-order: it's part of the not yet finalized full THF syntax \cite{xxx} but
most reasoning tools out there are first-order, and the higher-order syntax is
sufficiently unwieldy that it's unrealistic to expect apps to support it;
furthermore, the form of polymorphism offered there is much more general than
what we expect to need in the first-order setting.

This specification fills this gap by specifying TPTP TFF1 format, an extension
of the untyped TFF0 with rank-1 polymorphism.%
\footnote{What is now called TFF0 was initially simply called TFF.}
This specification was designed together with the TPTP community. Because of its
situation between TFF0 and full THF, many of the design decisions had already
been taken. One important design goal was to ensure that the format can easily
be processed by existing reasoning tools that support ML-style polymorphism.
TFF1 opens the door to useful middleware, such as monomorphizers and other
translation tools that encode polymorphism in FOF or TFF0. It is already
implemented in a few applications, including one theorem prover.

To facilitate implementation, more advanced features such as type classes and
dependent types are not covered by TFF1 (although type classes can easily be
encoded in terms of it). If there is enough interest in it, these features will
appear in TFF2.

This specification is structured as follows:
\begin{itemize}
\item Section 2 describes syntax.
\item Section 3 describes the typing rules and semantics.
\item Section 4 presents a reduction of TFF1 to TFF0.
\item Section 5 presents two simple preprocessing steps to eliminate type
quantifiers in TFF1 type signatures and formulas, necessary steps to translate
problems to ML-style formalisms, where type quantification is implicit
\item Section 6 presents some applications that implement TFF1 at the time of
writing, in addition to TPTP World.
\item Section 7 considers related work---polymorphic specification languages
and translation schemes to untyped or monomorphic logics.
\end{itemize}

This specification extends, rather than replace, the TFF0 specification
\cite{TFF0}. W attempted to make the document self-contained enough that it can
be read independently but refer to the TFF1 specification for the shared parts,
notably the support for arithmetic and parenthesization issues.
