\section{Introduction}
\label{sec_intro}

The TPTP World \cite{sutcliffe-2010-world} is a well-established infrastructure
for supporting research, development, and deployment of automatic theorem
provers and other reasoning tools. It includes a vast problem library, the
Thousands of Problems for Theorem Provers (TPTP) \cite{sutcliffe-2009-lib}, as
well as specifications of concrete syntaxes to facilitate interchange of
problems and solutions: The untyped conjunctive normal form (CNF) and
first-order form (FOF) are the most used syntaxes and are implemented in dozens
of reasoning tools, but a growing number of reasoners are moving to the core
typed first-order form (TFF0) \cite{TFF0} and the corresponding higher-order
form (THF0) \cite{benzmueller-et-al-2008-thf0}, which provide simple monomorphic
types and interpreted arithmetic.

Despite the variety of this offering, there is a strong desire in the automated
reasoning community for a polymorphic first-order format. Many applications
require polymorphism, notably interactive theorem provers and program
specification languages, but lacking an appropriate interchange format, these
must communicate via monomorphic formats. Furthermore, there is no entirely
satisfactory way to translate polymorphism away: Monomorphization is generally
incomplete, and it is difficult to encode polymorphism in a sound, complete, and
efficient manner, especially in the presence of interpreted types. Tool authors
are reduced to writing their own monomorphizers and type encodings, often using
suboptimal schemes. Ultimately, we contend that polymorphism belongs in provers,
where it can be implemented simply and efficiently, as demonstrated by the SMT
solver Alt-Ergo \cite{bobot-et-al-2008}.

This paper fills this gap by specifying the TPTP TFF1 format, an extension of the
untyped TFF0 with rank-1 polymorphism.%
\footnote{What is now called TFF0 was initially simply called TFF.}
This extension was designed together with the TPTP community. Besides compatibility
with TFF0 and the upcoming full THF format \cite{xxx}, one important design goal
was to ensure that the format can easily be processed by existing reasoning
tools that support ML-style polymorphism. TFF1 opens the door to useful
middleware, such as monomorphizers and other translation tools that encode
polymorphism in FOF or TFF0. It is already implemented in a few applications,
including one theorem prover.

This specification is structured as follows:
\begin{itemize}
\item Section 2 describes syntax.
\item Section 3 describes the typing rules and semantics.
\item Section 4 presents a reduction of TFF1 to TFF0.
\item Section 5 presents two simple preprocessing steps to eliminate type
quantifiers in TFF1 type signatures and formulas, necessary steps to translate
problems to ML-style formalisms, where type quantification is implicit
\item Section 6 presents some applications that implement TFF1 at the time of
writing, in addition to TPTP World.
\item Section 7 considers related work---polymorphic specification languages
and translation schemes to untyped or monomorphic logics.
\end{itemize}

This specification extends, rather than replace, the TFF0 specification
\cite{TFF0}. We attempted to make the document self-contained enough that it can
be read independently but refer to the TFF1 specification for the shared parts,
notably the support for arithmetic and parenthesization issues.

To facilitate implementation, more advanced features such as type classes and
dependent types are not covered by TFF1 (although type classes can easily be
encoded in terms of it). If there is enough interest in it, these features will
appear in TFF2.
