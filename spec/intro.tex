\section{Introduction}
\label{sec_intro}

The TPTP World \cite{sutcliffe-2010-world} is a well-established infrastructure
for supporting research, development, and deployment of automated reasoning
tools. It owes its name to its vast problem library, the Thousands of Problems for Theorem
Provers (TPTP) \cite{sutcliffe-2009-lib}. In addition, it specifies concrete
syntaxes for problems and solutions:
Dozens of reasoning tools implement the TPTP untyped conjunctive normal form
(CNF) and first-order form (FOF) for classical
first-order logic with equality.

It has often been argued that the gap between the features supported by provers
and those needed by applications is too wide, and that rich interchange formats
are needed to address this disconnect
\cite{schumann-2001,voronkov-2003,mccune-2003,stickel-2009,kuncak-2011}.
A growing number of reasoners can process the
recently introduced TPTP ``core'' typed first-order form (TFF0) \cite{sutcliffe-et-al-2012-tff0},
with monomorphic types and interpreted arithmetic \cite{SPASS-T,vampire-arith},
or the corresponding higher-order form (THF0) \cite{benzmueller-et-al-2008-thf0};
a polymorphic version of THF0, the full THF, is in the works~\cite{sutcliffe-benzmueller-2010}.

But despite the variety of this offering, there is a strong desire in part of the automated
reasoning community for a \emph{polymorphic first-order format.} Many applications
require polymorphism, notably interactive theorem provers and program
specification languages; but lacking a suitable syntax, applications
and provers must communicate via monomorphic formats. To make matters worse, there is no entirely
satisfactory way to eliminate polymorphism: Monomorphization is %generally
incomplete, and it is difficult to encode polymorphism in a complete but
also sound and efficient manner, especially in the presence of interpreted types. Tool authors
are reduced to developing their own monomorphizers and type encodings, often
using suboptimal schemes. Polymorphism arguably belongs in
provers, where it can be implemented simply and efficiently, as demonstrated by
the SMT (satisfiability modulo theories) solver Alt-Ergo \cite{bobot-et-al-2008}.

This paper introduces the TFF1
format, an extension of TFF0 with rank-1 polymorphism. The
extension was designed with the participation of members of the TPTP community,
reflecting its needs.
Besides compatibility with TFF0 and conceptual integrity with the upcoming full
THF, an important design goal was to ensure that the format can easily be
processed by existing reasoning tools that support ML-style polymorphism. TFF1
also opens the door to useful middleware, such as monomorphizers and other
tools that encode polymorphism in FOF or TFF0.

For SMT solvers, the SMT-LIB 2 format \cite{barrett-et-al-2010} specifies a
classical many-sorted logic with equality and interpreted arithmetic, much in
the style of TFF0 but with parametric symbol declarations (overloading).
Polymorphism would make sense there as well, as witnessed by Alt-Ergo.
However, the SMT community is still recovering from the %major
upgrade to SMT-LIB 2 and busy defining a standard proof format
\cite{besson-et-al-2011}; implementers would %likely
not welcome yet another
feature at this point. Moreover, with its support for arithmetic, TFF1 is a
reasonable format to implement in an SMT solver if polymorphism is desired.

\newcommand\cheat{\vskip0.3ex} %% TYPESETTING

This paper is structured as follows.
%\begin{itemize}
%\item
Section~\ref{sec_syntax} specifies the TFF1 syntax, and
%\cheat\item
Section~\ref{sec_semantics} specifies its typing rules and semantics.
%\cheat\item
Section~\ref{sec_trans} presents a translation from TFF1 to TFF0.
%\cheat\item
Section~\ref{sec_preproc} bridges the gap with ML-style formalisms by defining
two preprocessing steps that address type quantifiers in TFF1 type signatures
and formulas.
%\cheat\item
Section~\ref{sec_apps} briefly reviews the applications that already implement TFF1.
%\cheat\item
Section~\ref{sec_related} considers related work in the TPTP and SMT
communities:\ existing polymorphic formalisms and translation schemes
to untyped or monomorphic logics.
%\end{itemize}

This specification extends, rather than replaces, the TFF0 specification
\cite{sutcliffe-et-al-2012-tff0}. The parts
that TFF1 inherits directly from TFF0, such as the
concrete syntax for the logical connectives and the
(optional) arithmetic constructs, are described in more detail there.
