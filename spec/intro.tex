\section{Introduction}
\label{sec_intro}

In part motivated by the desire to support arithmetic, TPTP has been slowly
moving towards supporting types, in the form of simple types (many-typed/sorted)
in TFF (Typed First-order Form) and THF0 (Typed Higher-order Form). In the
coming days, a new version of Vampire supporting TFF (actually TFA, i.e., TFF +
arithmetic) will be released, and the developers of E and SPASS are expected to
follow suit.

Since such transitions take many years, it is not too early to be thinking about
the next move. Many specification languages rely on a polymorphic formalism
(Boogie, Why3, Isabelle/HOL, Mizar, ...), and there is no entirely satisfactory
way of translating it away and at the same time benefit from interpreted types
(e.g. for arithmetic). Couchot and Lescuyer (2007) show how to do it in the
context of SMT-LIB, but there's a performance cost (and some trickiness)
associated with the bridging between the general encoding and the interpreted
types. Monomorphization is simple but also not entirely satisfactory. Hybrid
schemes seem to perform best but are fairly complicated.

The time is ripe to develop a polymorphic version of the TFF format. Even if no
ATPs would support it, it could still be worthwhile as an intermediate language
(e.g. between MaLARea and various interactive theorem provers); but our ultimate
goal should be to convince ATP developers to support polymorphic TFF natively.

This wiki gathers documents related to a polymorphic TFF proposal. Such a
proposal should be straightforward to develop: The THF people already did some
work on polymorphism (although it's not in THF0), and monomorphic TFF would have
to be a special case. The syntax should also be future-proof, so that dependent
types, type classes, and other extensions are possible.

The envisioned polymorphic TFF syntax should fit nicely with the rest of TPTP
and subsume private experiments such as Josef Urban's extensions for MPTP 0.2. A
polymorphic TFF syntax, just like Alt-Ergo, could be a major sensitization tool,
and as ATP authors are poised to add types and arithmetic support to their
little babies, they might just as well do it right and deal with polymorphism.
