\section{Introduction}
\label{sec_intro}

The TPTP World \cite{sutcliffe-2010-world} is a well-established infrastructure
for supporting research, development, and deployment of automated reasoning
tools. It includes a vast problem library, the Thousands of Problems for Theorem
Provers (TPTP) \cite{sutcliffe-2009-lib}, as well as specifications of concrete
syntaxes to facilitate interchange of problems and solutions: The untyped
conjunctive normal form (CNF) and first-order form (FOF), for first-order logic with
equality, are implemented in dozens of reasoning tools.

It has been argued before that the gap between the features supported by
reasoners and those needed by applications is too wide
\cite{voronkov-2003}. Richer interchange formats are needed to address this
disconnect
\cite{kuncak-2011}. A growing number of reasoners can process the
recently introduced ``core'' typed first-order form (TFF0) \cite{sutcliffe-et-al-2012-tff0},
which includes monomorphic types and interpreted arithmetic \cite{SPASS-T,vampire-arith},
or the corresponding higher-order form (THF0) \cite{benzmueller-et-al-2008-thf0}.
A polymorphic version of THF0, the full THF, is in the works
\cite{sutcliffe-benzmueller-2010}.

Despite the variety of this offering, there is a strong desire in part of the automated
reasoning community for a \emph{polymorphic first-order format}. Many applications
require polymorphism, notably interactive theorem provers and program
specification languages, but lacking a suitable interchange format these
must communicate via monomorphic formats. Moreover, there is no entirely
satisfactory way to eliminate polymorphism: Monomorphization is generally
incomplete, and it is difficult to encode polymorphism in a sound, complete, and
efficient manner, especially in the presence of interpreted types. Tool authors
are reduced to developing their own monomorphizers and type encodings, often
using suboptimal schemes. We contend that polymorphism ultimately belongs in
provers, where it can be implemented simply and efficiently, as demonstrated by
the SMT (satisfiability modulo theories) solver Alt-Ergo \cite{bobot-et-al-2008}.

We introduce the TFF1
format, an extension of TFF0 with rank-1 polymorphism. The
extension was designed with the participation of members of the TPTP community,
reflecting its needs.
Besides compatibility with TFF0 and conceptual integrity with the upcoming full
THF, an important design goal was to ensure that the format can easily be
processed by existing reasoning tools that support ML-style polymorphism. TFF1
also opens the door to useful middleware, such as monomorphizers and other
tools that encode polymorphism in FOF or TFF0.

For SMT solvers, the SMT-LIB 2 format \cite{barrett-et-al-2010} specifies a
classical many-sorted logic with equality and interpreted arithmetic, much in
the style of TFF0 but with parametric symbol declarations (overloading).
Polymorphism would make sense there as well, as witnessed by Alt-Ergo.
However, the SMT community is still recovering from the major
upgrade to SMT-LIB 2 and busy defining a standard proof format
\cite{besson-et-al-2011}; implementers would likely not welcome yet another
feature at this point. Nonetheless, with its support for arithmetic, TFF1 is a
reasonable format to implement in an SMT solver if polymorphism is desired.

\newcommand\cheat{\vskip0.3ex} %% TYPESETTING

This paper is structured as follows.
%\begin{itemize}
%\item
Section~\ref{sec_syntax} specifies the format's syntax.
%\cheat\item
Section~\ref{sec_semantics} specifies its typing rules and semantics.
%\cheat\item
Section~\ref{sec_trans} presents a translation from TFF1 to TFF0.
%\cheat\item
Section~\ref{sec_preproc} presents two preprocessing steps that address type quantifiers
in TFF1 type signatures and formulas, bridging the gap with ML-style
formalisms. % (where type quantification is implicit).
%\cheat\item
Section~\ref{sec_apps} reviews applications that implement TFF1.
%\cheat\item
Section~\ref{sec_related} considers related work in the TPTP and SMT
communities:\ polymorphic formalisms and translation schemes from
TFF1-like formalisms to untyped or monomorphic logics.
%\end{itemize}

This specification extends, rather than replaces, the TFF0 specification
\cite{sutcliffe-et-al-2012-tff0}. We refer to that specification for the parts
that TFF1 inherits directly from TFF0, notably the
concrete syntax for the logical connectives and the
(optional) arithmetic constructs.
