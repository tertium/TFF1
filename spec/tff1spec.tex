\documentclass[draft,a4paper]{llncs}

\usepackage[utf8]{inputenc}
\usepackage[T1]{fontenc}
%\usepackage{lmodern}
\usepackage{mathptmx}
\usepackage{amsmath,amssymb}
\usepackage[scaled=.82]{beramono}
%\usepackage{xproof}
\usepackage{stmaryrd}
\usepackage{url}
\usepackage{array}
\usepackage{cite}
%\usepackage[final]{graphicx}
%\usepackage{bm}
%\usepackage{xcolor}
%\usepackage{multicol}

\urlstyle{tt}

\DeclareSymbolFont{letters}{OML}{txmi}{m}{it} % Redefines Greek
%letters and other symbols (e.g. \partial) to match the TXfonts

%%% Not so crazy calligraphic font
%\DeclareMathAlphabet{\mathcal}{OT1}{pzc}{m}{it}

%% Pour la redaction %%%
%\usepackage[notref,notcite]{showkeys}
%\usepackage{todonotes} %\usepackage{fixme}
%\usepackage{draftwatermark}
%\usepackage{everypage}
%\usepackage{datetime}
%\AddEverypageHook{\makebox(100,0){DRAFT : \today\ \xxivtime}}
%% Pour la redaction %%%

%\newcommand{\rinfer}[2]{\infer{#2}{#1}}
%\newcommand{\rdeduce}[2]{\deduce{#2}{#1}}

%% @ANDREI: I'm not religious here but I have a fairly strong preference for
%% --> and <--> for implication and equivalence. This leaves \equiv free for
%% syntactic equality (where it sometimes comes in handy) and \supset for
%% supersets. For equality, I find \simeq slightly less disturbing than \approx
%% but I find both acceptable (= is often too confusing, on the other hand).
%\newcommand{\limp}{\supset}
%\newcommand{\lequ}{\equiv}
\newcommand{\limp}{\rightarrow}
\newcommand{\revlimp}{\leftarrow}
\newcommand{\lequ}{\leftarrow\kern-1.75ex\rightarrow} %% TYPESETTING
\newcommand{\inlequ}{\kern.4875ex\not\kern-.4875ex\lequ} %% TYPESETTING
\newcommand{\llb}{\llbracket} \newcommand{\rrb}{\rrbracket}
\newcommand{\la}{\langle} \newcommand{\ra}{\rangle}
\newcommand{\eqdef}{\:\triangleq\:}
\newcommand{\inlineeqdef}{\mathrel{\triangleq}}
\newcommand\eq{\approx}

%\newcommand\typvers{_\mathrm{T}}
\newcommand\typvers{_\mathsf{T}} %% easier to read at low resolutions

\newcommand{\FOLT}{\ensuremath{\mathbf{FOL}\typvers}}
%\newcommand{\LKT}{\ensuremath{\mathbf{LK}\typvers}}

\newcommand{\VT}{\mathbb{V}\typvers}
\newcommand{\FT}{\mathbb{F}\typvers}
\newcommand{\VI}{\mathbb{V}}
\newcommand{\FI}{\mathbb{F}}
\newcommand{\PI}{\mathbb{P}}

\newcommand{\ALL}{\forall}
%\newcommand{\ALL}{{\prod}}
%\newcommand{\ALL}{\mathrm{\Pi}}

%\newcommand{\MI}{\mathrm{MI}}
%\newcommand{\SD}{\mathrm{SD}}
\newcommand{\FV}{\mathrm{FV}}
\newcommand{\FVT}{\mathrm{FV}\typvers}

%\newcommand{\TWU}{\text{\sc Tw}}
%\newcommand{\DIS}{\text{\sc Dis}}
%\newcommand{\PAR}{\text{\sc Par}}
%\newcommand{\DEC}{\text{\sc Dec}}
%\newcommand{\DECc}{\DEC^\circ}
%\newcommand{\EXP}{\text{\sc Exp}}
%\newcommand{\EXPc}{\EXP^\circ}
%\newcommand{\GRD}{\text{\sc Grd}}
%\newcommand{\PIU}{\text{\sc PI}_U}

\newcommand{\MON}{\mathcal{M}}
\newcommand{\ELI}{\mathcal{E}}
\newcommand{\SKO}{\mathcal{S}}

%\newcommand{\BOLDMON}{\mathcal{M}\hspace{-2.1875ex}\mathcal{M}\hspace{-2.1875ex}\mathcal{M}}
\newcommand{\BOLDMON}{\mathcal{M\hspace{-2.65ex}M}}

%\newcommand{\UF}{W}
%\newcommand{\UT}{U}

\newcommand{\eval}[1]{\llb#1\rrb^\cI_{\theta,\xi}}
\newcommand{\evalT}[1]{\llb#1\rrb^\cI_{\theta}}
\newcommand{\evalG}[1]{\llb#1\rrb^\cI}
\newcommand{\evalM}[1]{\llb#1\rrb^\cM}

%\newcommand{\LKTU}{\ensuremath{\mathbf{LK}^U\typvers}}

%\newcommand{\types}{\mathcal{T}(\FT,\VT)}
%\newcommand{\sorts}{\mathcal{T}(\FT)}
\newcommand{\sorts}{\mathbb{D}}
\newcommand{\univ}{\mathbb{U}}

\newcommand{\typeof}{\mathrm{Ty}}
\newcommand{\ghost}{\mathrm{Gh}}
\newcommand{\wit}{\mathrm{Wt}}

%\DeclareMathOperator{\dom}{Dom}
%\newcommand{\dom}{\operatorname{Dom\,}}

%\newcommand{\dom}{\mathcal{D}}
%\newcommand{\cP}{\mathcal{P}}

\newcommand{\cI}{\mathfrak{I}}
\newcommand{\cM}{\mathfrak{M}}

\newcommand{\bbS}{\mathbb{S}}
\newcommand{\bbN}{\mathbb{N}}

\newcommand{\und}{\mathsf{U}}
%\newcommand{\typ}{\mathsf{T}}
\newcommand{\typ}{\mathsf{D}}

\newcommand{\tsA}{\mathsf{A}}
\newcommand{\tsS}{\mathsf{S}}
\newcommand{\tsP}{\mathsf{P}}
\newcommand{\tsM}{\mathsf{M}}
\newcommand{\tsK}{\mathsf{K}}
\newcommand{\tsF}{\mathsf{F}}
\newcommand{\tsI}{\mathsf{I}}
\newcommand{\tsC}{\mathsf{C}}
\newcommand{\tsL}{\mathsf{L}}
\newcommand{\tsB}{\mathsf{B}}

%\newcommand{\toty}{\mathtt{to}}
%\newcommand{\fromty}{\mathtt{from}}
%\newcommand{\tdeco}{\mathtt{deco}}

\newcommand{\lsA}{\mathtt{A}}
\newcommand{\lsS}{\mathtt{S}}
\newcommand{\lsK}{\mathtt{K}}
\newcommand{\lsF}{\mathtt{F}}
\newcommand{\lsR}{\mathtt{R}}
\newcommand{\lsM}{\mathtt{M}}
\newcommand{\lsget}{\mathtt{get}}
\newcommand{\lsset}{\mathtt{set}}
\newcommand{\lsnull}{\mathtt{null}}
\newcommand{\lstrue}{\mathtt{True}}
\newcommand{\lsfalse}{\mathtt{False}}
\newcommand{\lslength}{\mathtt{length}}
\newcommand{\lsfinite}{\mathtt{finite}}
\newcommand{\lsisunit}{\mathtt{isUnit}}
\newcommand{\lszero}{\mathtt{0}}
\newcommand{\lsone}{\mathtt{1}}
\newcommand{\lssix}{\mathtt{6}}
\newcommand{\lsseven}{\mathtt{7}}
\newcommand{\lsfortytwo}{\mathtt{42}}
\newcommand{\lsI}{\mathtt{I}}

%\newcommand{\pty}[1]{\overline{#1}}
%\newcommand{\spty}[1]{\Bar{#1}}
%\newcommand{\sspty}[1]{\scriptscriptstyle{\pty{#1}}}

\newcommand{\ty}{\mathrel{:}}
%\DeclareMathOperator{\vb}{|}
\newcommand\vb{\mid}

%%% TYPESETTING
\renewcommand\labelitemi{\raise.065ex\hbox{\small\textbullet}}

\hyphenation{mono-mor-phic poly-mor-phic prop-o-si-tion-al
her-brand-i-za-tion dis-equa-lity}

\let\ourparagraph=\paragraph
%\newcommand\ourparagraph[1]{\paragraph{\upshape\bfseries#1}}

%\newcommand\sym[1]{\mathtt{#1}}
%\newcommand\sym[1]{\mathit{#1}}
\newcommand\sym[1]{\mathsf{#1}}

%\newcommand\GAM{\gamma}
\newcommand\GAM{\mathrm{\Gamma}}
\newcommand\DEL{\mathrm{\Delta}}

\newcommand\ourfrac[2]{\frac{\,#1\,}{\,#2\,}}
\newcommand\rand{\mathrel{\land}}

\newcommand\afterDot{\;}
\newcommand\vthinspace{\kern.08333ex}

\renewcommand\iff{iff}
%\renewcommand\iff{if and only if}

\begin{document}
\title{
TFF1: The TPTP Typed First-Order Form \\ Extended with Rank-1 Polymorphic Types}
\subtitle{Version 1.0 beta}

\author{
Andrei Paskevich\inst{1,2} \and Jasmin Christian Blanchette\inst{3}
}

\institute{
LRI, Université Paris-Sud 11, CNRS, France \and
ProVal, INRIA Saclay-Île de France, France \and
Institut für Informatik, Technische Universität München, Germany
}

\maketitle

%\vspace{-2ex}
\begin{abstract}
The TPTP (Thousands of Problems for Theorem Provers) is a library of problems
for theorem provers. TPTP defines concrete syntaxes for various logics, notably
an untyped first-order form (FOF) and a monomorphic, simply typed first-order
form (TFF0), that have become de facto standards in the automated reasoning
community. This \paper{} describes the TPTP TFF1 format, an extension of TFF
with rank-1 polymorphism. The format was designed to be easy to process
by existing reasoning tools that support ML-style polymorphism. It opens the
door for useful middleware---for example, monomorphizers and other translation
tools that encode polymorphism in FOF or TFF0. Ultimately, our hope is that TFF1
will be implemented directly in popular automatic theorem provers.
\end{abstract}

\section{Introduction}
\label{sec_intro}

%\noindent
%$\textbf{real}$ \qquad
%$\textbf{bird}$ \qquad
%$\alpha$ \qquad
%$\textbf{list}(\alpha)$ \qquad
%$\textbf{map}(\textbf{list},\, \beta)$
%
%\medskip
%
%\noindent
%$\textbf{bird}_0$ \qquad
%$\textbf{list}_1$ \qquad
%$\textbf{map}_2$
%
%\medskip
%
%\noindent
%$\textsf{pi} \,:\, \textbf{real}$ \qquad
%$\textsf{sin} \,:\, \textbf{real} \to \textbf{real}$ \qquad
%$\textsf{isEmpty} \,:\, \forall\alpha.\;\, \textbf{list}(\alpha) \to \omicron$ \\[\jot]
%$\textsf{cons} \,:\, \forall\alpha.\;\, \alpha \times \textbf{list}(\alpha) \to \textbf{list}(\alpha)$ \qquad
%$\textsf{lookup} \,:\, \forall\alpha,\, \beta.\;\, \textbf{map}(\alpha,\,\beta) \times \alpha \to \beta$
%
%\medskip
%
%\noindent
%$\textsf{sin}(\textsf{pi})$ \qquad
%$\lnot\;\textsf{isEmpty}\langle\textbf{bird}\rangle\bigl(\textsf{cons}\langle\textbf{bird}\rangle(b,\, \textit{bs})\bigr)$ \qquad
%$\textsf{lookup}\langle\textbf{int},\,\textbf{list}(\alpha)\rangle(m, 2)$
%
%\medskip
%
%\noindent
%$\textsf{sin}(\textsf{pi}) = 0$ \qquad
%$\exists b_1 : \textbf{bird},\, b_2 : \textbf{bird}.\;\, b_1 \not\approx b_2$ \\[\jot]
%$\forall b : \textbf{bird},\, \textit{bs} : \textbf{list}(\textbf{bird}).\; \lnot\;\textsf{isEmpty}\langle\textbf{bird}\rangle\bigl(\textsf{cons}\langle\textbf{bird}\rangle(b,\, \textit{bs})\bigr)$ \\[\jot]
%$\forall \alpha,\, \beta.\; \forall m : \textbf{map}(\alpha,\, \beta),\, k : \alpha,\, v : \beta.\;\,
%   \textsf{lookup}\langle\alpha,\,\beta\rangle\bigl(\textsf{update}\langle\alpha,\,\beta\rangle(m,\, k,\, v),\, k\bigr) \approx v$
%
%
%\medskip

The TPTP World \cite{sutcliffe-2010-world} is a well-established infrastructure
for supporting research, development, and deployment of automated reasoning
tools. It owes its name to its vast problem library, the Thousands of Problems for Theorem
Provers (TPTP) \cite{sutcliffe-2009-lib}. In addition, it specifies concrete
syntaxes for problems and solutions:
Dozens of reasoning tools implement the TPTP untyped clause normal form
(CNF) and first-order form (FOF) for classical
first-order logic with equality.

It has often been argued that the gap between the features supported by provers
and those needed by applications is too wide, and that rich interchange formats
are needed to address this disconnect \cite{voronkov-2003,kuncak-2011}.
A growing number of reasoners can process the
recently introduced TPTP ``core'' typed first-order form (TFF0) \cite{sutcliffe-et-al-2012-tff0},
with monomorphic types and interpreted arithmetic \cite{SPASS-T,vampire-arith},
or the corresponding higher-order form (THF0)~\cite{benzmueller-et-al-2008-thf0}.
A polymorphic version of THF0, the full THF, is in the works~\cite{sutcliffe-benzmueller-2010}.

Despite the variety of this offering, there is a strong desire in part of the automated
reasoning community for a portable \emph{polymorphic first-order format.} Many applications
require polymorphism, notably interactive theorem provers and program
specification languages; but lacking a suitable syntax, applications
and provers must communicate via monomorphic formats. To make matters worse, there is no entirely
satisfactory way to eliminate polymorphism: Monomorphization algorithms are %generally
necessarily incomplete,
and it is difficult to encode polymorphism in a complete yet
also sound and efficient manner,
especially in the presence of interpreted types
\cite{blanchette-et-al-2013-types,bobot-paskevich-2011,leino-ruemmer-2010}. Tool authors
are reduced to developing their own monomorphizers and type encodings, often
using suboptimal schemes. Polymorphism arguably belongs in
provers, where it can be implemented simply and efficiently, as demonstrated by
Alt-Ergo \cite{bobot-et-al-2008}.

\pagebreak

This paper describes the TFF1
format, an extension of TFF0 with rank-1 polymorphism.
The extension was designed with the participation of members of the TPTP
community, reflecting its needs.
Besides compatibility with TFF0 and conceptual integrity with the upcoming full
THF, an important design goal was to ensure that the format can easily be
processed by existing reasoning tools that support ML-style polymorphism. TFF1
also opens the door to useful middleware, such as monomorphizers and other
translation tools. The complete specification is available online.%
\footnote{\url{http://www21.in.tum.de/~blanchet/tff1spec.pdf}}
%
The parts that TFF1 inherits from TFF0
%, such as the concrete syntax for the logical connectives and the (optional) arithmetic constructs,
are described in the TFF0 specification \cite{sutcliffe-et-al-2012-tff0}.

\section{Syntax} \label{sec_syntax}

Briefly, the types, terms, and formulas of TFF1 are analogous to those of TFF0,
except that function and predicate symbols can be declared to be polymorphic,
types can contain type variables, and $n$-ary type constructors are allowed.
Type variables in types and formulas are explicitly bound. Instances of
polymorphic constants are specified by explicit type arguments, rather than
inferred.

\ourparagraph{Types.} The {\em types\/} of TFF1 are built from {\em type
variables\/} and {\em type constructors\/} of fixed arity. Nullary type
constructors are called {\em type constants}. We follow the usual notation
conventions of TPTP\@: Type variables start with an uppercase letter and type
constructors with a lowercase letter. The types \verb+A+, \verb+list(A)+,
\verb+list(bird)+, and \verb+map(nat,+ \verb+list(B))+ are all examples of
well-formed types. A type is {\em polymorphic} if it contains type variables;
otherwise, it is {\em monomorphic}.

%%% @ANDREI: $iType and $oType don't appear in the TFF0 spec. I'm not a big fan
%%% of aliases.

TFF1 includes several predefined type constants, inherited from TFF0.
The type {\tt \$i} %(also called \verb+$iType+)
of individuals is predefined but has no peculiar semantics, whereas the
arithmetic types
{\tt \$int}, {\tt \$rat}, and {\tt \$real} are modeled by $\mathbb{Z}$,
$\mathbb{Q}$, and $\mathbb{R}$, respectively. We refer
to the TFF0 specification \cite{TFF0} for the semantics of
the arithmetic types and the predefined operations on
these types. %, and focus here on the nonarithmetic fragment of TFF1.

\ourparagraph{Type Signatures.}
Each function and predicate symbol occurring in a formula must be associated
with a {\em type signature\/} that specifies the types of the arguments and, for
functions, the return type. Type signatures may take the following forms:
%
\begin{enumerate}
\item[(a)] a type;
\item[(b)] the Boolean ``type'' {\tt \$o}; % (also called {\tt \$oType});
\item[(c)] {\tt ($\tau_1$\;*\;${\cdots}$\;*\;$\tau_n$)\;>\;$\tau$} for $n > 0$,
where $\tau_1,\dots,\tau_n$ and $\tau$ are types;
\item[(d)] {\tt ($\tau_1$\;*\;${\cdots}$\;*\;$\tau_n$)\;>\;\$o} for $n > 0$,
where $\tau_1,\dots,\tau_n$ are types;
\item[(e)] {\tt !>\;[$\alpha_1$\::\:\$tType,}\;{\tt ${\dots}$,}\;{\tt
$\alpha_n$\::\:\$tType]:\;$\tau$}, where $\alpha_1,\dots,\alpha_n$ are distinct
type variables, $\tau$ has one of the previous four forms, and every type
variable occurring in $\tau$ appears among the $\alpha_i$'s.
%The converse is not true: Some $\alpha_i$ cannot occur in $\tau$.
\end{enumerate}
%
We omit the parentheses in forms (c) and (d) if $n = 1$. The binder {\tt !>} in
form (e) denotes %type-level  (it's really type-signature-level)
universal quantification.

%%% TODO: define monomorphic and polymorphic symbols?
Form (a) is used for variables and monomorphic constants; form (b), for
propositional constants, including the predefined symbols {\tt \$true} and
{\tt \$false}; form (c), for monomorphic functions; form (d), for monomorphic
predicates; and form (e), for polymorphic functions and predicates.
It is sometimes convenient to regard all the forms above as a special case of
the general syntax
\begin{center}
{\tt !>} {\tt [$\alpha_1$\;:\;\$tType,\;${\dots}$,\;$\alpha_m$\;:\;\$tType]:} {\tt
($\tau_1$\;*\;${\cdots}$\;*\;$\tau_n$)\;>\;$\widehat \tau$}
\end{center}
where $m$ and $n$ may be 0 and $\widehat \tau$ may be any type or {\tt \$o}.

Type variables that are bound by {\tt !>} without
occurring in the type signature's body are called \emph{ghost type variables}.
These make it possible to specify operations and relations directly on types,
providing a convenient way to encode type classes.
For example, we can declare a polymorphic propositional
constant {\tt is\_linear} with the signature
{\tt !>}~{\tt [A} {\tt :} {\tt \$tType]:} {\tt \$o} and use it as a guard to restrict the
axioms specifying that a binary predicate {\tt less\_eq} with the signature
{\tt |>}~{\tt [A} {\tt :} {\tt \$tType]:} {\tt (A} {\tt *} {\tt A)} {\tt >} {\tt \$o}
is a linear order to those types that satisfy the {\tt is\_linear} predicate.

%A type with a quantifier prefix is called {\em polytype},
%otherwise we call it {\em monotype}.
%A monomorphic type is called {\em sort}.
%Notice that a monotype can be a polymorphic type,
%i.e.~contain type variables.

\ourparagraph{Type Declarations.} Type constructors,
like function and predicate symbols,
may optionally be declared prior to use.
For example, the following declarations introduce a type
constant {\tt bird}, a unary type constructor {\tt list},
and a binary type constructor {\tt map}:
\begin{quote}
\verb+tff(bird_t, type, bird: $tType).+
\par\smallskip
\verb+tff(list_t, type, list: $tType > $tType).+
\par\smallskip
\verb+tff(map_t, type, map: ($tType * $tType) > $tType).+
\end{quote}
Type constructors must be fully applied, and therefore
their arity can be %unambiguously
determined at the first occurrence.
%%% @ANDREI: Too early for this. TODO: move where it belongs (and edit)
%However, one should keep in mind that a type
%expression can occur in a position of a term (see below
%the examples of {\tt cons} and other polymorphic symbols).
%A TFF parser should refer to the type signatures of function
%and predicate symbols in order to know whether it parses
%a term or a type.

A declaration of a function or predicate symbol specifies its {\em type
signature}. For example, the following declarations introduce a monomorphic
constant {\tt pi}, a polymorphic predicate
{\tt is\_empty}, and a pair of polymorphic functions {\tt cons} and {\tt lookup}:
\begin{quote}
\verb+tff(pi_t, type, pi: $real).+
\par\smallskip
\verb+tff(is_empty_t, type, is_empty : !> [A : $tType]: list(A) > $o).+\kern-10mm
\par\smallskip
\verb+tff(cons_t, type,+\\
\verb+    cons : !> [A : $tType]: (A * list(A)) > list(A)).+
\par\smallskip
\verb+tff(lookup_t, type,+\\
\verb+    lookup : !> [A : $tType, B : $tType]: (map(A, B) * A) > B).+
\end{quote}
If a function or predicate symbol is used before being declared, a
default type signature is assumed:\ {\tt (\$i\;*\;${\cdots}$\;*\;\$i)\;>\;\$i}
for functions and {\tt (\$i\;*\;${\cdots}$\;*\;\$i)\;>\;\$o} for predicates.

Each type constructor, function symbol, and predicate symbol may be declared at
most once. If a symbol is declared after its first use, the declared signature
must agree with the default signature. All symbols share the same namespace; in
particular, a type constructor cannot have the same name as a function or
predicate symbol.

%\footnote{Is this restriction present in TPTP FOF and TFF0? If not, we can
%lift it for TFF1.}.
%%% @ANDREI: FOF definitely shares the namespace for functions and symbols.
%%% TFF0 doesn't seem to mention anything but Sutcliffe's ToFoF cannot cope with
%%%   tff(foo_type, type, foo : $tType).
%%%   tff(foo_const, type, foo : foo).
%%%   tff(foo_ax, axiom, (![X : foo] : (X = foo))).
%%% Since TFFx is supposed to be about interoperability, in matters like these
%%% I tend not to put too heavy a burden on implementors.

\ourparagraph{Function and Predicate Application.} To keep the required type
inference to a minimum, every use of a polymorphic symbol must explicitly
specify the type instance. A function or predicate symbol with a type signature
\[\hbox{{\tt !>} {\tt [$\alpha_1$\;:\;\$tType,\;${\dots}$,\;$\alpha_m$\;:\;\$tType]:\;%
($\tau_1$\;*\;${\cdots}$\;*\;$\tau_n$)\;>\;$\widehat \tau$}}\]
must be applied to $m$ type arguments and $n$ term arguments. Given the above
signatures for {\tt is\_empty}, {\tt cons}, and {\tt lookup}, the term
\[\hbox{\tt lookup(\$int,\;\,list(A),\;\,M,\;\,2)}\]
and the atomic formula
\[\hbox{\tt is\_empty(\$i,\;\,cons(\$i,\;V,\;nil(\$i)))}\]
are well-formed, assuming that {\tt A} ranges over {\tt
\$tType} (i.e., it is a type variable), {\tt M} has type {\tt
map(\$int,} {\tt list(A))}, and {\tt V} has type {\tt \$i}.

In keeping with TFF1's rank-1 polymorphic nature, type variables can only be
instantiated with actual types. In particular, \verb+$o+, \verb+$tType+,
and {\tt |>} binders cannot occur in type arguments of polymorphic symbols.

For systems that implement type inference, the following extension
of TFF1 might be useful. When a type argument of
a polymorphic symbol can be inferred automatically, it may be
replaced with an underscore ({\tt \_}), for example:
\[\hbox{{\tt is\_empty(\_,} {\tt cons(\_,\;V,\;nil(\_)))}}\]
% \footnote{Should this be part of {\tt SyntaxBNF}?}
% @ANDREI: I don't care much either way... SyntaxBNF is anyway very slack; I
% wouldn't be surprised if it could be used to parse C++. ;)
Although {\tt nil}'s type argument cannot be inferred locally from the types of
the term arguments, it can be deduced using the Hindley--Milner type inference.
The producer of a TFF1 problem must be aware of the type inference algorithm
implemented in the consumer to omit only the redundant type arguments.

\ourparagraph{Type and Term Variables.}
Every variable in a TFF1 formula must be bound. It can be given a type at
binding time:
\begin{quote}
\begin{verbatim}
tff(bird_list_not_empty, axiom,
    ![B : bird, Bs : list(bird)]:
        ~ is_empty(bird, cons(bird, B, Bs))).
\end{verbatim}
\end{quote}
If the type and the preceding colon ({\tt :}) are omitted, the variable is given
the type~\verb+$i+. Every type variable occurring in a TFF1 formula
(whether in a type argument or in the type of a bound variable)
must also be bound:
\begin{quote}
\begin{verbatim}
tff(lookup_set, axiom,
    ![A : $tType, B : $tType, M : map(A, B), I : A, V : B]:
        lookup(A, B, set(A, B, M, I, V), I) = V).
\end{verbatim}
\end{quote}
A single quantifier cluster can bind both type variables and term variables. If
the type of a term variable contains a type variable, the latter must be bound
before the former.
%\footnote{We can permit omitting the {\tt\$tType} annotations,
%too, but should we? If an untyped variable can turn to be
%a type variable or an {\tt\$i}-variable, then we must look
%into the formula to know which it is.}.
%%% @ANDREI: We shouldn't allow that. Chances are that some implementations
%%% would get that wrong.


% (one might need
% to eliminate upper equivalences to establish the polarity
% of a quantifier)
%%% @ANDREI: What is an _upper_ equivalence? An outer one? I'd rather postpone
%%% the discussion on polarity until the part that treats skolemization. It is
%%% enough here to say "including equivalence", to hint at the issue, IMO.

Universal and existential quantifiers over type variables are allowed under the
propositional connectives, including equivalence, as well as under other
quantifiers over type variables, but not in the scope of a quantifier over a
term variable.
Rationale: A statement of the form ``for every integer $k$, there exists a type
$\alpha$ such that $\ldots$'' effectively makes $\alpha$ a dependent type.
On such statements, type skolemization (Section~\ref{ssec:skol}) is impossible,
and there is no easy translation to ML-style polymorphic formalisms.
%%% @ANDREI: We can be forceful here, because "easy" is ambiguous anyway.
%%% Nobody has ever managed to encode dependent types in a nice way in HOL, and
%%% this has been attempted for decades. Surely any new discovery there will not
%%% be "easy".
Moreover, type handling in an automatic prover would be more difficult were
such constructions allowed, since they require paramodulation into types.

On the other hand, all the notions and procedures described in this \paper,
except for type skolemization, are independent of this restriction.
The rules of type checking and the notion of interpretation are
directly applicable to unrestricted formulas. The encoding into a monomorphic
logic (Section~\ref{sec:trans}) is sound and complete on unrestricted formulas,
and the proofs require
%% Theorems \ref{thm:mon_sound}~and~\ref{thm:mon_compl} require
%% (I'm no fan of forward references)
no adjustments. This prepares the ground for TFF2, which is expected to lift the
restriction and support more elaborate forms of dependent types. Implementations
of TFF1 are encouraged to support unrestricted formulas, treating them according
to the semantics given here, if practicable.

%Another way to introduce a variable is via the ``let'' construct.
%A recent extension of the TFF0 format, discussed at the 7th TPTP Tea Party in
%Wrocław and not yet implemented, allows to introduce a function or predicate
%locally. For example, the formulas
%\begin{center}
%\begin{tabular}{l}
%{\tt :=\;[X\;:\;\$int,\;f(X)\;=\;g(X,\;X)]:\;p(f(a))} \\[\smallskipamount]
%{\tt :=\;[X\;:\;\$int,\;p(X)\;<=>\;q(X,\;X)]:\;p(a)}
%\end{tabular}
%\end{center}
%expand to {\tt p(g(a,\;a))} and {\tt q(a,\;a)}, respectively
%\cite{geoff-tptptp-notes}.
%Similarly, TFF1 admits binding a type~$\tau$ to $\kappa(\alpha_1, \ldots,
%\alpha_n)$, where $\alpha_1, \ldots, \alpha_n$ are distinct type variables bound
%in $\tau$, using the syntax $\kappa(\alpha_1, \ldots, \alpha_n)$~{\tt :>}~$\tau$ inside
%the ``let'' binder.

%
%bind a variable
%to a term as well as to a formula (a variable thus bound has
%``type'' \verb+$o+ and can only appear in a position of a formula)
%%% @ANDREI: The proposed syntax is actually more general.
%
%\footnote{Should we elaborate on the syntax of let?
%It was already in TFF0, but it was not described in
%the article on TFF0. The same about if/ite: should
%we talk about them, though they are completely
%orthogonal to polymorphism and types?}.
%In the former case,
%the syntax of binding is {\it variable}~{\tt :-}~{\it term\/} and
%the type of the variable is the type of the term. In the latter case,
%the syntax of binding is {\it variable}~{\tt :=}~{\it formula\/} and
%a variable thus bound has the ``type'' \verb+$o+ and
%can only appear in a position of a formula (notice that
%one cannot quantify over a variable of type \verb+$o+).
%
%\footnote{Since any type is a well-formed TFF1 term,
%we could reuse the existing syntax for these bindings.
%But then suppose that {\tt foo} is undeclared and
%we bind {\tt A} to it: {\tt A} {\tt :-} {\tt foo}.
%Until {\tt A} is used
%(as a type or as a value), we cannot know what it is.}.
For example, the axiom \verb+bird_list_not_empty+ above could also be as
follows:
\begin{quote}
\begin{verbatim}
tff(bird_list_not_empty, axiom,
    := [b :> bird, A : $tType, c(A) :> list(A)]:
        ![B : b, Bs : c(b)]: ~ is_empty(b, cons(b, B, Bs))).
\end{verbatim}
\end{quote}
%One can put a mixed list of bindings in a let-statement;
%in such lists each subsequent binding can refer
%to the variables introduced by preceding ones%
%\footnote{Is this true in TFF0? I didn't find any written
%specification of multi-variable let.}.
%
%The ``let'' bindings can be eliminated from TFF1 formulas using capture-avoiding
%substitution. Hence we do not need to consider them in the rest of this
%specification.

\ourparagraph{Terms and Formulas.} Apart from the differences described above,
the terms and formulas of TFF1 are identical to those of TFF0. We refer to the
TFF0 specification \cite{TFF0} for further information.

\ourparagraph{Notation.} In the sequel, we use standard
mathematical notation to write types, terms, and formulas. Our conventions
for metavariables are summarized below.
%
\begin{center}
\begin{tabular}{l@{\enskip}l@{\qquad}l@{\enskip}l}
Type variables: & $\alpha,$ $\beta$ &
  Function symbols: & $f,$ $g$ \\
Type constructors: & $\kappa$ &
  Predicate symbols: & $p,$ $q$ \\
Types: & $\sigma$, $\tau$ &
  Terms: & $s$, $t$ \\
Term variables: & $u,$ $v$ &
  Formulas: & $\varphi,$ $\psi$
\end{tabular}
\end{center}
%
%We write the names of type constructors, function, and predicate symbols using a
%fixed-width font:\ {\tt list}, {\tt map}, {\tt cons}, {\tt set}.
%% @ANDREI: There were very few examples of this in the rest of the document (lessEq and isOrdered) and
%% it was obvious from the context what they were. Also, I'm trying italics for those, to mix
%% better with the maths; but I'm using the macro \sym, so it's easy to change.
Possibly empty lists of types and terms are denoted by $\bar{\tau}$ and
$\bar{t}$, respectively.

We use the symbols ${\times}$, ${\to}$, and ${\ALL}$
%\footnote{We can also use ${\prod}$, but should we,
%given that there are no dependent types in TFF1?}
to write type signatures, and write $\omicron$ (lowercase omicron) for the
Boolean type {\tt \$o}. It is convenient to treat
equality ($\eq$), negation ($\lnot$), conjunction ($\rand$), and universal
quantification ($\forall$) as logical symbols and regard disequality
($\not\eq$), disjunction ($\lor$), implication ($\limp$), reverse implication
($\revlimp$), equivalence ($\lequ$), inequivalence ($\inlequ$), and existential
quantification ($\exists$) as abbreviations.
%The symbols $\top$ and $\bot$ stand for ``true'' and ``false'',
%respectively.

Equality can be seen as a polymorphic predicate with the signature
$\forall\alpha.\; \alpha\times\alpha\to\omicron$, but the type instance is
implicitly specified by the type of the arguments, instead of explicitly via a
type argument. Hence, it is preferable to consider it a logical
symbol.

The set of type variables occurring freely in a formula $\varphi$
(in the type arguments of polymorphic symbols or
in the types of bound variables) is denoted by $\FVT(\varphi)$.
The set of free term variables of $\varphi$
is denoted by $\FV(\varphi)$. The formula $\varphi$ is {\em closed\/}
if both $\FVT(\varphi)$ and $\FV(\varphi)$ are empty.

A {\em type substitution\/} $\rho$ is a mapping of type variables
to types. A {\em monomorphic\/} type substitution
maps every type variable either to itself or to a monomorphic type.
%
%A type $\tau$ is said to {\em match\/} another type $\tau'$ whenever there exists a
%type substitution~$\rho$ such that $\tau\rho = \tau'$.
%We reserve $\rho$ for type substitutions.
%
%The operator $\circ$ composes two substitutions:\
%$\tau(\rho \circ \rho') \eqdef (\tau\rho)\rho' = \tau\rho\rho'$.

\ourparagraph{Type Checking.}
Let $\GAM$ be a {\em type context}, a function that maps
every variable symbol to a type.
A type judgment $\GAM \,\vdash t \ty \tau$ expresses that that the term $t$
is {\em well-typed\/} and has type $\tau$ in context~$\GAM$.
A type judgment $\GAM \,\vdash \varphi \ty \omicron$ expresses that the formula
$\varphi$ is {\em well-typed\/} in~$\GAM$.
We write $f \ty \forall \alpha_1\dots\alpha_m .\;
\tau_1 \times \dots \times \tau_n \to \tau$ and
$p \ty \forall \alpha_1\dots\alpha_m .\;
\tau_1 \times \dots \times \tau_n \to \omicron$ to specify
type signatures of function and predicate symbols,
where both $m$ and $n$ can be 0.

The typing rules of TFF1 are given below:
%in Figure~\ref{fig:typing}.

\vskip\abovedisplayskip

\centerline{$\displaystyle \ourfrac{}{\GAM \,\vdash\, v \ty \GAM(v)}$}

\vskip3\smallskipamount

\centerline{$\displaystyle
\ourfrac{f \ty \forall \alpha_1\dots\alpha_m .\;
    \tau_1 \times \dots \times \tau_n \to \tau
\qquad
\GAM \,\vdash\, t_1 \ty \tau_1\,\rho
\quad\cdots\quad
\GAM \,\vdash\, t_n \ty \tau_n\,\rho
}{\GAM \,\vdash\,
f(\alpha_1\,\rho,\>\dots,\>\alpha_m\,\rho,\>t_1,\>\dots,\>t_n) \ty \tau\rho}
$}

\vskip3\smallskipamount

\centerline{$\displaystyle
\ourfrac{p \ty \forall \alpha_1\dots\alpha_m .\;
    \tau_1 \times \dots \times \tau_n \to \omicron
\qquad
\GAM \,\vdash\, t_1 \ty \tau_1\,\rho
\quad\cdots\quad
\GAM \,\vdash\, t_n \ty \tau_n\,\rho
}{\GAM \,\vdash\,
p(\alpha_1\,\rho,\>\dots,\>\alpha_m\,\>\rho,\>t_1,\>\dots,\>t_n) \ty \omicron}
$}

\vskip3\smallskipamount

\centerline{$\displaystyle
\ourfrac{\GAM \,\vdash\, t_1 \ty \tau \qquad \GAM \,\vdash\, t_2 \ty \tau}{
\GAM \,\vdash\, t_1 \eq t_2 \ty \omicron}$\qquad$\displaystyle
\ourfrac{\GAM \,\vdash\, \varphi \ty \omicron \qquad
\GAM \,\vdash\, \psi \ty \omicron}{
\GAM \,\vdash\, \varphi \rand \psi \ty \omicron}$}

\vskip3\smallskipamount

\centerline{$\displaystyle
\ourfrac{\GAM \,\vdash\, \varphi \ty \omicron}{
\GAM \,\vdash\, \lnot\, \varphi \ty \omicron}\qquad\qquad
\ourfrac{\GAM[v \mapsto \tau] \,\vdash\, \varphi \ty \omicron}{
\GAM \,\vdash\, \forall v \mathbin{\ty} \tau .\; \varphi \ty \omicron}\qquad\qquad
\ourfrac{\GAM \,\vdash\, \varphi[\alpha'\!/\alpha] \ty \omicron}{
\GAM \,\vdash\, \forall \alpha .\; \varphi \ty \omicron}$}

\vskip\belowdisplayskip

%\caption{Typing rules of TFF1}
%\label{fig:typing}
%\end{figure}

We write $\GAM[u \mapsto \tau]$ to denote a type context that maps
variable $u$ to type $\tau$ and every other variable $v$ to $\GAM(v)$.
In the last rule, $\alpha'$ is an arbitrary type variable that
occurs neither in $\varphi$ nor in the values of~$\GAM$.
The renaming is necessary to reject formulas such as
$\forall\alpha.\:\forall u\ty\alpha.\:\forall\alpha.\:\forall v\ty\alpha.\;
u \eq v$, where the types of $u$ and $v$ are actually different.
In order to simplify our subsequent definitions, we assume from now on
that no type variable can be both free and bound in the same formula;
we call this the {\em no-clash assumption}.
%Consequently, nested quantifiers on the same type
%variable are not allowed, either.
Then we can do without explicit renaming of type variables,
and the last typing rule's antecedent  is simply
$\GAM \,\vdash\, \varphi \ty \omicron$.

A closed TFF1 formula $\varphi$ is {\em well-typed\/} if and only if
the judgment $\GAM \vdash \varphi \ty \omicron$ is derivable
for any $\GAM$.
Obviously, if a closed formula is
well-typed in one type context, it is well-typed in any other one;
hence, we omit $\GAM$ and simply write
${} \vdash \varphi \ty \omicron$.
Closed well-typed formulas are called {\em sentences}.

\section{Semantics}
\label{sec_semantics}

The semantics of a polymorphic formula coincides with that of its
monomorphic instances.

The semantics postulates a nonempty collection $\sorts$ of (not necessarily mutually disjoint)
nonempty sets, the {\em domains}. The union of all domains is the {\em
universe}, %denoted by
$\univ$. An interpretation $\cI$ for a given set of type constructors,
function symbols, and predicate symbols is constructed as follows.

An $n$-ary type constructor $\kappa$ is interpreted as a function
$\kappa^{\cI} : \sorts^n \to \sorts$.
Let $\theta$ be a {\em type valuation}, a function that maps every
type variable to a domain. Types are evaluated according to the following
equations:
\begin{align*}
\evalT{\alpha} &\eqdef \theta(\alpha) &
\evalT{\kappa(\tau_1,\dots,\tau_n)} &\eqdef \kappa^{\cI}(\evalT{\tau_1},\,\dots,\,\evalT{\tau_n})
\end{align*}
Since type evaluation depends only on the values of $\theta$
on the type variables occurring in a type, we write $\evalG{\tau}$
to denote the domain of a monomorphic type $\tau$.
%% NEEDED?:
%We also write
%$\evalG{\tau}_{[\alpha_1\,\mapsto\,D_1,\,\dots,\,\alpha_n\,\mapsto\,D_n]}$
%to evaluate a type whose free type variables
%are among $\alpha_1,\dots,\alpha_n$.

A predicate symbol $p : \forall \alpha_1\dots\alpha_m .\; \tau_1 \times \dots \times \tau_n
\to \omicron$ is interpreted as a relation
$\smash{p^{\cI}} \subseteq \sorts^m \times \univ^n$.
A function symbol $f : \forall \alpha_1\dots\alpha_m .\; \tau_1 \times \dots \times \tau_n \to \tau$
is interpreted as a function
$f^{\cI}$ on $\sorts^m \times \univ^n$ that
maps any $m$ domains $D_1,\dots,D_m$ and
any $n$ universe elements to an element of
$\evalG{\tau}_\theta$,
where $\theta$ maps each $\alpha_i$ to $D_i$.
%Recall that every type variable occurring in a type signature
%must be bound, and therefore every type variable in $\tau$ belongs
%to $\{ \alpha_1,\dots,\alpha_m \}$.

Let $\xi$ be a {\em variable valuation}, a function that assigns
to every variable an element of $\univ$. TFF1 terms
and formulas are evaluated according to the following equations:
\begin{align*}
\eval{u} &\eqdef \xi(u) &
\eval{f(\bar{\tau},\,\bar{t}\,)} &\eqdef f^\cI(\evalT{\bar{\tau}},\eval{\bar{t}\,}) \\
\eval{t_1 \eq t_2} &\eqdef (\eval{t_1} = \eval{t_2}) &
\eval{p(\bar{\tau},\,\bar{t}\,)} &\eqdef p^\cI(\evalT{\bar{\tau}},\eval{\bar{t}\,}) \\
\eval{\lnot\, \varphi} &\eqdef \lnot\, \eval{\varphi} &
\eval{\varphi \rand \psi} &\eqdef \eval{\varphi} \rand \eval{\psi} \\
\eval{\forall u \ty \tau .\; \varphi} &\eqdef \forall {a \mathbin\in \evalT{\tau}} .\;
\evalG{\varphi}_{\theta,\,\xi[u\,\mapsto\,a]} &
\eval{\forall \alpha .\; \varphi} &\eqdef \forall {D \in \sorts} .\;
\evalG{\varphi}_{\theta[\alpha\,\mapsto\,D],\,\xi} % \\
%\eval{\forall u \ty \tau .\; \varphi} &\eqdef \!\!\!\!
%\bigwedge_{a \in \evalT{\tau}} \!
%\evalG{\varphi}_{\theta,\,\xi[u\,\mapsto\,a]} &
%\eval{\forall \alpha .\; \varphi} &\eqdef
%\bigwedge_{D \in \sorts}\;
%\evalG{\varphi}_{\theta[\alpha\,\mapsto\,D],\,\xi}
\end{align*}
The expression $\xi[u \mapsto a]$ stands for the function
that maps $u$ to $a$ and every other variable $v$ to $\xi(v)$, and likewise for
$\theta[\alpha \mapsto D]$.
%Likewise, $\theta[\alpha \mapsto D]$ is the function that maps
%type variable $\alpha$ to domain $D$ and every other type
%variable $\beta$ to $\theta(\beta)$.
We omit irrelevant subscripts and write
$\evalG{\varphi}$ to denote the evaluation of a closed formula.

A sentence $\varphi$ is {\em true} in an interpretation $\cI$,
written $\cI \models \varphi$, if and only if $\evalG{\varphi}$ is true.
The interpretation $\cI$ is then a {\em model\/} of $\varphi$.
A sentence that has a model is {\em satisfiable}.
A sentence that is true in every interpretation is {\em valid}.
These notions are extended to sets and sequents of TFF1 sentences in
the usual way.

\section{Reduction to TFF0} \label{sec:trans}
% Or: Encoding into a Many-Sorted Logic
% Or: Translation to a Many-Sorted Logic
% Or simply: Translation

We describe a translation from TFF1 to the monomorphic simply typed first-order
logic TFF0. The translation is included here for illustrative purposes; more
practical encoding schemes are discussed in Section~\ref{sec_related_work}. To
lighten the presentation, we call the TFF0 types ``sorts,'' keeping ``types''
for TFF1.

Our strategy for translating polymorphic types is to encode them as terms and
use a special binary predicate to encode type information.
%
To avoid mixing types
and terms in the encoded problem, we introduce two sorts, $\typ$ and $\und$,
corresponding to the set of domains $\sorts$ and the universe $\univ$,
respectively. We give the symbols we generate for types the sort $\typ$ while
keeping $\und$ for every other term.

Let $\DEL$ be a set of TFF1 sentences.
%, that is, closed and well-typed formulas.
We construct an equisatisfiable set of monomorphic two-sorted
formulas $\MON(\DEL)$ as follows.
%
To every type variable $\alpha$ in $\DEL$ or in the type signature of a function
symbol, we assign a fresh variable $\hat{\alpha}$ of sort $\typ$.
To every term variable $u$, we assign
a fresh variable $\hat{u}$ of sort $\und$.
To every type constructor $\kappa$ of arity $m$, we assign
a function symbol $\hat{\kappa} : \typ^m \to \typ$.
To every function symbol $f : \forall \alpha_1\dots\alpha_m .\; \tau_1 \times \dots \times \tau_n \to \tau$,
we assign a function symbol $\hat{f} : \typ^m \times \und^n \to \und$.
To every predicate symbol $p : \forall \alpha_1\dots\alpha_m .\; \tau_1 \times \dots \times \tau_n \to \omicron$,
we assign a predicate symbol $\hat{p} : \typ^m \times \und^n \to \omicron$.
Finally, we introduce the predicate symbol $\typeof : \und \times \typ$.

The $\MON$ transformation translates TFF1 types, terms, and formulas
according to the following equations:
\begin{align*}
\MON(\alpha) &\eqdef \hat{\alpha} &
\MON(\kappa(\bar{\tau})) &\eqdef \hat{\kappa}(\MON(\bar{\tau})) \\
\MON(u) &\eqdef \hat{u} &
\MON(f(\bar{\tau},\,\bar{t}\,)) &\eqdef \hat{f}(\MON(\bar{\tau}),\,\MON(\bar{t}\,)) \\
\MON(t_1 \eq t_2) &\eqdef \MON(t_1) \eq \MON(t_2) &
\MON(p(\bar{\tau},\,\bar{t}\,)) &\eqdef \hat{p}(\MON(\bar{\tau}),\,\MON(\bar{t}\,)) \\
\MON(\lnot\, \varphi) &\eqdef \lnot\, \MON(\varphi) &
\MON(\varphi \rand \psi) &\eqdef \MON(\varphi) \rand \MON(\psi) \\
\MON(\forall u \ty \tau .\; \varphi) &\eqdef
\forall \hat{u} .\; \typeof(\hat{u},\, \MON(\tau)) \limp \MON(\varphi) &
\MON(\forall \alpha .\; \varphi) &\eqdef
\forall \hat{\alpha} .\; \MON(\varphi)
\end{align*}
%
\newcommand{\AxD}{\text{\sc Ax}_{\DEL}}%
\newcommand{\Inh}{\text{\sc Inh}}%
\newcommand{\Dom}{\mathrm{Dom}}%
%
The {\em inhabitation axiom}, $\Inh$, is the formula
$\forall \hat{\alpha}.\, \exists \hat{u} .\;
\typeof(\hat{u},\,\hat{\alpha}).$
For each function symbol $f :
\forall \alpha_1\dots\alpha_m .\; \tau_1 \times \dots \times \tau_n \to \tau$,
the associated {\em typing axiom\/} is the formula
$$
\forall \hat{\alpha}_1\dots\hat{\alpha}_m .\;
\forall \hat{u}_1\dots\hat{u}_n .\;
\typeof(\hat{f}(\hat{\alpha}_1,\dots,\hat{\alpha}_m,
\hat{u}_1,\dots,\hat{u}_n), \MON(\tau))%.
$$
We let $\AxD$ denote the set of typing axioms associated with all function
symbols occurring in $\DEL$.
%
Finally, we define
$$
\MON(\DEL) \eqdef \{ \MON(\varphi) \vb \varphi \in \DEL \} \mathrel\cup
\AxD \mathrel\cup \{ \Inh \}%.
$$
%
It is easy to see that $\MON$ converts TFF1 types into well-formed
terms of sort $\typ$, TFF1 terms into well-formed terms of sort $\und$,
and (not necessarily well-typed) TFF1 formulas into well-formed formulas.

\begin{theorem}[Soundness of $\BOLDMON$]%
\label{thm:mon_sound}%
\afterDot
If a set of sentences $\DEL$ is satisfiable in TFF1,
then $\MON(\DEL)$ is satisfiable in classical
many-sorted first-order logic with equality.
\end{theorem}
\begin{proof}
Let $\cM$ be a model of $\DEL$.
We construct an interpretation $\cI$ of $\MON(\DEL)$ as follows.
Let $\sorts$ and $\univ$ stand for
the set of domains and the universe in $\cM$, respectively.
In $\cI$, we define the domain of sort $\typ$ to be $\sorts$
and the domain of sort $\und$ to be $\univ$.
Each function symbol~$\hat{\kappa}$, function symbol $\hat{f}$,
or predicate symbol $\hat{p}$ is interpreted in $\cI$
exactly as the type constructor $\kappa$, function symbol $f$,
or predicate symbol $p$ in $\cM$.
The predicate symbol $\typeof$ is interpreted as the membership relation.
%This concludes the definition of interpretation $\cI$.

We show that for every type $\sigma$, term $t$,
and formula $\varphi$ occurring in $\DEL$,
for every variable valuation $\nu$ in $\cI$, we have
\begin{align*}
\llb \MON(\sigma) \rrb^\cI_{\nu} &= \llb \sigma \rrb^\cM_{\theta} &
\llb \MON(t) \rrb^\cI_{\nu} &= \llb t \rrb^\cM_{\theta,\xi} &
\llb \MON(\varphi) \rrb^\cI_{\nu} &= \llb \varphi \rrb^\cM_{\theta,\xi}
\end{align*}
where $\theta(\alpha) \inlineeqdef \nu(\hat{\alpha})$ and
$\xi(u) \inlineeqdef \nu(\hat{u})$ for every
type variable $\alpha$ and every variable $u$.
%
We prove these equalities by induction on the structure
of types, terms, and formulas. The only nontrivial case is
that of a quantified formula, $\forall u \ty \tau .\; \varphi$.
It is easy to see that any $a \in \univ$ belongs
to $\llb \tau \rrb^\cM_\theta$ if and only if
$\llb \typeof(\hat{u},\MON(\tau)) \rrb^\cI_{\nu'}$,
where $\nu' = \nu[\hat{u} \mapsto a]$, holds.
Indeed, the latter is equivalent to
$a \in \llb \MON(\tau) \rrb^\cI_{\nu'}$.
Since $\nu$ and $\nu'$ produce the same
$\theta$ in $\cM$,
$\llb \MON(\tau) \rrb^\cI_{\nu'} = \llb \tau \rrb^\cM_\theta$.
Then
$\llb \forall \hat{u} .\; \typeof(\hat{u},\MON(\tau))
\limp \MON(\varphi) \rrb^\cI_{\nu}$ is exactly
$\llb \forall u \ty \tau .\; \varphi \rrb^\cM_{\theta,\xi}$
by induction hypothesis.

Now we only have to check that the typing axioms
$\AxD$ and the inhabitation axiom $\Inh$ hold in $\cI$.
This immediately follows from the definition of
interpretation in TFF1 and the first equality above.
\qed
\end{proof}

\begin{theorem}[Completeness of $\BOLDMON$] \label{thm:mon_compl}
Given a set of sentences $\DEL$, if $\MON(\DEL)$
is satisfiable in classical many-sorted first-order logic
with equality, then $\DEL$ is satisfiable in TFF1.
\end{theorem}
\begin{proof}
Let $\cM_0$ be a model of $\MON(\DEL)$. We first construct a model $\cM$ of $\MON(\DEL)$
from $\cM_0$ by replacing every element $d$ in
the domain of $\typ$ by the set $D = \{\la a,\,d \ra \vb \exists a.\; \typeof^{\cM_0\!}(a,\,d)\}$
and updating the interpretations of type constructors
and function and predicate symbols in $\cM_0$ accordingly.
We can safely perform this substitution: Since the inhabitation
axiom $\Inh$ holds in $\cM_0$, every set~$D$ is nonempty, and clearly distinct
elements are mapped to distinct sets.
The predicate $\typeof^{\cM\!}(a,\,D)$ holds if and only if
$D$ contains a pair of the form $\la a,\, d \ra$ for some $d$.
%
Below, $\pi_1$ and $\pi_2$ stand for the first and the second projection
of a pair, respectively.

We construct an interpretation $\cI$ of $\DEL$ as follows.
The set of domains $\sorts$ is the new domain
of $\typ$ in $\cM$. As usual, $\univ$ denotes the union
of all domains in $\sorts$. Note that some elements of
the domain of $\und$ may not appear in any pair in $\univ$.
%
The symbols of $\DEL$ are interpreted in $\cI$
according to the equations
\begin{align*}
\kappa^\cI(D_1,\dots,D_m) &\eqdef \hat{\kappa}{\,}^\cM(D_1,\dots,D_m) \\
f^\cI(D_1,\dots,D_m,\la a_1,d_1 \ra,\dots,\la a_n,d_n \ra) &\eqdef
\bigl\la \hat{f}{\,}^\cM(D_1,\dots,D_m,a_1,\dots,a_n),\, d \bigr\ra \\
p^\cI(D_1,\dots,D_m,\la a_1,d_1 \ra,\dots,\la a_n,d_n \ra) &\eqdef
\hat{p}{\,}^\cM(D_1,\dots,D_m,a_1,\dots,a_n)
\end{align*}
where
$\kappa$ is of arity $m$,
$p : \forall \alpha_1\dots\alpha_m .\; \tau_1 \times \dots \times \tau_n \to
\omicron$,
$f : \forall \alpha_1\dots\alpha_m .\; \tau_1 \times \dots \times \tau_n \to \tau$,
and $d$ is the fixed second coordinate of pairs in the domain
%
\[D = \llb \MON(\tau) \rrb^\cM_{[\hat{\alpha}_1\,\mapsto\,D_1,\dots,
\hat{\alpha}_m\,\mapsto\,D_m]} =
\llb \tau \rrb^\cI_{[\alpha_1\,\mapsto\,D_1,\dots,\alpha_m\,\mapsto\,D_m]}\]%.
%
Since the typing axioms $\AxD$ hold in $\cM$, we have
$\typeof^\cM(f^\cM(D_1,\dots,D_n,a_1,\dots,a_m), D)$,
and therefore, the result of $f^\cI$ belongs indeed to $D$.
%This concludes the definition of $\cI$.

Given a type context $\GAM$, a type valuation $\theta$,
and a variable valuation $\xi$, we say that they are \emph{admissible}
for a TFF1 formula $\varphi$ when the following conditions are satisfied:
\begin{itemize}
\item $\varphi$ is well-typed in context $\GAM$;
\item for every variable $v$ free in $\varphi$, no type variable
occurring in $\GAM(v)$ is bound in $\varphi$;
\item for every variable $v$ free in $\varphi$, we have $\xi(v) \in \evalT{\GAM(v)}$.
\end{itemize}
Obviously, any triple is admissible for a sentence.
We must show that for every type $\sigma$, term $t$, and formula $\varphi$,
and for all $\GAM$, $\theta$, $\xi$ admissible for $\varphi$, we have
\begin{align*}
\evalT{\sigma} &= \llb \MON(\sigma) \rrb^\cM_{\nu} &
\pi_1(\eval{t}) &= \llb \MON(t) \rrb^\cM_{\nu} &
\eval{\varphi} &= \llb \MON(\varphi) \rrb^\cM_{\nu}
\end{align*}
where $\nu(\hat{\alpha}) \inlineeqdef \theta(\alpha)$ and
$\nu(\hat{u}) \inlineeqdef \pi_1(\xi(u))$
for every type variable $\alpha$ and variable $u$.

The proof goes by induction on the structure of types, terms, and formulas.
There are three nontrivial cases:\ equality, quantification over a term
variable, and quantification over a type variable.

Let $\varphi$ be an equality $t_1 \eq t_2$.
Let $\la a_1,\, d_1 \ra$ be $\eval{t_1}$ and $\la a_2,\, d_2 \ra$ be $\eval{t_2}$.
By induction hypothesis, $\llb \MON(t_1) \rrb^\cM_{\nu} = a_1$ and
$\llb \MON(t_2) \rrb^\cM_{\nu} = a_2$. We must show $d_1 = d_2$.
By assumption, $\varphi$ is well-typed in $\GAM$.
Thus, $\GAM\,\vdash t_1 \ty \sigma$ and $\GAM\,\vdash t_2 \ty \sigma$
for some type~$\sigma$.
If $t_1$ is a variable $v$, then
$\eval{t_1} = \xi(v) \in \eval{\GAM(v)} = \evalT{\sigma}$. Otherwise,
$t_1$ is a function application $f(\sigma_1,\dots,\sigma_m,s_1,\dots,s_n)$,
for $f \ty
\forall \alpha_1\dots\alpha_m .\; \tau_1 \times \dots \times \tau_n \to \tau$.
Let $D_i = \evalT{\sigma_i}$ for every $i \in [1,m]$.
Then $\sigma = \tau[\sigma_1/\alpha_1,\dots,\sigma_n/\alpha_n]$ and
$\evalT{\sigma} = \llb \tau \rrb^\cI_{[\alpha_1 \,\mapsto\, D_1,\dots,
\alpha_n \,\mapsto\, D_n]}$.
By construction of $\cI$, $\eval{t_1} \in \evalT{\sigma}$.
By the same argument, $\eval{t_2} \in \evalT{\sigma}$.
Hence, $d_1 = d_2$ and $\eval{\varphi} = \llb \MON(\varphi) \rrb^\cM_{\nu}$.

Let $\varphi$ be a quantified formula $\forall u \ty \tau .\; \psi$.
We must show that for every pair $\la a,\, d \ra$ in $\evalT{\tau}$, we have
$\typeof^\cM(a, \llb \MON(\tau) \rrb^\cM_{\nu'})$,
where $\nu' = \nu[\hat{u} \mapsto a]$,
and vice versa, for every $a$ in the domain of $\und$, if
$\typeof^\cM(a, \llb \MON(\tau) \rrb^\cM_{\nu'})$
holds, then there exists some $d$ such that $\la a,\, d \ra \in
\evalT{\tau}$.
%
Since term variables do not occur in types,
$\llb \MON(\tau) \rrb^\cM_{\nu'} =
\llb \tau \rrb^\cI_\theta$.
By construction of $\cM$,
$\typeof^\cM(a, \llb \tau \rrb^\cI_\theta)$
holds if and only if there exists some $d'$ such that
$\la a,\, d' \ra \in \llb \tau \rrb^\cI_\theta$.
%
Now, notice that the triple $\GAM[u \mapsto \tau],
\theta,\xi[u \mapsto \la a,\, d \ra]$ is admissible for $\psi$.
Indeed, $\psi$ is well-typed in $\GAM[u \mapsto \tau]$,
and for every $v \in \FV(\psi)$, we have
\[\xi[u \mapsto \la a,\, d \ra](v) \in \evalT{\GAM[u \mapsto \tau](v)}\]%.
Also, if $\tau$ contains a type variable $\beta$ bound in $\psi$,
then $\beta$ is both free and bound in $\varphi$, which violates
the no-clash assumption.
%
Then, by induction hypothesis, $\eval{\forall u \ty \tau .\; \psi}$
is exactly $\llb \forall \hat{u} .\; \typeof(\hat{u},\MON(\tau))
\limp \MON(\psi) \rrb^\cM_{\nu}$.

Let $\varphi$ be a quantified formula $\forall \alpha .\; \psi$.
Let $D$ be an element of $\sorts$. Under the no-clash assumption,
formula $\psi$ is well-typed in $\GAM$, and for all
$v \in \FV(\psi) = \FV(\varphi)$, we have
$\evalG{\GAM(v)}_{\theta[\alpha\,\mapsto\,D]} = \evalT{\GAM(v)}$.
Then $\eval{\varphi} = \llb \MON(\varphi) \rrb^\cM_{\nu}$ by induction hypothesis.
\qed
\end{proof}

\section{Preprocessing of Type Quantifiers}

We describe two preprocessing steps that preserve both satisfiability and
unsatisfiability of TFF1 problems. They eliminate ghost type variables in type
signatures and alternating $\forall\raise.15ex\hbox{\ensuremath{\slash}}\vthinspace\exists$ type quantifier prefixes in
formulas, two features that are not directly supported by ML-style formalisms
(which otherwise are a good match for TFF1).

\subsection{Elimination of Ghost Type Arguments} \label{ssec:ghost}

%\footnote{Other relevant citations, please? What about Mizar?}.
% @ANDREI: I'd rather stick to "ML-style formalisms" here. Mizar is based on
% set theory and has at most soft types; it would be better treated in the
% Related Work section.

ML-style formalisms, as implemented in Alt-Ergo~\cite{conchon08smt},
Boogie~\cite{Barnett06boogie}, HOL \cite{gordon-melham-1993}, HOL
Light~\cite{harrison-1996}, Isabelle\slash HOL~\cite{nipkow-et-al-2002},
Why3~\cite{boogie11why3}, and several other systems, allow type variables
in type signatures, but without explicit $\forall$-binders.
Function and predicate symbols do not take explicit type
arguments; instead, the concrete instance of the symbol's type signature is
determined by the types of its term arguments, the type of the result, and
optional type annotations inside terms.

The natural translation from TFF1 to such a formalism would map $\forall
\alpha\beta.\; \alpha\times\beta\to\omicron$ (and $\forall \beta\alpha.\;
\alpha\times\beta\to\omicron$) to $\alpha\times\beta\to\omicron$, simply
omitting any $\forall$-binders. To compensate for the missing type arguments,
type annotations are sometimes needed to guide the Hindley--Milner type
inference. This is slightly awkward but not difficult to implement.

The main difficulties arise in conjunction with ghost type variables: If
$\forall\alpha.\;\omicron$ collapses to $\omicron$, the dependency on the type
is lost.
%
To ease the adoption of TFF1 in such systems, we suggest a preprocessing step
that eliminates ghost type variables. This step is lightweight, as it requires
only the introduction of one term argument per ghost type. In particular,
it is the identity for formulas that do not rely on ghost type variables.

%that is, the type variables in the symbol's type signature that
%do not appear in the quantifier-free matrix.

Let $\DEL$ be a set of sentences.
We construct an equisatisfiable set $\ELI(\DEL)$ as follows.
%
We introduce a special unary type constructor $\ghost$.
We replace every function symbol $f : \forall \alpha_1\dots\alpha_m .\; \tau_1 \times \dots \times \tau_n \to \tau$
with a new function symbol $\hat{f} : \forall \alpha_1\dots\alpha_m .\;
\ghost(\alpha_{i_1}) \times \dots \times \ghost(\alpha_{i_r}) \times
\tau_1 \times \dots \times \tau_n \to \tau$,
where $\alpha_{i_1},\dots,\alpha_{i_r}$ are the type
variables from the quantifier prefix that do not occur in
$\tau_1,\dots,\tau_n$ or $\tau$.
Similarly, every predicate symbol $p \,\ty\,
\forall \alpha_1\dots\alpha_m .\;\allowbreak \tau_1 \times \dots \times \tau_n \to \omicron$
is replaced with a new predicate symbol $\hat{p} \,\ty\,
\forall \alpha_1\dots\alpha_m .\;\allowbreak
\ghost(\alpha_{i_1}) \times \dots \times \ghost(\alpha_{i_r}) \times
\tau_1 \times \dots \times \tau_n \to \omicron$,
where $\alpha_{i_1},\dots,\alpha_{i_r}$ are the type
variables from the quantifier prefix that do not occur in
$\tau_1,\dots,\tau_n$.
Finally, we add a ``witness'' constant $\wit : \forall \alpha.\;\ghost(\alpha)$.
The $\ELI$ transformation translates terms and atomic
formulas according to the following equations:
%\begin{align*}
%\ELI(u) & \eqdef u &
%\ELI(s \eq t) & \eqdef \ELI(s) \eq \ELI(t)
%\end{align*}
\begin{align*}
\ELI(u) \eqdef u
\kern1.4em & \kern5.55em %% TYPESETTING
\ELI(s \eq t) \eqdef \ELI(s) \eq \ELI(t) \\
\ELI(f(\sigma_1,\dots,\sigma_m,\bar t\,)) &\eqdef
\hat{f}(\sigma_1,\dots,\sigma_m,\wit(\sigma_{i_1}),\dots,\wit(\sigma_{i_r}),
\ELI(\bar t\,)) \\
\ELI(p(\sigma_1,\dots,\sigma_m,\bar t\,)) &\eqdef
\hat{p}(\sigma_1,\dots,\sigma_m,\wit(\sigma_{i_1}),\dots,\wit(\sigma_{i_r}),
\ELI(\bar t\,))
\end{align*}
Given a formula $\varphi$ or a set of sentences $\DEL$,
$\ELI(\varphi)$ and $\ELI(\DEL)$ denote the result of
applying $\ELI$ to every atomic formula in $\varphi$ and
$\DEL$, respectively.

\begin{theorem} \label{thm:eli}
Any $\DEL$ is equisatisfiable to $\ELI(\DEL)$.
\end{theorem}
\begin{proof}
A model of $\DEL$ can be converted to a model of $\ELI(\DEL)$
as follows. We choose an arbitrary value $e$ and take the singleton
$\{ e \}$ as the domain of $\ghost(D)$ for any domain $D$.
Accordingly, $\wit$ evaluates to $e$ on any argument.
Interpretations of other function and predicate symbols are
adjusted in the obvious way.

To convert a model of $\ELI(\DEL)$
to a model of $\DEL$, we evaluate any functional symbol
$f$ on $D_1,\dots,D_m,a_1,\dots,a_n$ exactly as
$\hat{f}$ on $D_1,\dots,D_m,e_1,\dots,e_r,a_1,\dots,a_n$,
where each $e_k$ is the evaluation of $\wit$ on $D_{i_k}$, and similarly
for predicate symbols.
\qed
\end{proof}

\subsection{Type Skolemization} \label{ssec:skol}

Another feature of TFF1 that is not universally supported
by systems with polymorphic types is explicit
quantification over type variables.
For example, while Boogie and Coq \cite{CoqManualV8}
allow quantifiers over types in their logics,
HOL and other ML-style formalisms provide no such syntax:
Instead, they consider all type variables
implicitly universally quantified at the top of a formula, which is similar to
the treatment of term variables in TPTP CNF clauses.
A simple and practical solution to translate TFF1 problems to prepare problems
for ML-style logics is type skolemization. This is slightly involved because
type quantifiers may appear under equivalence, where they are unpolarized.

We introduce two transformations, $\SKO^-$ and $\SKO^+$, to perform type
skolemization in premises and in goals, respectively. (The latter is more
correctly called ``type herbrandization.'')
%Let us call a formula $\varphi$ {\em dependently typed} whenever
%it contains a quantifier over a type variable in the scope
%of a quantifier over a term variable%
%\footnote{We might take polarities into account, but then
%the proof of Theorem~\ref{thm:sko} wouldn't be so sweet.
%Types-under-values are nasty anyway.}.
Their definitions follow:
\begin{align*}
\SKO^-(p(\bar{\sigma},\bar{t}\,)) &\eqdef p(\bar{\sigma},\bar{t}\,) &
\SKO^+(p(\bar{\sigma},\bar{t}\,)) &\eqdef p(\bar{\sigma},\bar{t}\,) \\
\SKO^-(t_1 \eq t_2) &\eqdef t_1 \eq t_2 &
\SKO^+(t_1 \eq t_2) &\eqdef t_1 \eq t_2 \\
\SKO^-(\forall u \ty \tau .\; \varphi) &\eqdef \forall u \ty \tau .\; \varphi &
\SKO^+(\forall u \ty \tau .\; \varphi) &\eqdef \forall u \ty \tau .\; \varphi \\
\SKO^-(\varphi \rand \psi) &\eqdef \SKO^-(\varphi) \rand \SKO^-(\psi) &
\SKO^+(\varphi \rand \psi) &\eqdef \SKO^+(\varphi) \rand \SKO^+(\psi) \\
\SKO^-(\lnot\, \varphi) &\eqdef \lnot\, \SKO^+(\varphi) &
\SKO^+(\lnot\, \varphi) &\eqdef \lnot\, \SKO^-(\varphi) \\
\SKO^-(\forall \alpha .\; \varphi) &\eqdef \forall \alpha .\; \SKO^-(\varphi) &
\SKO^+(\forall \alpha .\; \varphi) &\eqdef
\SKO^+(\varphi[\kappa(\bar\beta)/\alpha])
\end{align*}
where $\kappa$ is a fresh type constructor and $\bar\beta$ are
the free type variables of $\forall \alpha .\; \varphi$.
Recall that TFF1 forbids quantifiers over type variables in the scope
of a quantifier over a term variable; otherwise, Skolem type constructors
would have to take term variables as arguments. This explains why $\SKO^-$ and
$\SKO^+$ simply stop at the outermost term quantifier.

Given a set of TFF1 sentences $\DEL$,
$\SKO^-(\DEL)$ and $\SKO^+(\DEL)$ denote the result of applying
the corresponding transformations to every formula in $\DEL$.

\begin{theorem} \label{thm:sko}
Any $\DEL$ is equisatisfiable to $\SKO^-(\DEL)$.
\end{theorem}
\begin{proof}
It is easy to see that
skolemizing the $\typ$-sorted variables in $\MON(\DEL)$ gives
exactly $\MON(\SKO^-(\DEL))$ modulo renaming of Skolem symbols and
permutation of their arguments. Theorems~\ref{thm:mon_sound} and
\ref{thm:mon_compl} conclude the proof.
\qed
\end{proof}

%%% @ANDREI: Which restriction are you talking about? I'm not following you here:
%
%Given a TFF1 problem that does not satisfy the aforementioned restriction,
%one can submit it to a system like Isabelle/HOL or Alt-Ergo by encoding it
%with $\MON$ or any other sound translation into monomorphic logic.

\section{Applications}
\label{sec_apps}

A number of applications already support TFF1. Geoff
Sutcliffe has extended the TPTP World infrastructure to process TFF1 problems
and solutions. This involved in particular adapting the BNF specification of the
TPTP syntaxes, from which parsers are generated. Some TPTP tools still need to be
ported to TFF1; this is ongoing work.

%%% TODO: update: grep "pure"
The Why3 \cite{bobot-et-al-2011} environment, which defines its own ML-like
polymorphic specification language, can parse pure TFF1. Why3 translates
between TFF1 and a wide range of
formats, including FOF, SMT-LIB, and
Alt-Ergo's native syntax. In addition, Why3's TFF1 parser is being ported to
Alt-Ergo \cite{bobot-et-al-2008}, so that it can directly process TFF1. % problems.
%without taking a detour through Why3.
%Code sharing is facilitated here by the use of
%a common implementation language (OCaml).

Sledgehammer \cite{paulson-blanchette-2010}, a tool that bridges the interactive
theorem prover Isabelle\slash HOL and various automatic provers, has now been
extended to output pure TFF1 problems for Alt-Ergo and Why3
(in addition to FOF, TFF0, and THF0). For the first time, Sledgehammer
exploits the polymorphic potential of these tools---without having
to implement their (incompatible) native file formats.
Moreover, using Sledgehammer, we produced 987 problems to populate the TPTP
library.%
\footnote{\url{http://www.cs.miami.edu/~tptp/TPTP/Proposals/TFF1.html}}
%PFF (``\relax{P}olymorphic T\relax{FF}'')
%domain of the TPTP library.
By extending the tool with a TFF1 parser,
we hope to transform the tool into a versatile translator to FOF and
TFF0.

%Other tools, notably an LF-based type checker similar to TwHelFTC for THF0, are
%in development.

%  * ToFoF, Monotonox, other -ox tools?

%  * TPTP Library:
%     * accepts all problems
%     * populated it with problems generated by Sledgehammer and Why3
%       (and Alt-Ergo?)

\section{Related Work}
\label{sec_related_work}

The Boogie language~\cite{Barnett06boogie} supports explicit quantifiers over
types, and the encoding methods presented in Leino and R\"ummer
\cite{leino10tacas} are devised for the unrestricted setting.

\section{Conclusion}
\label{sec_concl}

This paper described the TPTP TFF1 format, an extension of the monomorphic TFF0 format
with rank-1 polymorphism. The new format nicely complements the existing TPTP
offerings. %, taking over where TFF0 left off.
For reasoning tools that already
support polymorphism, TFF1 is a portable alternative to the existing ad hoc
syntaxes. But more importantly, the format is a vehicle to foster native
polymorphism support in automatic theorem provers. The time is ripe: After years
upon years of untyped reasoning, the last decade witnessed the rise of
interpreted arithmetic embedded in restrictive monomorphic type systems. TFF1
lifts some of the most obvious restrictions of these type systems.

TFF1 is part of TPTP World. The TPTP library already contains
nearly a thousand TFF1 problems, and although the format is in its
infancy, it is supported by several applications, including the SMT solver
Alt-Ergo (via Why3). Given that many applications today require polymorphism, it
is likely that other reasoning tools will gradually follow suit.
The annual CADE Automated
System Competition (CASC), starting with the 2013 edition, will certainly have a
role to play driving adoption of the format. But regardless of progress in
prover technology, equipped with a concrete syntax and suitable middleware,
users can already turn their favorite automatic theorem prover into a
fairly efficient polymorphic prover.

Rank-1 polymorphism is, of course, no panacea. More advanced features, such as
type classes and dependent types, are not catered for (although type
classes can be comfortably encoded in TFF1). These are expected to be
part of a future TFF2, with the proviso that there be sufficient interest from
users and implementers.

\def\ackname{Acknowledgment}
\paragraph{\textbf{\upshape\ackname.}}
%
The present specification is largely the result of consensus among
participants of the {\tt polymorphic-tptp-tff} mailing list, especially
Fran\c{c}ois Bobot, Chad Brown, Florian Rabe, Philipp R\"ummer, Stephan Schulz,
Geoff Sutcliffe, and Josef Urban.
We are grateful to Geoff Sutcliffe, TPTP Master of Ceremonies, for giving TFF1
his benediction and adapting the TPTP BNF and other infrastructure, as well as
suggesting many textual improvements to our text.
We also thank Viktor Kuncak, Tobias Nipkow, and Nicholas Smallbone for their
support and ideas.


\bibliographystyle{splncs03}
\bibliography{tff1spec}

\end{document}
