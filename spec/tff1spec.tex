\documentclass[draft,a4paper]{llncs}

\usepackage[utf8]{inputenc} \usepackage[T1]{fontenc} \usepackage{lmodern}
\usepackage{amsmath} \usepackage{amssymb} %\usepackage{xproof}
\usepackage{stmaryrd}
\usepackage{url}
\usepackage{array}
\usepackage{cite}
\usepackage{mathptmx} %% Times New Roman (gives more space + looks less amateurish)
%\usepackage[final]{graphicx}
%\usepackage{bm}
%\usepackage{xcolor}
%\usepackage{multicol}

\urlstyle{tt}

%% Pour la redaction %%%
%\usepackage[notref,notcite]{showkeys}
%\usepackage{todonotes} %\usepackage{fixme}
%\usepackage{draftwatermark}
%\usepackage{everypage}
%\usepackage{datetime}
%\AddEverypageHook{\makebox(100,0){DRAFT : \today\ \xxivtime}}
%% Pour la redaction %%%

%\newcommand{\rinfer}[2]{\infer{#2}{#1}}
%\newcommand{\rdeduce}[2]{\deduce{#2}{#1}}

\newcommand{\limp}{\supset} \newcommand{\lequ}{\equiv}
\newcommand{\la}{\langle} \newcommand{\ra}{\rangle}
\newcommand{\eqdef}{\:\triangleq\:}

\newcommand{\FOLT}{\ensuremath{\mathbf{FOL}_\mathrm{T}}}
%\newcommand{\LKT}{\ensuremath{\mathbf{LK}_\mathrm{T}}}

\newcommand{\VT}{\mathbb{V}_\mathrm{T}}
\newcommand{\FT}{\mathbb{F}_\mathrm{T}}
\newcommand{\VI}{\mathbb{V}}
\newcommand{\FI}{\mathbb{F}}
\newcommand{\PI}{\mathbb{P}}

\newcommand{\ALL}{\forall}
%\newcommand{\ALL}{{\prod}}
%\newcommand{\ALL}{\mathrm{\Pi}}

%\newcommand{\MI}{\mathrm{MI}}
%\newcommand{\SD}{\mathrm{SD}}
\newcommand{\FV}{\mathrm{FV}}
\newcommand{\FVT}{\mathrm{FV}_\mathrm{T}}

\newcommand{\TWU}{\text{\sc Tw}}
\newcommand{\DIS}{\text{\sc Dis}}
%\newcommand{\PAR}{\text{\sc Par}}
\newcommand{\DEC}{\text{\sc Dec}}
\newcommand{\DECc}{\DEC^\circ}
\newcommand{\EXP}{\text{\sc Exp}}
\newcommand{\EXPc}{\EXP^\circ}
\newcommand{\GRD}{\text{\sc Grd}}

%\newcommand{\PIU}{\text{\sc PI}_U}

\newcommand{\UF}{W}
\newcommand{\UT}{U}

%\newcommand{\LKTU}{\ensuremath{\mathbf{LK}^U_\mathrm{T}}}

%\newcommand{\types}{\mathcal{T}(\FT,\VT)}
%\newcommand{\sorts}{\mathcal{T}(\FT)}
\newcommand{\sorts}{\mathbb{S}}
\newcommand{\univ}{\mathbb{U}}

%\newcommand{\MS}{\mathcal{M\!\!\!S}}

\DeclareMathOperator{\dom}{Dom}
%\newcommand{\dom}{\operatorname{Dom\,}}

%\newcommand{\dom}{\mathcal{D}}
%\newcommand{\cP}{\mathcal{P}}

\newcommand{\cI}{\mathfrak{I}}
\newcommand{\cM}{\mathfrak{M}}

\newcommand{\bbS}{\mathbb{S}}
\newcommand{\bbN}{\mathbb{N}}

\newcommand{\und}{\mathsf{U}}
\newcommand{\dec}{\mathsf{D}}
\newcommand{\typ}{\mathsf{T}}

\newcommand{\tsA}{\mathsf{A}}
\newcommand{\tsS}{\mathsf{S}}
\newcommand{\tsP}{\mathsf{P}}
\newcommand{\tsM}{\mathsf{M}}
\newcommand{\tsK}{\mathsf{K}}
\newcommand{\tsF}{\mathsf{F}}
\newcommand{\tsI}{\mathsf{I}}
\newcommand{\tsC}{\mathsf{C}}
\newcommand{\tsL}{\mathsf{L}}
\newcommand{\tsB}{\mathsf{B}}

\newcommand{\toty}{\mathtt{to}}
\newcommand{\fromty}{\mathtt{from}}
\newcommand{\tdeco}{\mathtt{deco}}

\newcommand{\lsA}{\mathtt{A}}
\newcommand{\lsS}{\mathtt{S}}
\newcommand{\lsK}{\mathtt{K}}
\newcommand{\lsF}{\mathtt{F}}
\newcommand{\lsR}{\mathtt{R}}
\newcommand{\lsM}{\mathtt{M}}
\newcommand{\lsget}{\mathtt{get}}
\newcommand{\lsset}{\mathtt{set}}
\newcommand{\lsnull}{\mathtt{null}}
\newcommand{\lstrue}{\mathtt{True}}
\newcommand{\lsfalse}{\mathtt{False}}
\newcommand{\lslength}{\mathtt{length}}
\newcommand{\lsfinite}{\mathtt{finite}}
\newcommand{\lsisunit}{\mathtt{isUnit}}
\newcommand{\lszero}{\mathtt{0}}
\newcommand{\lsone}{\mathtt{1}}
\newcommand{\lssix}{\mathtt{6}}
\newcommand{\lsseven}{\mathtt{7}}
\newcommand{\lsfortytwo}{\mathtt{42}}
\newcommand{\lsI}{\mathtt{I}}

\newcommand{\pty}[1]{\overline{#1}}
\newcommand{\spty}[1]{\Bar{#1}}
\newcommand{\sspty}[1]{\scriptscriptstyle{\pty{#1}}}

\DeclareMathOperator{\ty}{:}
\DeclareMathOperator{\vb}{|}


\begin{document}
\title{
TFF1: The TPTP Typed First-Order Form \\ Extended with Rank-1 Polymorphic Types}
\subtitle{Version 1.0 beta}

\author{
Andrei Paskevich\inst{1,2} \and Jasmin Christian Blanchette\inst{3}
}

\institute{
LRI, Université Paris-Sud 11, CNRS, France \and
ProVal, INRIA Saclay-Île de France, France \and
Institut für Informatik, Technische Universität München, Germany
}

\maketitle

%\vspace{-2ex}
\begin{abstract}
The TPTP (Thousands of Problems for Theorem Provers) is a library of problems
for theorem provers. TPTP defines concrete syntaxes for various logics, notably
an untyped first-order form (FOF) and a monomorphic, simply typed first-order
form (TFF0), that have become de facto standards in the automated reasoning
community. This \paper{} describes the TPTP TFF1 format, an extension of TFF
with rank-1 polymorphism. The format was designed to be easy to process
by existing reasoning tools that support ML-style polymorphism. It opens the
door for useful middleware---for example, monomorphizers and other translation
tools that encode polymorphism in FOF or TFF0. Ultimately, our hope is that TFF1
will be implemented directly in popular automatic theorem provers.
\end{abstract}

\section{Introduction}
\label{sec_intro}

The TPTP World \cite{sutcliffe-2010-world} is a well-established infrastructure
for supporting research, development, and deployment of automated reasoning
tools. It owes its name to its vast problem library, the Thousands of Problems for Theorem
Provers (TPTP) \cite{sutcliffe-2009-lib}. In addition, it specifies concrete
syntaxes for problems and solutions:
Dozens of reasoning tools implement the TPTP untyped clause normal form
(CNF) and first-order form (FOF) for classical
first-order logic with equality.

It has often been argued that the gap between the features supported by provers
and those needed by applications is too wide, and that rich interchange formats
are needed to address this disconnect.
%\cite{schumann-2001,voronkov-2003,mccune-2003,stickel-2009,kuncak-2011}.
A growing number of reasoners can process the
recently introduced TPTP ``core'' typed first-order form (TFF0) \cite{sutcliffe-et-al-2012-tff0},
with monomorphic types and interpreted arithmetic \cite{SPASS-T,vampire-arith},
or the corresponding higher-order form (THF0) \cite{benzmueller-et-al-2008-thf0};
a polymorphic version of THF0, the full THF, is in the works~\cite{sutcliffe-benzmueller-2010}.

Despite the variety of this offering, there is a strong desire in part of the automated
reasoning community for a portable \emph{polymorphic first-order format.} Many applications
require polymorphism, notably interactive theorem provers and program
specification languages; but lacking a suitable syntax, applications
and provers must communicate via monomorphic formats. To make matters worse, there is no entirely
satisfactory way to eliminate polymorphism: Monomorphization algorithms are %generally
necessarily incomplete,
and it is difficult to encode polymorphism in a complete yet
also sound and efficient manner,
especially in the presence of interpreted types
\cite{blanchette-et-al-2013-types,bobot-paskevich-2011,leino-ruemmer-2010}. Tool authors
are reduced to developing their own monomorphizers and type encodings, often
using suboptimal schemes. Polymorphism arguably belongs in
provers, where it can be implemented simply and efficiently, as demonstrated by
Alt-Ergo \cite{bobot-et-al-2008}.

\pagebreak

This paper describes the TFF1
format, an extension of TFF0 with rank-1 polymorphism.
The extension was designed with the participation of members of the TPTP
community, reflecting its needs.
Besides compatibility with TFF0 and conceptual integrity with the upcoming full
THF, an important design goal was to ensure that the format can easily be
processed by existing reasoning tools that support ML-style polymorphism. TFF1
also opens the door to useful middleware, such as monomorphizers and other
translation tools. The complete specification is available online.%
\footnote{\url{http://www21.in.tum.de/~blanchet/tff1spec.pdf}}
%
The parts that TFF1 inherits from TFF0
%, such as the concrete syntax for the logical connectives and the (optional) arithmetic constructs,
are described in the TFF0 specification \cite{sutcliffe-et-al-2012-tff0}.

\section{Syntax} \label{sec_syntax}

Briefly, the types, terms, and formulas of TFF1 are analogous to those of TFF0,
except that function and predicate symbols can be declared to be polymorphic,
types can contain type variables, and $n$-ary type constructors are allowed.
Type variables in type signatures and formulas are explicitly bound. Instances of
polymorphic symbols are specified by explicit type arguments, rather than
inferred.

\ourparagraph{Types.} The {\em types\/} of TFF1 are built from {\em type
variables\/} and {\em type constructors\/} of fixed arities.
The usual conventions of TPTP
apply: Type variables start with an uppercase letter and type constructors
with a lowercase letter. The types \verb+A+, \verb+list(A)+, \verb+list(bird)+,
and \verb+map(nat,+ \verb+list(B))+ are all examples of well-formed types.
%A type is {\em polymorphic} if it contains any type variables; otherwise, it is {\em
% monomorphic.}

As in TFF0, the type {\tt \$i} %(also called \verb+$iType+)
of individuals is predefined but has no fixed semantics, whereas the
arithmetic types
{\tt \$int}, {\tt \$rat}, and {\tt \$real} are modeled by $\mathbb{Z}$,
$\mathbb{Q}$, and $\mathbb{R}$ \cite{sutcliffe-et-al-2012-tff0}.
It is perfectly acceptable for a TFF
implementation to restrict itself to ``pure TFF\kern.06ex$k$,''
without arithmetic. %For clarity,
TFF\kern.06ex$k$ with arithmetic is sometimes labeled ``TFA\kern.06ex$k$.''

\ourparagraph{Type Signatures.}
Each function and predicate symbol occurring in a formula must be associated
with a {\em type signature\/} that specifies the types of the arguments and, for
functions, the result type. Type signatures can take any of the following forms:
%
\begin{enumerate}
\item[(a)] a type (predefined or user-defined);
\item[(b)] the Boolean pseudotype {\tt \$o}  (the result ``type'' of predicate symbols);
\item[(c)] {\tt ($\tau_1$\;*\;${\cdots}$\;*\;$\tau_n$)\;>\;$\tilde \tau$}
\ for $n > 0$, where $\tau_1,\dots,\tau_n$ are types and $\tilde \tau$ is
a type or {\tt \$o};
\item[(d)] {\tt !>[$\alpha_1$\;:\;\$tType,}\;{\tt ${\dots}$,}\;{\tt
$\alpha_n$\;:\;\$tType]:\;$\varsigma$}
\ for $n > 0$, where $\alpha_1,\dots,\alpha_n$ are distinct
type variables and $\varsigma$ has one of the previous three forms.
%\footnote{This restriction is necessary for the TPTP World's Prolog-based parser.}
%The converse is not true: Some $\alpha_i$ cannot occur in $\tau$.
\end{enumerate}
%
In accordance with TFF0, the parentheses in form (c) are omitted if $n = 1$.
The binder {\tt !>} in form (d) denotes universal quantification.
If $\varsigma$ is of form (c), it must be enclosed in parentheses.
All type variables must be bound by a {\tt !>}-binder.

Form (a) is used for monomorphic constants; form (b), for
propositional constants, including the predefined symbols {\tt \$true} and
{\tt \$false}; form (c), for monomorphic functions and predicates;
and form (d), for polymorphic functions and predicates.
%A few examples:\enskip
%(a) {\tt \$int}, {\tt monkey}, {\tt banana};\enskip
%(b) {\tt \$o};\enskip
%(c) {\tt monkey\;>\allowbreak\;banana},
%    {\tt (monkey\;*\;banana)\allowbreak\;>\allowbreak\;\$o};\enskip
%(d) {\tt !>[A\;:\;\$tType]:\;((A\;*\;list(A))\;>\;list(A))}.

%It is often
%convenient to regard all forms above as instances of the general syntax
%\begin{center}
%{\tt !>[$\alpha_1$\;:\;\$tType,\;${\dots}$,\;$\alpha_m$\;:\;\$tType]:} {\tt
%(($\tau_1$\;*\;${\cdots}$\;*\;$\tau_n$)\;>\;$\tilde \tau$)}
%\end{center}
%where $m$ and $n$ can be 0. % and $\tilde \tau$ can be any type or {\tt \$o}.

%%%% GEOFF: Add examples here.

Type variables that are bound by {\tt !>} without
occurring in the type signature's body are called \emph{\theghost{} type variables.}
These make it possible to specify operations and relations directly on types and
provide a convenient way to encode type classes.
%For example, we can declare a polymorphic propositional
%constant {\tt is\_linear} with the type signature
%{\tt !>[A} {\tt :} {\tt \$tType]:} {\tt \$o} and use it as a guard to restrict the
%axioms specifying that a binary predicate {\tt less\_eq} with the type signature
%{\tt !>[A} {\tt :} {\tt \$tType]:} {\tt ((A} {\tt *} {\tt A)} {\tt >} {\tt \$o)}
%is a linear order to those types that satisfy the {\tt is\_linear} predicate.

%A type with a quantifier prefix is called {\em polytype},
%otherwise we call it {\em monotype}.
%A monomorphic type is called {\em sort}.
%Notice that a monotype can be a polymorphic type,
%i.e.~contain type variables.

\pagebreak[3] %% TYPESETTING

\ourparagraph{Type Declarations.} Type constructors
%, like function and predicate symbols,
can optionally be declared.
The following declarations introduce a nullary type
constructor {\tt bird}, a unary type constructor {\tt list},
and a binary type constructor {\tt map}:
\begin{quote}
\verb+tff(bird, type, bird: $tType).+
\par
\verb+tff(list, type, list: $tType > $tType).+
\par
\verb+tff(map, type, map: ($tType * $tType) > $tType).+
\end{quote}
If a type constructor is used before being declared, its arity is
determined by the first occurrence. Any later declaration
must give it the same arity.

A declaration of a function or predicate symbol specifies its type signature.
Every type variable occurring in a type signature must be bound by a
{\tt !>}-binder.
The following declarations introduce a monomorphic
constant {\tt pi}, a polymorphic predicate
{\tt is\_empty}, and a pair of polymorphic functions {\tt cons} and {\tt lookup}:
\begin{quote}
\verb+tff(pi, type, pi: $real).+
\par
\verb+tff(is_empty, type, is_empty : !>[A : $tType]: (list(A) > $o)).+\kern-10mm
\par
\pagebreak[2] %% TYPESETTING
\verb+tff(cons, type, cons : !>[A : $tType]: ((A * list(A)) > list(A))).+
\par
\pagebreak[2] %% TYPESETTING
\verb+tff(lookup, type,+\\
\verb+    lookup : !>[A : $tType, B : $tType]: ((map(A, B) * A) > B)).+\kern-10mm
\end{quote}
If a function or predicate symbol is used before being declared, a
default type signature is assumed:\ {\tt (\$i\;*\;${\cdots}$\;*\;\$i)\;>\;\$i}
for functions and {\tt (\$i\;*\;${\cdots}$\;*\;\$i)\;>\;\$o} for predicates.
If a symbol is declared after its first use, the declared signature
must agree with the assumed signature.
%
If a type constructor, function symbol, or predicate symbol is declared more
than once, it must be given the same type signature up to renaming of bound
type variables.
All symbols share the same namespace.
%; in particular, a type constructor
% cannot have the same name as a function or predicate symbol.

\ourparagraph{Function and Predicate Application.} To keep the required type
inference to a minimum, every use of a polymorphic symbol must explicitly
specify the type instance. A symbol with a type signature
{{\tt !>[$\alpha_1$\;:\;\$tType,\;${\dots}$,\;$\alpha_m$\;:\;\$tType]:\;%
(($\tau_1$\;*\;${\cdots}$\;*\;$\tau_n$)\;>\;$\tilde \tau$)}}
must be applied to $m$ type arguments and $n$ term arguments. Given the above
type signatures for {\tt is\_empty}, {\tt cons}, and {\tt lookup}, the term
\hbox{\tt lookup(\$int,\;\,list(A),\;\,M,\;\,2)}
and the atom
\hbox{\tt is\_empty(\$i,\;\,cons(\$i,\;X,\;nil(\$i)))}
are well-formed and contain free occurrences of the type variable {\tt A}
and the term variables {\tt M} and {\tt X}.

In keeping with TFF1's rank-1 polymorphic nature, type variables can only be
instantiated with actual types. In particular, \verb+$o+, \verb+$tType+,
and {\tt !>}-binders cannot occur in type arguments of polymorphic symbols.

For systems that implement type inference, the following %nonstandard
extension of TFF1 might be useful. When a type argument of
a polymorphic symbol can be inferred automatically, it may be
replaced with the wildcard {\tt \$\_}. For example:
{\tt is\_empty(\$\_,\allowbreak\;\,cons(\$\_,\allowbreak\;X,\;nil(\$\_)))}.
%Although {\tt nil}'s type argument cannot be inferred from the types of
%its term arguments (since there are none), it can be
%deduced from {\texttt X}'s type.
The producer of a TFF1 problem must be aware of the type
inference algorithm implemented in the consumer to omit only redundant type
arguments.

\ourparagraph{Type and Term Variables.}
Every variable in a TFF1 formula must be bound. The variable's type must be specified
at binding time:
\begin{quote}
\begin{verbatim}
tff(bird_list_not_empty, axiom,
    ![B : bird, Bs : list(bird)]:
        ~ is_empty(bird, cons(bird, B, Bs))).
\end{verbatim}
\end{quote}

\pagebreak[3] %% TYPESETTING

\noindent
If the type and the preceding colon ({\tt :}) are omitted, the variable is given
type~{\tt\$i}. Every type variable occurring in a TFF1 formula
(whether in a type argument or in the type of a bound variable)
must also be bound, with the pseudotype {\tt\$tType}:
\begin{quote}
\begin{verbatim}
tff(lookup_update_same, axiom,
    ![A : $tType, B : $tType, M : map(A, B), K : A, V : B]:
        lookup(A, B, update(A, B, M, K, V), K) = V).
\end{verbatim}
\end{quote}

A single quantifier cluster can bind both type %variables
and term variables.
%If the type of a term variable contains a type variable, the latter must be bound
%before the former.
Universal and existential quantifiers over type variables are allowed under the
propositional connectives, including equivalence, as well as under other
quantifiers over type variables, but not in the scope of a quantifier over a
term variable, to avoid dependent types.
%Rationale: A statement of the form ``for every integer $k$, there exists a type
%$\alpha$ such that $\ldots$''\ effectively makes~$\alpha$ a dependent type.
%On such statements, type skolemization (Section~\ref{ssec:skol}) is impossible,
%and there is no easy translation to ML-style polymorphic formalisms.
%Moreover, type handling in an automatic prover would be more difficult were
%such constructions allowed, since they require paramodulation into types.

%On the other hand, all the notions and procedures described in this
%specification---except for type skolemization---are independent of this
%restriction. The rules of type checking and the notion of interpretation are
%directly applicable to unrestricted formulas. The encoding into a monomorphic
%logic (Section~\ref{sec_trans}) is sound and complete on unrestricted formulas,
%and the proofs require
%%% Theorems \ref{thm:mon_sound}~and~\ref{thm:mon_compl} require
%%% (I'm no fan of forward references)
%no adjustments. This prepares the ground for TFF2, which is expected to lift the
%restriction and support more elaborate forms of dependent types. Implementations
%of TFF1 are encouraged to support unrestricted formulas, treating them according
%to the semantics given here, if practicable.

\ourparagraph{Example.} The following problem gives the general flavor of TFF1.
It declares and axiomatizes {\tt lookup} and {\tt update} operations on
maps and conjectures that {\tt update} is idempotent for fixed keys and
values. %Its SZS status \cite{sutcliffe-2008-szs} is Theorem.

\begin{quote}
\verb+tff(map, type, map : ($tType * $tType) > $tType).+
\par
\verb+tff(lookup, type,+\\
\verb+    lookup : !>[A : $tType, B : $tType]: ((map(A, B) * A) > B)).+\kern-10mm
\par
\pagebreak[1] %% TYPESETTING
\verb+tff(update, type,+\\
\verb+    update : !>[A : $tType, B : $tType]:+\\
\verb+                 ((map(A, B) * A * B) > map(A, B))).+
\par%\medskip %% deliberate (between declarations and axioms)
\pagebreak[2] %% TYPESETTING
\verb+tff(lookup_update_same, axiom,+\\
\verb+    ![A : $tType, B : $tType, M : map(A, B), K : A, V : B]:+\\
\verb+        lookup(A, B, update(A, B, M, K, V), K) = V).+
\par
\pagebreak[1] %% TYPESETTING
\verb+tff(lookup_update_diff, axiom,+\\
\verb+    ![A : $tType, B : $tType, M : map(A, B), V : B, K : A, L : A]:+\kern-10mm\\
\verb+        (K != L => lookup(A, B, update(A, B, M, K, V), L) =+\\
\verb+                   lookup(A, B, M, L))).+
\par
\pagebreak[1] %% TYPESETTING
\verb+tff(map_ext, axiom,+\\
\verb+    ![A : $tType, B : $tType, M : map(A, B), N : map(A, B)]:+\\
\verb+        ((![K : A]: lookup(A, B, M, K) = lookup(A, B, N, K)) =>+\kern-10mm\\
\verb+         M = N)).+
\par%\medskip %% deliberate (between axioms and conjecture)
\pagebreak[2] %% TYPESETTING
\verb+tff(update_idem, conjecture,+\\
\verb+    ![A : $tType, B : $tType, M : map(A, B), K : A, V : B]:+\\
\verb+        update(A, B, update(A, B, M, K, V), K, V) =+\\
\verb+        update(A, B, M, K, V)).+
\end{quote}

\section{Type Checking and Semantics} \label{sec_semantics}

\ourparagraph{Notation.} Starting with this section, we use standard
mathematical notation to write types, terms, and formulas. Our conventions
for metavariables are summarized below.
%
\begin{center}
\begin{tabular}{l@{\enskip}l@{\qquad\quad}l@{\enskip}l}
Type variables: & $\alpha,$ $\beta$ &
  Function symbols: & $f,$ $g$ \\
Type constructors: & $\kappa$ &
  Predicate symbols: & $p,$ $q$ \\
Types: & $\sigma$, $\tau$ &
  Terms: & $s$, $t$ \\
Term variables: & $u,$ $v$ &
  Formulas: & $\varphi,$ $\psi$
\end{tabular}
\end{center}
%
%We write the names of type constructors, function, and predicate symbols using a
%fixed-width font:\ {\tt list}, {\tt map}, {\tt cons}, {\tt set}.
%% @ANDREI: There were very few examples of this in the rest of the document (lessEq and isOrdered) and
%% it was obvious from the context what they were. Also, I'm trying italics for those, to mix
%% better with the maths; but I'm using the macro \sym, so it's easy to change.
Possibly empty lists of types and terms are denoted by $\bar{\tau}$ and
$\bar{t}$, respectively.

We use the symbols ${\times}$, ${\to}$, and ${\ALL}$
%\footnote{We can also use ${\prod}$, but should we,
%given that there are no dependent types in TFF1?}
to write type signatures, and write $\omicron$ (lowercase omicron) for the
Boolean pseudotype {\tt \$o}. It is convenient to treat
equality ($\eq$), negation ($\lnot$), conjunction ($\rand$), and universal
quantification ($\forall$) as logical symbols and regard disequality
($\not\eq$), disjunction ($\lor$), implication ($\limp$), reverse implication
($\revlimp$), equivalence ($\lequ$), inequivalence ($\inlequ$), and existential
quantification ($\exists$) as abbreviations.
%The symbols $\top$ and $\bot$ stand for ``true'' and ``false'',
%respectively.

Equality can be seen as a polymorphic predicate with the signature
$\forall\alpha.\; \alpha\times\alpha\to\omicron$, but the type instance is
implicitly specified by the type of either argument, instead of explicitly via a
type argument. Hence, it is preferable to consider it a logical symbol.

The set of type variables occurring freely in a formula $\varphi$
(in the type arguments of polymorphic symbols or
in the types of bound variables) is denoted by $\FVT(\varphi)$.
The set of free term variables of $\varphi$
is denoted by $\FV(\varphi)$. The formula $\varphi$ is {\em closed\/}
if both $\FVT(\varphi)$ and $\FV(\varphi)$ are empty.

A {\em type substitution\/} $\rho$ is a mapping of type variables
to types. A {\em monomorphic\/} type substitution
maps every type variable either to itself or to a monomorphic type.

The expression $f[x \mapsto a]$ denotes the function that maps $x$ to $a$ and
every other element $y$ in $f$'s domain to $f(y)$.

%
%A type $\tau$ is said to {\em match\/} another type $\tau'$ whenever there exists a
%type substitution~$\rho$ such that $\tau\rho = \tau'$.
%We reserve $\rho$ for type substitutions.
%
%The operator $\circ$ composes two substitutions:\
%$\tau(\rho \circ \rho') \eqdef (\tau\rho)\rho' = \tau\rho\rho'$.

\ourparagraph{Type Checking.}
Let $\GAM$ be a {\em type context}, a function that maps
every variable symbol to a type.
A type judgment $\GAM \,\vdash t \ty \tau$ expresses that the term $t$
is {\em well-typed\/} and has type $\tau$ in context~$\GAM$.
A type judgment $\GAM \,\vdash \varphi \ty \omicron$ expresses that the formula
$\varphi$ is {\em well-typed\/} in~$\GAM$.
We write $f \ty \forall \alpha_1\dots\alpha_m .\;
\tau_1 \times \dots \times \tau_n \to \tau$ and
$p \ty \forall \alpha_1\dots\alpha_m .\;
\tau_1 \times \dots \times \tau_n \to \omicron$ to specify
type signatures of function and predicate symbols,
where $m$ and $n$ can be 0.

The typing rules of TFF1 are given below:
%in Figure~\ref{fig:typing}.

%% TYPESETTING: cheat
\vskip.25\abovedisplayshortskip
\vskip.75\abovedisplayskip

\centerline{$\displaystyle \ourfrac{}{\GAM \,\vdash\, u \ty \GAM(u)}$}

\vskip3\smallskipamount

\centerline{$\displaystyle
\ourfrac{f \ty \forall \alpha_1\dots\alpha_m .\;
    \tau_1 \times \dots \times \tau_n \to \tau
\qquad
\GAM \,\vdash\, t_1 \ty \tau_1\,\rho
\quad\cdots\quad
\GAM \,\vdash\, t_n \ty \tau_n\,\rho
}{\GAM \,\vdash\,
f(\alpha_1\,\rho,\>\dots,\>\alpha_m\>\rho,\>t_1,\>\dots,\>t_n) \ty \tau\rho}
$}

\vskip3\smallskipamount

\centerline{$\displaystyle
\ourfrac{p \ty \forall \alpha_1\dots\alpha_m .\;
    \tau_1 \times \dots \times \tau_n \to \omicron
\qquad
\GAM \,\vdash\, t_1 \ty \tau_1\,\rho
\quad\cdots\quad
\GAM \,\vdash\, t_n \ty \tau_n\,\rho
}{\GAM \,\vdash\,
p(\alpha_1\,\rho,\>\dots,\>\alpha_m\>\rho,\>t_1,\>\dots,\>t_n) \ty \omicron}
$}

\vskip3\smallskipamount

\centerline{$\displaystyle
\ourfrac{\GAM \,\vdash\, s \ty \tau \qquad \GAM \,\vdash\, t \ty \tau}{
\GAM \,\vdash\, s \eq t \ty \omicron}$\qquad$\displaystyle
\ourfrac{\GAM \,\vdash\, \varphi \ty \omicron \qquad
\GAM \,\vdash\, \psi \ty \omicron}{
\GAM \,\vdash\, \varphi \rand \psi \ty \omicron}$}

\vskip3\smallskipamount

\centerline{$\displaystyle
\ourfrac{\GAM \,\vdash\, \varphi \ty \omicron}{
\GAM \,\vdash\, \lnot\, \varphi \ty \omicron}\qquad\qquad
\ourfrac{\GAM[u \mapsto \tau] \,\vdash\, \varphi \ty \omicron}{
\GAM \,\vdash\, \forall u \mathbin{\ty} \tau .\; \varphi \ty \omicron}\qquad\qquad
\ourfrac{\GAM \,\vdash\, \varphi[\alpha'\!/\alpha] \ty \omicron}{
\GAM \,\vdash\, \forall \alpha .\; \varphi \ty \omicron}$}

\vskip\belowdisplayskip

%\caption{Typing rules of TFF1}
%\label{fig:typing}
%\end{figure}

%We write $\GAM[u \mapsto \tau]$ to denote a type context that maps
%variable $u$ to type $\tau$ and every other variable $v$ to $\GAM(v)$.
In the last rule, $\alpha'$ is an arbitrary type variable that
occurs neither in $\varphi$ nor in the values of~$\GAM$.
The renaming is necessary to reject formulas such as
$\forall\alpha.\:\forall u\ty\alpha.\:\forall\alpha.\:\forall v\ty\alpha.\;
u \eq v$, where the types of $u$ and $v$ are actually different.
To simplify the subsequent definitions, we assume from now on
that no type variable can be both free and bound in the same formula;
we call this the {\em no-clash assumption}.
%Consequently, nested quantifiers on the same type
%variable are not allowed, either.
Then we can do without explicit renaming, % of type variables,
and the last typing rule's premise becomes
$\GAM \,\vdash\, \varphi \ty \omicron$.

A closed TFF1 formula $\varphi$ is {\em well-typed\/} if and only if
the judgment $\GAM \vdash \varphi \ty \omicron$ is derivable
for any $\GAM$.
Obviously, if a closed formula is
well-typed in one type context, it is well-typed in any other one;
hence, we omit $\GAM$ and write
${} \vdash \varphi \ty \omicron$.
Closed well-typed formulas are called {\em sentences}.

\ourparagraph{Semantics.}
The semantics of TFF1 postulates a nonempty collection $\sorts$ of
nonempty sets, the {\em domains}. The union of all domains is called the
{\em universe}, %denoted by
$\univ$.
An interpretation $\cI$ for a given set of type constructors,
function symbols, and predicate symbols is constructed as follows.

An $n$-ary type constructor $\kappa$ is interpreted as a function
$\kappa^{\cI} : \sorts^{\,n} \to \sorts$.
Let $\theta$ be a {\em type valuation}, a function that maps every
type variable to a domain. Types are evaluated according to the following
equations:
\begin{align*}
\evalT{\alpha} &\eqdef \theta(\alpha) &
\evalT{\kappa(\tau_1,\dots,\tau_n)} &\eqdef \kappa^{\cI}\bigl(\evalT{\tau_1},\,\dots,\,\evalT{\tau_n}\bigr)
\end{align*}
Since type evaluation depends only on the values of $\theta$
on the type variables occurring in a type, we write $\evalG{\tau}$
to denote the domain of a monomorphic type $\tau$.
We use the notation $[\alpha_1\,\mapsto\,D_1,\,\dots,\,\alpha_m\,\mapsto\,D_m]$
to specify $\theta$ for types whose free type variables are among
$\alpha_1,\dots,\alpha_m$.

A predicate symbol $p : \forall \alpha_1\dots\alpha_m .\; \tau_1 \times \dots \times \tau_n
\to \omicron$ is interpreted as a relation
$\smash{p^{\cI}} \subseteq \sorts^m \times \univ^n$.
A function symbol $f : \forall \alpha_1\dots\alpha_m .\; \tau_1 \times \dots \times \tau_n \to \tau$
is interpreted as a function
$f^{\cI}$ on $\sorts^m \times \univ^n$ that
maps any $m$ domains $D_1,\dots,D_m$ and
$n$ universe elements to an element of
$\evalG{\tau}_\theta$, where $\theta$ maps each $\alpha_i$ to $D_i$.
%$\evalG{\tau}_{[\alpha_1\,\mapsto\,D_1,\,\dots,\,\alpha_m\,\mapsto\,D_m]}$.
%Recall that every type variable occurring in a type signature
%must be bound, and therefore every type variable in $\tau$ belongs
%to $\{ \alpha_1,\dots,\alpha_m \}$.

Let $\xi$ be a {\em variable valuation}, a function that assigns
to every variable an element of $\univ$. TFF1 terms
and formulas are evaluated according to the following equations:
\begin{align*}
\eval{u} &\eqdef \xi(u) &
  \eval{\lnot\, \varphi} &\eqdef \lnot\, \eval{\varphi} \\
\eval{f(\bar{\tau},\,\bar{t}\,)} &\eqdef f^\cI\bigl(\evalT{\bar{\tau}},\eval{\bar{t}\,}\bigr) &
  \eval{\varphi \rand \psi} &\eqdef \eval{\varphi} \rand \eval{\psi} \\
\eval{p(\bar{\tau},\,\bar{t}\,)} &\eqdef p^\cI\bigl(\evalT{\bar{\tau}},\eval{\bar{t}\,}\bigr) &
  \eval{\forall u \ty \tau .\; \varphi} &\eqdef \forall {a \mathbin\in \evalT{\tau}} .\;
  \smash{\evalG{\varphi}_{\theta,\,\xi[u\,\mapsto\,a]}} \\
\eval{t_1 \eq t_2} &\eqdef \bigl(\eval{t_1} = \eval{t_2}\bigr) &
  \eval{\forall \alpha .\; \varphi} &\eqdef \forall {D \in \sorts} .\;
  \smash{\evalG{\varphi}_{\theta[\alpha\,\mapsto\,D],\,\xi}}
% \\
%\eval{\forall u \ty \tau .\; \varphi} &\eqdef \!\!\!\!
%\bigwedge_{a \in \evalT{\tau}} \!
%\evalG{\varphi}_{\theta,\,\xi[u\,\mapsto\,a]} &
%\eval{\forall \alpha .\; \varphi} &\eqdef
%\bigwedge_{D \in \sorts}\;
%\evalG{\varphi}_{\theta[\alpha\,\mapsto\,D],\,\xi}
\end{align*}
%The expression $\xi[u \mapsto a]$ stands for the function
%that maps $u$ to $a$ and every other variable $v$ to $\xi(v)$, and likewise for
%$\theta[\alpha \mapsto D]$.
%Likewise, $\theta[\alpha \mapsto D]$ is the function that maps
%type variable $\alpha$ to domain $D$ and every other type
%variable $\beta$ to $\theta(\beta)$.
We omit irrelevant subscripts and write
$\evalG{\varphi}$ to denote the evaluation of a sentence.
%closed formula. (for TYPESETTING)

A sentence $\varphi$ is {\em true} in an interpretation $\cI$,
written $\cI \models \varphi$, if and only if $\evalG{\varphi}$ is true.
The interpretation $\cI$ is then a {\em model\/} of $\varphi$.
A sentence that has a model is {\em satisfiable}.
A sentence that is true in every interpretation is {\em valid}.
These notions are extended to sets and sequents of TFF1 sentences in
the usual way.

\section{Translations} \label{sec:trans}

\subsection{Encoding into a Many-Sorted Logic} \label{ssec:tff0}

We describe here a simple translation from TFF1 to a traditional
many-sorted first-order logic. More sophisticated encodings which
preserve selected types (by translating them into separate sorts)
or do not change the propositional structure of encoded formulas
are described in \cite{leino10tacas,bobot11frocos}.
Notice that both papers assume preliminary elimination
of ghost type arguments (cf.~Section~\ref{ssec:ghost});
the latter also assumes preliminary type skolemization
(cf.~Section~\ref{ssec:skol}).

Let $\Delta$ be a set of TFF sentences, that is, closed and well-typed
formulas. We construct an equisatisfiable set of monomorphic two-sorted
formulas $\MON(\Delta)$ as follows.
%
We introduce two sorts, $\typ$ and $\und$.
To every type variable $\alpha$ in $\Delta$ we assign
a fresh variable $\hat{\alpha}$ of sort $\typ$.
To every ordinary variable $u$ we assign
a fresh variable $\hat{u}$ of sort $\und$.
To every type constructor $K$ of arity $n$ we assign
a function symbol $\hat{K}$ of sort signature $\typ^n \to \typ$.
To every function symbol $f$ of type signature
$\forall \alpha_1\dots\alpha_n \,.\, T_1 \times \dots \times T_m \to T$
we assign a function symbol $\hat{f}$ of sort signature
$\typ^n \times \und^m \to \und$.
To every predicate symbol $p$ of type signature
$\forall \alpha_1\dots\alpha_n \,.\, T_1 \times \dots \times T_m \to \omicron$
we assign a predicate symbol $\hat{p}$ of sort signature
$\typ^n \times \und^m \to \omicron$.
Finally, we add a new function symbol $\typeof$ of sort signature
$\und \to \typ$.

The $\MON$ transformation translates TFF types, terms, and formulas
according to the following equations ($\bar{S}$ and $\bar{t}$ denote
sequences of types and terms, respectively):
\begin{align*}
\MON(\alpha) &\eqdef \hat{\alpha} &
\MON(K(\bar{S})) &\eqdef \hat{K}(\MON(\bar{S})) \\
\MON(u) &\eqdef \hat{u} &
\MON(f(\bar{S},\bar{t})) &\eqdef \hat{f}(\MON(\bar{S}),\MON(\bar{t})) \\
\MON(t_1 \approx t_2) &\eqdef \MON(t_1) \approx \MON(t_2) &
\MON(p(\bar{S},\bar{t})) &\eqdef \hat{p}(\MON(\bar{S}),\MON(\bar{t})) \\
\MON(\lnot F) &\eqdef \lnot\, \MON(F) &
\MON(F \land G) &\eqdef \MON(F) \land\, \MON(G) \\
\MON(\forall u \ty T .\, F) &\eqdef
\forall \hat{u} \,.\,
\typeof(\hat{u}) \approx \MON(T) \limp \MON(F) &
\MON(\forall \alpha \,.\, F) &\eqdef
\forall \hat{\alpha} \,.\, \MON(F)
\end{align*}

Consider a function symbol $f$ of signature
$\forall \alpha_1\dots\alpha_n \,.\, T_1 \times \dots \times T_m \to T$.
The {\em typing axiom} for $f$ is a formula
$$
\forall \hat{\alpha}_1\dots\hat{\alpha}_n \,.\,
\forall \hat{v}_1\dots\hat{v}_m \,.\,
\typeof(\hat{f}(\hat{\alpha}_1,\dots,\hat{\alpha}_n,
\hat{v}_1,\dots,\hat{v}_m)) \approx \MON(T).
$$
We denote with $\mathrm{Ax}_{\Delta}$ the set of typing axioms
for all function symbols occurring in $\Delta$. Then we define
$$
\MON(\Delta) \eqdef \{ \MON(F) \vb F \in \Delta \} \cup
\mathrm{Ax}_{\Delta}.
$$

It is easy to see that $\MON$ converts TFF types into well-formed
terms of sort $\typ$, TFF terms into well-formed terms of sort $\und$,
and TFF formulas (not necessarily well-typed) into well-formed formulas.

\begin{theorem} \label{thm:mon_sound}
If a set of sentences $\Delta$ is satisfiable in TFF,
then $\MON(\Delta)$ is satisfiable in classical
many-sorted first-order logic with equality.
\end{theorem}

\begin{theorem} \label{thm:mon_compl}
Given a set of TFF sentences $\Delta$, if $\MON(\Delta)$
is satisfiable in classical many-sorted first-order logic
with equality, then $\Delta$ is satisfiable in TFF.
\end{theorem}

\subsection{Elimination of Ghost Type Arguments} \label{ssec:ghost}
\subsection{Type Skolemization} \label{ssec:skol}



\section{Preprocessing of Type Quantifiers}

We describe two preprocessing steps that preserve both satisfiability and
unsatisfiability of TFF1 problems. They eliminate ghost type variables in type
signatures and alternating $\forall\raise.15ex\hbox{\ensuremath{\slash}}\vthinspace\exists$ type quantifier prefixes in
formulas, two features that are not directly supported by ML-style formalisms
(which otherwise are a good match for TFF1).

\subsection{Elimination of Ghost Type Arguments} \label{ssec:ghost}

%\footnote{Other relevant citations, please? What about Mizar?}.
% @ANDREI: I'd rather stick to "ML-style formalisms" here. Mizar is based on
% set theory and has at most soft types; it would be better treated in the
% Related Work section.

ML-style formalisms, as implemented in Alt-Ergo~\cite{conchon08smt},
Boogie~\cite{Barnett06boogie}, HOL \cite{gordon-melham-1993}, HOL
Light~\cite{harrison-1996}, Isabelle\slash HOL~\cite{nipkow-et-al-2002},
Why3~\cite{boogie11why3}, and several other systems, allow type variables
in type signatures, but without explicit $\forall$-binders.
Function and predicate symbols do not take explicit type
arguments; instead, the concrete instance of the symbol's type signature is
determined by the types of its term arguments, the type of the result, and
optional type annotations inside terms.

The natural translation from TFF1 to such a formalism would map $\forall
\alpha\beta.\; \alpha\times\beta\to\omicron$ (and $\forall \beta\alpha.\;
\alpha\times\beta\to\omicron$) to $\alpha\times\beta\to\omicron$, simply
omitting any $\forall$-binders. To compensate for the missing type arguments,
type annotations are sometimes needed to guide the Hindley--Milner type
inference. This is slightly awkward but not difficult to implement.

The main difficulties arise in conjunction with ghost type variables: If
$\forall\alpha.\;\omicron$ collapses to $\omicron$, the dependency on the type
is lost.
%
To ease the adoption of TFF1 in such systems, we suggest a preprocessing step
that eliminates ghost type variables. This step is lightweight, as it requires
only the introduction of one term argument per ghost type. In particular,
it is the identity for formulas that do not rely on ghost type variables.

%that is, the type variables in the symbol's type signature that
%do not appear in the quantifier-free matrix.

Let $\DEL$ be a set of sentences.
We construct an equisatisfiable set $\ELI(\DEL)$ as follows.
%
We introduce a special unary type constructor $\ghost$.
We replace every function symbol $f : \forall \alpha_1\dots\alpha_m .\; \tau_1 \times \dots \times \tau_n \to \tau$
with a new function symbol $\hat{f} : \forall \alpha_1\dots\alpha_m .\;
\ghost(\alpha_{i_1}) \times \dots \times \ghost(\alpha_{i_r}) \times
\tau_1 \times \dots \times \tau_n \to \tau$,
where $\alpha_{i_1},\dots,\alpha_{i_r}$ are the type
variables from the quantifier prefix that do not occur in
$\tau_1,\dots,\tau_n$ or $\tau$.
Similarly, every predicate symbol $p \,\ty\,
\forall \alpha_1\dots\alpha_m .\;\allowbreak \tau_1 \times \dots \times \tau_n \to \omicron$
is replaced with a new predicate symbol $\hat{p} \,\ty\,
\forall \alpha_1\dots\alpha_m .\;\allowbreak
\ghost(\alpha_{i_1}) \times \dots \times \ghost(\alpha_{i_r}) \times
\tau_1 \times \dots \times \tau_n \to \omicron$,
where $\alpha_{i_1},\dots,\alpha_{i_r}$ are the type
variables from the quantifier prefix that do not occur in
$\tau_1,\dots,\tau_n$.
Finally, we add a ``witness'' constant $\wit : \forall \alpha.\;\ghost(\alpha)$.
The $\ELI$ transformation translates terms and atomic
formulas according to the following equations:
%\begin{align*}
%\ELI(u) & \eqdef u &
%\ELI(s \eq t) & \eqdef \ELI(s) \eq \ELI(t)
%\end{align*}
\begin{align*}
\ELI(u) \eqdef u
\kern1.4em & \kern5.55em %% TYPESETTING
\ELI(s \eq t) \eqdef \ELI(s) \eq \ELI(t) \\
\ELI(f(\sigma_1,\dots,\sigma_m,\bar t\,)) &\eqdef
\hat{f}(\sigma_1,\dots,\sigma_m,\wit(\sigma_{i_1}),\dots,\wit(\sigma_{i_r}),
\ELI(\bar t\,)) \\
\ELI(p(\sigma_1,\dots,\sigma_m,\bar t\,)) &\eqdef
\hat{p}(\sigma_1,\dots,\sigma_m,\wit(\sigma_{i_1}),\dots,\wit(\sigma_{i_r}),
\ELI(\bar t\,))
\end{align*}
Given a formula $\varphi$ or a set of sentences $\DEL$,
$\ELI(\varphi)$ and $\ELI(\DEL)$ denote the result of
applying $\ELI$ to every atomic formula in $\varphi$ and
$\DEL$, respectively.

\begin{theorem} \label{thm:eli}
Any $\DEL$ is equisatisfiable to $\ELI(\DEL)$.
\end{theorem}
\begin{proof}
A model of $\DEL$ can be converted to a model of $\ELI(\DEL)$
as follows. We choose an arbitrary value $e$ and take the singleton
$\{ e \}$ as the domain of $\ghost(D)$ for any domain $D$.
Accordingly, $\wit$ evaluates to $e$ on any argument.
Interpretations of other function and predicate symbols are
adjusted in the obvious way.

To convert a model of $\ELI(\DEL)$
to a model of $\DEL$, we evaluate any functional symbol
$f$ on $D_1,\dots,D_m,a_1,\dots,a_n$ exactly as
$\hat{f}$ on $D_1,\dots,D_m,e_1,\dots,e_r,a_1,\dots,a_n$,
where each $e_k$ is the evaluation of $\wit$ on $D_{i_k}$, and similarly
for predicate symbols.
\qed
\end{proof}

\subsection{Type Skolemization} \label{ssec:skol}

Another feature of TFF1 that is not universally supported
by systems with polymorphic types is explicit
quantification over type variables.
For example, while Boogie and Coq \cite{CoqManualV8}
allow quantifiers over types in their logics,
HOL and other ML-style formalisms provide no such syntax:
Instead, they consider all type variables
implicitly universally quantified at the top of a formula, which is similar to
the treatment of term variables in TPTP CNF clauses.
A simple and practical solution to translate TFF1 problems to prepare problems
for ML-style logics is type skolemization. This is slightly involved because
type quantifiers may appear under equivalence, where they are unpolarized.

We introduce two transformations, $\SKO^-$ and $\SKO^+$, to perform type
skolemization in premises and in goals, respectively. (The latter is more
correctly called ``type herbrandization.'')
%Let us call a formula $\varphi$ {\em dependently typed} whenever
%it contains a quantifier over a type variable in the scope
%of a quantifier over a term variable%
%\footnote{We might take polarities into account, but then
%the proof of Theorem~\ref{thm:sko} wouldn't be so sweet.
%Types-under-values are nasty anyway.}.
Their definitions follow:
\begin{align*}
\SKO^-(p(\bar{\sigma},\bar{t}\,)) &\eqdef p(\bar{\sigma},\bar{t}\,) &
\SKO^+(p(\bar{\sigma},\bar{t}\,)) &\eqdef p(\bar{\sigma},\bar{t}\,) \\
\SKO^-(t_1 \eq t_2) &\eqdef t_1 \eq t_2 &
\SKO^+(t_1 \eq t_2) &\eqdef t_1 \eq t_2 \\
\SKO^-(\forall u \ty \tau .\; \varphi) &\eqdef \forall u \ty \tau .\; \varphi &
\SKO^+(\forall u \ty \tau .\; \varphi) &\eqdef \forall u \ty \tau .\; \varphi \\
\SKO^-(\varphi \rand \psi) &\eqdef \SKO^-(\varphi) \rand \SKO^-(\psi) &
\SKO^+(\varphi \rand \psi) &\eqdef \SKO^+(\varphi) \rand \SKO^+(\psi) \\
\SKO^-(\lnot\, \varphi) &\eqdef \lnot\, \SKO^+(\varphi) &
\SKO^+(\lnot\, \varphi) &\eqdef \lnot\, \SKO^-(\varphi) \\
\SKO^-(\forall \alpha .\; \varphi) &\eqdef \forall \alpha .\; \SKO^-(\varphi) &
\SKO^+(\forall \alpha .\; \varphi) &\eqdef
\SKO^+(\varphi[\kappa(\bar\beta)/\alpha])
\end{align*}
where $\kappa$ is a fresh type constructor and $\bar\beta$ are
the free type variables of $\forall \alpha .\; \varphi$.
Recall that TFF1 forbids quantifiers over type variables in the scope
of a quantifier over a term variable; otherwise, Skolem type constructors
would have to take term variables as arguments. This explains why $\SKO^-$ and
$\SKO^+$ simply stop at the outermost term quantifier.

Given a set of TFF1 sentences $\DEL$,
$\SKO^-(\DEL)$ and $\SKO^+(\DEL)$ denote the result of applying
the corresponding transformations to every formula in $\DEL$.

\begin{theorem} \label{thm:sko}
Any $\DEL$ is equisatisfiable to $\SKO^-(\DEL)$.
\end{theorem}
\begin{proof}
It is easy to see that
skolemizing the $\typ$-sorted variables in $\MON(\DEL)$ gives
exactly $\MON(\SKO^-(\DEL))$ modulo renaming of Skolem symbols and
permutation of their arguments. Theorems~\ref{thm:mon_sound} and
\ref{thm:mon_compl} conclude the proof.
\qed
\end{proof}

%%% @ANDREI: Which restriction are you talking about? I'm not following you here:
%
%Given a TFF1 problem that does not satisfy the aforementioned restriction,
%one can submit it to a system like Isabelle/HOL or Alt-Ergo by encoding it
%with $\MON$ or any other sound translation into monomorphic logic.

\section{Applications}
\label{sec_apps}

A number of applications already support TFF1. Geoff
Sutcliffe has extended the TPTP World infrastructure to process TFF1 problems
and solutions. This involved adapting the Backus--Naur form specification of the
TPTP syntaxes, from which parsers are generated.%
\footnote{\url{http://www.cs.miami.edu/~tptp/TPTP/SyntaxBNF.html}}
Some TPTP tools still need to be ported to TFF1; this is ongoing work.

%%% TODO: update: grep "pure"
The Why3 \cite{bobot-et-al-2011} environment, which defines its own ML-like
polymorphic specification language, can parse pure TFF1. Why3 translates
between TFF1 and a wide range of
formats, including FOF, SMT-LIB, and
Alt-Ergo's native syntax.
In addition, Why3's TFF1 parser is being ported to
Alt-Ergo \cite{bobot-et-al-2008}, so that it can directly process TFF1. % problems.
%without taking a detour through Why3.
%Code sharing is facilitated here by the use of
%a common implementation language (OCaml).

HOL(y)Hammer \cite{kaliszyk-urban-2013}
and Sledgehammer \cite{paulson-blanchette-2010}
integrate various automatic provers
in the proof assistants HOL~Light and Isabelle\slash HOL. They have been
extended to output pure TFF1 problems for Alt-Ergo and Why3
(in addition to FOF, TFF0, and THF0).
For the first time, Sledgehammer
exploits the polymorphic potential of these tools---without having
to implement their (incompatible) native file formats.
Moreover, using Sledgehammer, we produced 987 problems to populate the TPTP
library.%
\footnote{\url{http://www.cs.miami.edu/~tptp/TPTP/Proposals/TFF1.html}}
%PFF (``\relax{P}olymorphic T\relax{FF}'')
%domain of the TPTP library.
By extending the tool with a TFF1 parser,
we hope to transform it into a versatile translator to FOF and
TFF0.

%Other tools, notably an LF-based type checker similar to TwHelFTC for THF0, are
%in development.

%  * ToFoF, Monotonox, other -ox tools?

%  * TPTP Library:
%     * accepts all problems
%     * populated it with problems generated by Sledgehammer and Why3
%       (and Alt-Ergo?)

\section{Related Work}
\label{sec_related}

\ourparagraph{Formalisms.}
The TPTP family of formats is well established in the automated reasoning
community. As part of the MPTP project \cite{urban-2006}, Urban designed private
extensions of the TPTP FOF syntax with dependent types to
accommodate Mizar, as well as translations; TFF2 is expected to subsume this work. Also
related to TFF1 is the full THF syntax \cite{sutcliffe-benzmueller-2010}, which
is not yet finalized or implemented. Like Urban's extensions, full THF supports
dependent types.
        %most reasoning tools out there are first-order, and the higher-order syntax is
        %sufficiently unwieldy that it's unrealistic to expect apps to support it;
        %furthermore, the form of polymorphism offered there is much more general than
        %what we expect to need in the first-order setting.

For SMT solvers, the SMT-LIB 2 format \cite{barrett-et-al-2010} specifies a
classical many-sorted logic with equality and interpreted arithmetic, much in
the style of TFF0 but with parametric symbol declarations (overloading).
Polymorphism would make sense there as well, as witnessed by Alt-Ergo
\cite{bobot-et-al-2008}.
% provides its own polymorphic format with a syntax very close to ML.
However, the SMT community is still recovering from the major
upgrade to SMT-LIB 2 and busy defining a standard proof format
\cite{besson-et-al-2011}; implementers would likely not welcome yet another
feature at this point. Nonetheless, with its support for arithmetic, TFF1 is a
reasonable format to implement in an SMT solver if polymorphism is desired.

In the world of interactive theorem proving, polymorphism is the norm rather
than the exception. HOL systems \cite{gordon-melham-1993} provide simple types
with top-level, ML-style (rank-1) polymorphism, while Coq
\cite{bertot-casteran-2004} supports dependent types and higher-rank
polymorphism.

The intermediate verification language and tool Boogie 2 \cite{leino-ruemmer-2010}
supports a restricted form of higher-rank polymorphism (due to its polymorphic maps),
and its cousin Why3 \cite{bobot-et-al-2011}
provides rank-1 polymorphism.% Both provide TPTP and SMT-LIB backends.

\ourparagraph{Encodings.}
Early descriptions of type encodings are due to Enderton
\cite[\S4.3]{enderton-1972}, Stickel \cite[p.~99]{stickel-1986}, and Wick and
McCune \cite[\S4]{wick-mccune-1989}. TFF1 type arguments are reminiscent of
System~F; a FOF-based encoding that exploits them is described by Meng and
Paulson \cite{meng-paulson-2008-trans}, who also present a translation of
axiomatic type classes.

Considerable progress has been made lately toward sound, complete, and efficient
encodings of polymorphic logics in untyped or monomorphic logics. Leino and
R\"ummer \cite{leino-ruemmer-2010} present a translation of higher-rank
polymorphism, including explicit quantifiers over types, into the many-sorted
SMT-LIB syntax, while preserving interpreted types. They also show how to
exploit SMT triggers to prevent unsound variable instantiations in a
translation based on type arguments. Bobot and Paskevich
\cite{bobot-paskevich-2011} extend earlier work by Couchot and Lescuyer
\cite{couchot-lescuyer-2007} to encode polymorphism while preserving
arbitrary monomorphic types (e.g., array types).
Blanchette et
al.\ \cite{blanchette-et-al-2012-mono}, building on work by Claessen et al.\
\cite{claessen-et-al-2011}, present a %particularly
lightweight encoding of
polymorphic types that exploits type monotonicity. All of these translations
assume %preliminary
elimination of ghost type arguments
(Section~\ref{ssec:ghost}); the last two also assume type skolemization
(Section~\ref{ssec:skol}).

An easy approach to pass TFF1 problems can be passed to TFF0
reasoning tools is to monomorphize the problem first---that is, to heuristically
instantiate all type variables with appropriate
ground types. Monomorphization is incomplete and often overlooked
or derided in the literature \cite[p.\ 3]{couchot-lescuyer-2007}, but it was
applied with much success in Sledgehammer \cite[\S6]{blanchette-et-al-2012-mono}
and Why3.
%% @ANDREI: Reference for Why3?

\section{Conclusion}
\label{sec_concl}

This paper described the TPTP TFF1 format, an extension of the monomorphic TFF0 format
with rank-1 polymorphism. The new format nicely complements the existing TPTP
offerings. %, taking over where TFF0 left off.
For reasoning tools that already
support polymorphism, TFF1 is a portable alternative to the existing ad hoc
syntaxes. But more importantly, the format is a vehicle to foster native
polymorphism support in automatic theorem provers. The time is ripe: After years
upon years of untyped reasoning, the last decade witnessed the rise of
interpreted arithmetic embedded in restrictive monomorphic type systems. TFF1
lifts some of the most obvious restrictions of these type systems.

TFF1 is part of TPTP World. The TPTP library already contains
nearly a thousand TFF1 problems, and although the format is in its
infancy, it is supported by several applications, including the SMT solver
Alt-Ergo (via Why3). Given that many applications today require polymorphism, it
is likely that other reasoning tools will gradually follow suit.
The annual CADE Automated
System Competition (CASC), starting with the 2013 edition, will certainly have a
role to play driving adoption of the format. But regardless of progress in
prover technology, equipped with a concrete syntax and suitable middleware,
users can already turn their favorite automatic theorem prover into a
polymorphic prover.

Rank-1 polymorphism is, of course, no panacea. More advanced features, such as
type classes and dependent types, are not catered for (although type
classes can be comfortably encoded in TFF1). These are expected to be
part of a future TFF2, with the proviso that there be sufficient interest from
users and implementers.

\def\ackname{Acknowledgment}
\paragraph{\textbf{\upshape\ackname.}}
%
The present specification is largely the result of consensus among
participants of the {\tt polymorphic-tptp-tff} mailing list, especially
Fran\c{c}ois Bobot, Chad Brown, Florian Rabe, Philipp R\"ummer, Stephan Schulz,
Geoff Sutcliffe, and Josef Urban.
We are grateful to Geoff Sutcliffe, TPTP Master of Ceremonies, for giving TFF1
his benediction and adapting the TPTP BNF and other infrastructure.
We also thank Viktor Kuncak, Tobias Nipkow, and Nicholas Smallbone for their
support and ideas.


\bibliographystyle{splncs03}
\bibliography{tff1spec}

\end{document}
