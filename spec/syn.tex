%%% TODO: "If a symbol’s signature is declared later to be different from an assumed signature, that is an error."
%%% TODO: "Observe that the TFF syntax makes it impossible to overload predicate and function symbols."
%%% TODO: Example

\section{Syntax} \label{sec_syntax}

Briefly, the types, terms, and formulas of TFF1 are analogous to those of TFF0,
except that function and predicate symbols can be declared to be polymorphic,
types can contain type variables, and $n$-ary type constructors are allowed.
Type variables in type signatures and formulas are explicitly bound. Instances of
polymorphic constants are specified by explicit type arguments, rather than
inferred.

\ourparagraph{Types.} The {\em types\/} of TFF1 are built from {\em type
variables\/} and {\em type constructors\/} of fixed arities. Nullary type
constructors are called {\em type constants}. The usual conventions of TPTP
apply: Type variables start with an uppercase letter and type constructors
with a lowercase letter. The types \verb+A+, \verb+list(A)+, \verb+list(bird)+,
and \verb+map(nat,+ \verb+list(B))+ are all examples of well-formed types. A
type is {\em polymorphic} if it contains type variables; otherwise, it is {\em
monomorphic.}

%%% @ANDREI: $iType and $oType don't appear in the TFF0 spec. I'm not a big fan
%%% of aliases.

TFF1 inherits several predefined type constants from TFF0.
The type {\tt \$i} %(also called \verb+$iType+)
of individuals is predefined but has no peculiar semantics, whereas the
arithmetic types
{\tt \$int}, {\tt \$rat}, and {\tt \$real} are modeled by $\mathbb{Z}$,
$\mathbb{Q}$, and $\mathbb{R}$, respectively. We refer
to the TFF0 specification \cite{sutcliffe-et-al-2011-tff0} for the semantics of
the arithmetic types and their operations.

\ourparagraph{Type Signatures.}
Each function and predicate symbol occurring in a formula must be associated
with a {\em type signature\/} that specifies the types of the arguments and, for
functions, the return type. Type signatures can take the following forms:
%
\begin{enumerate}
\item[(a)] a type;
\item[(b)] the Boolean pseudotype {\tt \$o}; % (also called {\tt \$oType});
\item[(c)] {\tt ($\tau_1$\;*\;${\cdots}$\;*\;$\tau_n$)\;>\;$\tau$} for $n > 0$,
where $\tau_1,\dots,\tau_n$ and $\tau$ are types;
\item[(d)] {\tt ($\tau_1$\;*\;${\cdots}$\;*\;$\tau_n$)\;>\;\$o} for $n > 0$,
where $\tau_1,\dots,\tau_n$ are types;
\item[(e)] {\tt !>\;[$\alpha_1$\,:\,\$tType,}\;{\tt ${\dots}$,}\;{\tt
$\alpha_n$\,:\,\$tType]:\;$\varsigma$}, where $\alpha_1,\dots,\alpha_n$ are distinct
type variables, $\varsigma$ has one of the previous four forms, and every type
variable occurring in $\varsigma$ appears among the $\alpha_{\!j}$'s. If
$\varsigma$ is of form (c) or (d), it must be put in parentheses.%
\footnote{This restriction is necessary for the TPTP World's Prolog-based parser.}
%The converse is not true: Some $\alpha_i$ cannot occur in $\tau$.
\end{enumerate}
%
The parentheses in forms (c) and (d) are omitted if $n = 1$. The binder {\tt !>}
in form (e) denotes %type-level (it's really type-signature-level)
universal quantification.

%%% TODO: define monomorphic and polymorphic symbols?
Form (a) is used for variables and monomorphic constants; form (b), for
propositional constants, including the predefined symbols {\tt \$true} and
{\tt \$false}; form (c), for monomorphic functions; form (d), for monomorphic
predicates; and form (e), for polymorphic functions and predicates. It is often
convenient to regard all forms above as instances of the general syntax
\begin{center}
{\tt !>} {\tt [$\alpha_1$\;:\;\$tType,\;${\dots}$,\;$\alpha_m$\;:\;\$tType]:} {\tt
(($\tau_1$\;*\;${\cdots}$\;*\;$\tau_n$)\;>\;$\tilde \tau$)}
\end{center}
where $m$ and $n$ can be 0 and $\tilde \tau$ can be any type or {\tt \$o}.

Type variables that are bound by {\tt !>} without
occurring in the type signature's body are called \emph{ghost type variables}.
These make it possible to specify operations and relations directly on types,
providing a convenient way to encode type classes.
For example, we can declare a polymorphic propositional
constant {\tt is\_linear} with the signature
{\tt !>}~{\tt [A} {\tt :} {\tt \$tType]:} {\tt \$o} and use it as a guard to restrict the
axioms specifying that a binary predicate {\tt less\_eq} with the signature
{\tt !>}~{\tt [A} {\tt :} {\tt \$tType]:} {\tt ((A} {\tt *} {\tt A)} {\tt >} {\tt \$o)}
is a linear order to those types that satisfy the {\tt is\_linear} predicate.

%A type with a quantifier prefix is called {\em polytype},
%otherwise we call it {\em monotype}.
%A monomorphic type is called {\em sort}.
%Notice that a monotype can be a polymorphic type,
%i.e.~contain type variables.

\ourparagraph{Type Declarations.} Type constructors,
like function and predicate symbols,
can optionally be declared prior to use.
The following declarations introduce a type
constant {\tt bird}, a unary type constructor {\tt list},
and a binary type constructor {\tt map}:
\begin{quote}
\verb+tff(bird_t, type, bird: $tType).+
\par\smallskip
\verb+tff(list_t, type, list: $tType > $tType).+
\par\smallskip
\verb+tff(map_t, type, map: ($tType * $tType) > $tType).+
\end{quote}
Type constructors must be fully applied, and therefore
their arity can be %unambiguously
determined at the first occurrence.
%%% @ANDREI: Too early for this. TODO: move where it belongs (and edit)
%However, one should keep in mind that a type
%expression can occur in a position of a term (see below
%the examples of {\tt cons} and other polymorphic symbols).
%A TFF parser should refer to the type signatures of function
%and predicate symbols in order to know whether it parses
%a term or a type.

A declaration of a function or predicate symbol specifies its {\em type
signature}. The following declarations introduce a monomorphic
constant {\tt pi}, a polymorphic predicate
{\tt is\_empty}, and a pair of polymorphic functions {\tt cons} and {\tt lookup}:
\begin{quote}
\verb+tff(pi_t, type, pi: $real).+
\par\smallskip
\verb+tff(is_empty_t, type, is_empty : !> [A : $tType]: (list(A) > $o)).+\kern-10mm
\par\smallskip
\verb+tff(cons_t, type,+\\
\verb+    cons : !> [A : $tType]: ((A * list(A)) > list(A))).+
\par\smallskip
\verb+tff(lookup_t, type,+\\
\verb+    lookup : !> [A : $tType, B : $tType]: ((map(A, B) * A) > B)).+
\end{quote}
If a function or predicate symbol is used before being declared, a
default type signature is assumed:\ {\tt (\$i\;*\;${\cdots}$\;*\;\$i)\;>\;\$i}
for functions and {\tt (\$i\;*\;${\cdots}$\;*\;\$i)\;>\;\$o} for predicates.

Each type constructor, function symbol, and predicate symbol may be declared at
most once. If a symbol is declared after its first use, the declared signature
must agree with the default signature. All symbols share the same namespace; in
particular, a type constructor cannot have the same name as a function or
predicate symbol.

%\footnote{Is this restriction present in TPTP FOF and TFF0? If not, we can
%lift it for TFF1.}.
%%% @ANDREI: FOF definitely shares the namespace for functions and symbols.
%%% TFF0 doesn't seem to mention anything but Sutcliffe's ToFoF cannot cope with
%%%   tff(foo_type, type, foo : $tType).
%%%   tff(foo_const, type, foo : foo).
%%%   tff(foo_ax, axiom, (![X : foo] : (X = foo))).
%%% Since TFFx is supposed to be about interoperability, in matters like these
%%% I tend not to put too heavy a burden on implementors.

\ourparagraph{Function and Predicate Application.} To keep the required type
inference to a minimum, every use of a polymorphic symbol must explicitly
specify the type instance. A function or predicate symbol with a type signature
\[\hbox{{\tt !>} {\tt [$\alpha_1$\;:\;\$tType,\;${\dots}$,\;$\alpha_m$\;:\;\$tType]:\;%
(($\tau_1$\;*\;${\cdots}$\;*\;$\tau_n$)\;>\;$\tilde \tau$)}}\]
must be applied to $m$ type arguments and $n$ term arguments. Given the above
signatures for {\tt is\_empty}, {\tt cons}, and {\tt lookup}, the term
\[\hbox{\tt lookup(\$int,\;\,list(A),\;\,M,\;\,2)}\]
and the atomic formula
\[\hbox{\tt is\_empty(\$i,\;\,cons(\$i,\;X,\;nil(\$i)))}\]
are well-formed.
%, assuming that {\tt A} ranges over {\tt
%\$tType} (i.e., it is a type variable), {\tt M} has type {\tt
%map(\$int,} {\tt list(A))}, and {\tt X} has type {\tt \$i}.

In keeping with TFF1's rank-1 polymorphic nature, type variables can only be
instantiated with actual types. In particular, \verb+$o+, \verb+$tType+,
and {\tt !>}-binders cannot occur in type arguments of polymorphic symbols.

For systems that implement type inference, the following extension
of TFF1 might be useful. When a type argument of
a polymorphic symbol can be inferred automatically, it may be
replaced with an underscore ({\tt \_}). For example:
\[\hbox{{\tt is\_empty(\_,} {\tt cons(\_,\;X,\;nil(\_)))}}\]
% \footnote{Should this be part of {\tt SyntaxBNF}?}
% @ANDREI: I don't care much either way... SyntaxBNF is anyway very slack; I
% wouldn't be surprised if it could be used to parse C++. ;)
Although {\tt nil}'s type argument cannot be inferred locally from the types of
the term arguments, it can be deduced using the Hindley--Milner type inference.
The producer of a TFF1 problem must be aware of the type inference algorithm
implemented in the consumer to omit only redundant type arguments.

\ourparagraph{Type and Term Variables.}
Every variable in a TFF1 formula must be bound. It can be given a type at
binding time:
\begin{quote}
\begin{verbatim}
tff(bird_list_not_empty, axiom,
    ![B : bird, Bs : list(bird)]:
        ~ is_empty(bird, cons(bird, B, Bs))).
\end{verbatim}
\end{quote}
If the type and the preceding colon ({\tt :}) are omitted, the variable is given
type~\verb+$i+. Every type variable occurring in a TFF1 formula
(whether in a type argument or in the type of a bound variable)
must also be bound:
\begin{quote}
\begin{verbatim}
tff(lookup_update, axiom,
    ![A : $tType, B : $tType, M : map(A, B), I : A, V : B]:
        lookup(A, B, update(A, B, M, I, V), I) = V).
\end{verbatim}
\end{quote}
A single quantifier cluster can bind both type variables and term variables.
%If the type of a term variable contains a type variable, the latter must be bound
%before the former.
%\footnote{We can permit omitting the {\tt\$tType} annotations,
%too, but should we? If an untyped variable can turn to be
%a type variable or an {\tt\$i}-variable, then we must look
%into the formula to know which it is.}.
%%% @ANDREI: We shouldn't allow that. Chances are that some implementations
%%% would get that wrong.
% (one might need
% to eliminate upper equivalences to establish the polarity
% of a quantifier)
%%% @ANDREI: What is an _upper_ equivalence? An outer one? I'd rather postpone
%%% the discussion on polarity until the part that treats skolemization. It is
%%% enough here to say "including equivalence", to hint at the issue, IMO.
%
Universal and existential quantifiers over type variables are allowed under the
propositional connectives, including equivalence, as well as under other
quantifiers over type variables, but not in the scope of a quantifier over a
term variable.
Rationale: A statement of the form ``for every integer $k$, there exists a type
$\alpha$ such that $\ldots$'' effectively makes $\alpha$ a dependent type.
On such statements, type skolemization (Section~\ref{ssec:skol}) is impossible,
and there is no easy translation to ML-style polymorphic formalisms.
%%% @ANDREI: We can be forceful here, because "easy" is ambiguous anyway.
%%% Nobody has ever managed to encode dependent types in a nice way in HOL, and
%%% this has been attempted for decades. Surely any new discovery there will not
%%% be "easy".
Moreover, type handling in an automatic prover would be more difficult were
such constructions allowed, since they require paramodulation into types.

On the other hand, all the notions and procedures described in this
specification, except for type skolemization, are independent of this
restriction. The rules of type checking and the notion of interpretation are
directly applicable to unrestricted formulas. The encoding into a monomorphic
logic (Section~\ref{sec:trans}) is sound and complete on unrestricted formulas,
and the proofs require
%% Theorems \ref{thm:mon_sound}~and~\ref{thm:mon_compl} require
%% (I'm no fan of forward references)
no adjustments. This prepares the ground for TFF2, which is expected to lift the
restriction and support more elaborate forms of dependent types. Implementations
of TFF1 are encouraged to support unrestricted formulas, treating them according
to the semantics given here, if practicable.

%Another way to introduce a variable is via the ``let'' construct.
%A recent extension of the TFF0 format, discussed at the 7th TPTP Tea Party in
%Wrocław and not yet implemented, allows to introduce a function or predicate
%locally. For example, the formulas
%\begin{center}
%\begin{tabular}{l}
%{\tt :=\;[X\;:\;\$int,\;f(X)\;=\;g(X,\;X)]:\;p(f(a))} \\[\smallskipamount]
%{\tt :=\;[X\;:\;\$int,\;p(X)\;<=>\;q(X,\;X)]:\;p(a)}
%\end{tabular}
%\end{center}
%expand to {\tt p(g(a,\;a))} and {\tt q(a,\;a)}, respectively
%\cite{geoff-tptptp-notes}.
%Similarly, TFF1 admits binding a type~$\tau$ to $\kappa(\alpha_1, \ldots,
%\alpha_n)$, where $\alpha_1, \ldots, \alpha_n$ are distinct type variables bound
%in $\tau$, using the syntax $\kappa(\alpha_1, \ldots, \alpha_n)$~{\tt :>}~$\tau$ inside
%the ``let'' binder.

%
%bind a variable
%to a term as well as to a formula (a variable thus bound has
%``type'' \verb+$o+ and can only appear in a position of a formula)
%%% @ANDREI: The proposed syntax is actually more general.
%
%\footnote{Should we elaborate on the syntax of let?
%It was already in TFF0, but it was not described in
%the article on TFF0. The same about if/ite: should
%we talk about them, though they are completely
%orthogonal to polymorphism and types?}.
%In the former case,
%the syntax of binding is {\it variable}~{\tt :-}~{\it term\/} and
%the type of the variable is the type of the term. In the latter case,
%the syntax of binding is {\it variable}~{\tt :=}~{\it formula\/} and
%a variable thus bound has the ``type'' \verb+$o+ and
%can only appear in a position of a formula (notice that
%one cannot quantify over a variable of type \verb+$o+).
%
%\footnote{Since any type is a well-formed TFF1 term,
%we could reuse the existing syntax for these bindings.
%But then suppose that {\tt foo} is undeclared and
%we bind {\tt A} to it: {\tt A} {\tt :-} {\tt foo}.
%Until {\tt A} is used
%(as a type or as a value), we cannot know what it is.}.
%For example, the axiom \verb+bird_list_not_empty+ above could also be as
%follows:
%\begin{quote}
%\begin{verbatim}
%tff(bird_list_not_empty, axiom,
%    := [b :> bird, A : $tType, c(A) :> list(A)]:
%        ![B : b, Bs : c(b)]: ~ is_empty(b, cons(b, B, Bs))).
%\end{verbatim}
%\end{quote}
%One can put a mixed list of bindings in a let-statement;
%in such lists each subsequent binding can refer
%to the variables introduced by preceding ones%
%\footnote{Is this true in TFF0? I didn't find any written
%specification of multi-variable let.}.
%
%The ``let'' bindings can be eliminated from TFF1 formulas using capture-avoiding
%substitution. Hence we do not need to consider them in the rest of this
%specification.

\ourparagraph{Terms and Formulas.} Apart from the differences described above,
the terms and formulas of TFF1 are identical to those of TFF0. We refer to the
TFF0 specification \cite{sutcliffe-et-al-2011-tff0} for further information.

