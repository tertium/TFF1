\section{Translations} \label{sec:trans}

\subsection{Encoding into a Many-Sorted Logic} \label{ssec:tff0}

We describe here a simple translation from TFF1 to a traditional
many-sorted first-order logic. More sophisticated encodings which
preserve selected types by translating them into separate sorts,
or do not change the propositional structure of encoded formulas
are described in \cite{leino10tacas,bobot11frocos}.
Notice that both papers assume preliminary elimination
of ghost type arguments (cf.~Section~\ref{ssec:ghost});
the latter also assumes preliminary type skolemization
(cf.~Section~\ref{ssec:skol}).

Let $\Delta$ be a set of TFF1 sentences, that is, closed and well-typed
formulas. We construct an equisatisfiable set of monomorphic two-sorted
formulas $\MON(\Delta)$ as follows.
%
We introduce two sorts, $\typ$ and $\und$.
To every type variable $\alpha$ in $\Delta$ we assign
a fresh variable $\hat{\alpha}$ of sort $\typ$.
To every ordinary variable $u$ we assign
a fresh variable $\hat{u}$ of sort $\und$.
To every type constructor $K$ of arity $n$ we assign
a function symbol $\hat{K}$ of sort signature $\typ^n \to \typ$.
To every function symbol $f$ of type signature
$\forall \alpha_1\dots\alpha_n \,.\, T_1 \times \dots \times T_m \to T$
we assign a function symbol $\hat{f}$ of sort signature
$\typ^n \times \und^m \to \und$.
To every predicate symbol $p$ of type signature
$\forall \alpha_1\dots\alpha_n \,.\, T_1 \times \dots \times T_m \to \omicron$
we assign a predicate symbol $\hat{p}$ of sort signature
$\typ^n \times \und^m \to \omicron$.
Finally, we add a new predicate symbol $\typeof$ of sort signature
$\und \times \typ$.

The $\MON$ transformation translates TFF1 types, terms, and formulas
according to the following equations ($\bar{S}$ and $\bar{t}$ denote
sequences of types and terms, respectively):
\begin{align*}
\MON(\alpha) &\eqdef \hat{\alpha} &
\MON(K(\bar{S})) &\eqdef \hat{K}(\MON(\bar{S})) \\
\MON(u) &\eqdef \hat{u} &
\MON(f(\bar{S},\bar{t})) &\eqdef \hat{f}(\MON(\bar{S}),\MON(\bar{t})) \\
\MON(t_1 \eq t_2) &\eqdef \MON(t_1) \eq \MON(t_2) &
\MON(p(\bar{S},\bar{t})) &\eqdef \hat{p}(\MON(\bar{S}),\MON(\bar{t})) \\
\MON(\lnot F) &\eqdef \lnot\, \MON(F) &
\MON(F \land G) &\eqdef \MON(F) \land\, \MON(G) \\
\MON(\forall u \ty T .\, F) &\eqdef
\forall \hat{u} \,.\, \typeof(\hat{u}, \MON(T)) \limp \MON(F) &
\MON(\forall \alpha \,.\, F) &\eqdef
\forall \hat{\alpha} \,.\, \MON(F)
\end{align*}

\newcommand{\AxD}{\text{\sc Ax}_{\Delta}}
\newcommand{\Inh}{\text{\sc Inh}}
\newcommand{\Dom}{\mathrm{Dom}}

Consider a function symbol $f$ with type signature
$\forall \alpha_1\dots\alpha_n \,.\, T_1 \times \dots \times T_m \to T$.
The {\em typing axiom\/} for $f$ is a formula
$$
\forall \hat{\alpha}_1\dots\hat{\alpha}_n \,.\,
\forall \hat{v}_1\dots\hat{v}_m \,.\,
\typeof(\hat{f}(\hat{\alpha}_1,\dots,\hat{\alpha}_n,
\hat{v}_1,\dots,\hat{v}_m), \MON(T)).
$$
We denote with $\AxD$ the set of typing axioms
for all function symbols occurring in $\Delta$.
The {\em inhabitance axiom}, denoted $\Inh$,
is a formula
$\forall \hat{\alpha} \exists \hat{u} \,.\,
\typeof(\hat{u},\hat{\alpha}).$
Then we define
$$
\MON(\Delta) \eqdef \{ \MON(F) \vb F \in \Delta \} \cup
\AxD \cup \{ \Inh \}.
$$

It is easy to see that $\MON$ converts TFF1 types into well-formed
terms of sort $\typ$, TFF1 terms into well-formed terms of sort $\und$,
and TFF1 formulas (not necessarily well-typed) into well-formed formulas.

\begin{theorem}[Soundness of $\MON$] \label{thm:mon_sound}
If a set of sentences $\Delta$ is satisfiable in TFF1,
then $\MON(\Delta)$ is satisfiable in classical
many-sorted first-order logic with equality.
\end{theorem}
\begin{proof}
Let $\cM$ be a model of $\Delta$.
We construct an interpretation $\cI$ of $\MON(\Delta)$ as follows.
Let us denote with $\sorts$ and $\univ$, respectively,
the set of domains and the universe in $\cM$.
In $\cI$, we define the domain of sort $\typ$ to be $\sorts$
and the domain of sort $\und$ to be $\univ$.
Every function symbol $\hat{K}$ (function symbol $\hat{f}$,
predicate symbol $\hat{p}$) is interpreted in $\cI$
exactly as the type constructor $K$ (function symbol $f$,
predicate symbol $p$) in $\cM$.
The predicate symbol $\typeof$ is interpreted as the membership relation.
%This concludes the definition of interpretation $\cI$.

We are going to show that for every type $S$, term $t$,
and formula $F$ occurring in $\Delta$,
for every variable valuation $\nu$ in $\cI$, we have
\begin{align*}
\llb \MON(S) \rrb^\cI_{\nu} &= \llb S \rrb^\cM_{\theta} &
\llb \MON(t) \rrb^\cI_{\nu} &= \llb t \rrb^\cM_{\theta,\xi} &
\llb \MON(F) \rrb^\cI_{\nu} &= \llb F \rrb^\cM_{\theta,\xi}
\end{align*}
where $\theta(\alpha) \eqdef \nu(\hat{\alpha})$ and
$\xi(u) \eqdef \nu(\hat{u})$ for every
type variable $\alpha$ and every variable $u$.
%
We prove these equalities by induction on the structure
of types, terms, and formulas. The only non-trivial case is
that of a quantified formula, $\forall u \ty T .\, F$.
It is easy to see that any $a \in \univ$ belongs
to $\llb T \rrb^\cM_\theta$ if and only if
$\llb \typeof(\hat{u},\MON(T)) \rrb^\cI_{\nu[\hat{u}\,\mapsto\,a]}$
holds. Indeed, the latter is equivalent to
$a \in \llb \MON(T) \rrb^\cI_{\nu[\hat{u}\,\mapsto\,a]}$.
Since $\nu$ and $\nu[\hat{u} \mapsto a]$ produce the same
$\theta$ in $\cM$,
$\llb \MON(T) \rrb^\cI_{\nu[\hat{u}\,\mapsto\,a]} =
\llb T \rrb^\cM_\theta$.
Then
$\llb \forall \hat{u} \,.\, \typeof(\hat{u},\MON(T))
\limp \MON(F) \rrb^\cI_{\nu}$ is exactly
$\llb \forall u \ty T .\, F \rrb^\cM_{\theta,\xi}$
by induction hypothesis.

Now we only have to check that the typing axioms
$\AxD$ and the inhabitance axiom $\Inh$ hold in $\cI$.
This immediately follows from the definition of
interpretation in TFF1 and the first equality above.
\qed
\end{proof}

\begin{theorem}[Completeness of $\MON$] \label{thm:mon_compl}
Given a set of sentences $\Delta$, if $\MON(\Delta)$
is satisfiable in classical many-sorted first-order logic
with equality, then $\Delta$ is satisfiable in TFF1.
\end{theorem}
\begin{proof}
Let $\cM$ be a model of $\MON(\Delta)$. Let us replace every element $d$ in
the domain of $\typ$ by the set $D = \{\la a,d \ra \,|\, \typeof^\cM(a,d)\}$
and update correspondingly the interpretations of type constructors
and function and predicate symbols in $\cM$.
We can safely perform this substitution, because the inhabitance
axiom $\Inh$ holds in $\cM$, and therefore every such set is non-empty
and there is one-to-one correspondence with the initial domain of $\typ$.
In the model $\cM$ thus modified, $\typeof^\cM(a,D)$ holds if and only if
$D$ contains a pair $\la a,d \ra$.
%
Below, $\pi_1$ and $\pi_2$ stand for the first and the second projection
of a pair, respectively.

We construct an interpretation $\cI$ of $\Delta$ as follows.
We define the set of domains $\sorts$ to be the new domain
of $\typ$ in $\cM$. As usual, $\univ$ denotes the union
of all domains in $\sorts$. Notice that some elements of
the domain of $\und$ may not appear in any pair in $\univ$.

The symbols of $\Delta$ are interpreted in $\cI$
according to the following equations:
\begin{align*}
K^\cI(D_1,\dots,D_n) &\eqdef \hat{K}^\cM(D_1,\dots,D_n) \\
f^\cI(D_1,\dots,D_n,\la a_1,d_1 \ra,\dots,\la a_m,d_m \ra) &\eqdef
\la \hat{f}^\cM(D_1,\dots,D_n,a_1,\dots,a_m), d \ra \\
p^\cI(D_1,\dots,D_n,\la a_1,d_1 \ra,\dots,\la a_m,d_m \ra) &\eqdef
\hat{p}^\cM(D_1,\dots,D_n,a_1,\dots,a_m)
\end{align*}
where
$K$ is of arity $n$,
the signature of $p$ is
$\forall \alpha_1\dots\alpha_n \,.\, T_1 \times \dots \times T_m \to
\omicron$,
the signature of $f$ is
$\forall \alpha_1\dots\alpha_n \,.\, T_1 \times \dots \times T_m \to T$,
and $d$ is the fixed second coordinate of pairs in the domain
$D = \llb \MON(T) \rrb^\cM_{[\hat{\alpha}_1\,\mapsto\,D_1,\dots,
\hat{\alpha_n}\,\mapsto\,D_n]} =
\llb T \rrb^\cI_{[\alpha_1\,\mapsto\,D_1,\dots,\alpha_n\,\mapsto\,D_n]}$.
Since the typing axioms $\AxD$ hold in $\cM$, we have
$\typeof^\cM(f^\cM(D_1,\dots,D_n,a_1,\dots,a_m), D)$,
and therefore, the result of $f^\cI$ belongs indeed to $D$.
This concludes the definition of $\cI$.

Given a type context $\gamma$, a type valuation $\theta$,
and a variable valuation $\xi$, we say that they are admissible
for a TFF1 formula $F$, when the following conditions are satisfied:
\begin{itemize}
\item $F$ is well-typed in context $\gamma$;
\item for every variable $v$ free in $F$, no type variable
occurring in $\gamma(v)$ is bound in $F$;
\item for every variable $v$ free in $F$, $\xi(v) \in \evalT{\gamma(v)}$.
\end{itemize}
Obviously, every triple is admissible for a sentence.
Let us show that for every type $S$, term $t$, and formula $F$,
and for all $\gamma$, $\theta$, $\xi$ admissible for $F$, we have
\begin{align*}
\evalT{S} &= \llb \MON(S) \rrb^\cM_{\nu} &
\pi_1(\eval{t}) &= \llb \MON(t) \rrb^\cM_{\nu} &
\eval{F} &= \llb \MON(F) \rrb^\cM_{\nu}
\end{align*}
where $\nu(\hat{\alpha}) \eqdef \theta(\alpha)$ and
$\nu(\hat{u}) \eqdef \pi_1(\xi(u))$
for every type variable $\alpha$ and variable $u$.

The proof goes by induction on the structure of types, terms, and formulas.
We have three non-trivial cases: equality, quantification over an ordinary
variable, and quantification over a type variable.

Let $F$ be an equality $t_1 \eq t_2$.
Let $\la a_1, d_1 \ra$ be $\eval{t_1}$ and $\la a_2, d_2 \ra$ be $\eval{t_2}$.
By induction hypothesis, $\llb \MON(t_1) \rrb^\cM_{\nu} = a_1$ and
$\llb \MON(t_2) \rrb^\cM_{\nu} = a_2$. Let us show that $d_1 = d_2$.
By assumption, $F$ is well-typed in context $\gamma$.
Thus, $\gamma\,\vdash t_1 \ty S$ and $\gamma\,\vdash t_2 \ty S$
for some type $S$.
If $t_1$ is a variable $v$, then
$\eval{t_1} = \xi(v) \in \eval{\gamma(v)} = \evalT{S}$.
Let $t_1$ be a function application $f(S_1,\dots,S_n,s_1,\dots,s_m)$,
where $f \ty
\forall \alpha_1\dots\alpha_n \,.\, T_1 \times \dots \times T_m \to T$.
Let $D_i = \evalT{S_i}$ for every $i \in [1,n]$.
Then $S = T[S_1/\alpha_1,\dots,S_n/\alpha_n]$ and
$\evalT{S} = \llb T \rrb^\cI_{[\alpha_1 \,\mapsto\, D_1,\dots,
\alpha_n \,\mapsto\, D_n]}$.
By construction of $\cI$, $\eval{t_1} \in \evalT{S}$.
By the same argument, $\eval{t_2} \in \evalT{S}$.
Hence, $d_1 = d_2$ and $\eval{F} = \llb \MON(F) \rrb^\cM_{\nu}$.

Let $F$ be a quantified formula $\forall u \ty T .\, G$.
We must show that for every pair $\la a,d \ra$ in $\evalT{T}$, we have
$\typeof^\cM(a, \llb \MON(T) \rrb^\cM_{\nu[\hat{u}\,\mapsto\,a]})$,
and vice versa, for every $a$ in the domain of $\und$, if
$\typeof^\cM(a, \llb \MON(T) \rrb^\cM_{\nu[\hat{u}\,\mapsto\,a]})$
holds, then there exists some $d$ such that $\la a,d \ra \in
\evalT{T}$.
%
Since ordinary variables do not occur in types,
$\llb \MON(T) \rrb^\cM_{\nu[\hat{u}\,\mapsto\,a]} =
\llb T \rrb^\cI_\theta$.
By construction of $\cM$,
$\typeof^\cM(a, \llb T \rrb^\cI_\theta)$
holds if and only if there is some $d'$ such that
$\la a,d' \ra \in \llb T \rrb^\cI_\theta$.
%
Now, notice that the triple $\gamma[u \mapsto T],
\theta,\xi[u \mapsto \la a,d \ra]$ is admissible for $G$.
Indeed, $G$ is well-typed in $\gamma[u \mapsto T]$,
and for every $v \in \FV(G)$, we have
$\xi[u \mapsto \la a,d \ra](v) \in \evalT{\gamma[u \mapsto T](v)}$.
Also, if $T$ contains a type variable $\beta$ bound in $G$,
then $\beta$ is both free and bound in $F$, which violates
the no-clash assumption we have made.
%
Then, by induction hypothesis, $\eval{\forall u \ty T .\, G}$
is exactly $\llb \forall \hat{u} \,.\, \typeof(\hat{u},\MON(T))
\limp \MON(G) \rrb^\cM_{\nu}$.

Let $F$ be a quantified formula $\forall \alpha \,.\, G$.
Let $D$ be an element of $\sorts$. Under the no-clash assumption,
formula $G$ is well-typed in context $\gamma$, and for all
$v \in \FV(G) = \FV(F)$, we have
$\evalG{\gamma(v)}_{\theta[\alpha\,\mapsto\,D]} = \evalT{\gamma(v)}$.
Then $\eval{F} = \llb \MON(F) \rrb^\cM_{\nu}$ by induction hypothesis.
\qed
\end{proof}

\subsection{Elimination of Ghost Type Arguments} \label{ssec:ghost}
The explicit type arguments of polymorphic symbols
in TFF1 let us introduce operations and relations directly on types.
Thus, one could declare and axiomatize a polymorphic propositional
constant $\sym{is\_ordered}$ of type signature $\forall\alpha\,.\,\omicron$.
Such predicate symbols can be used to encode type classes:
for example, the axioms of order for a polymorphic predicate
$\sym{less\_eq} \,\ty\, \forall\alpha\,.\,\alpha\times\alpha\to\omicron$
can be restricted to types that satisfy $\sym{is\_ordered}$.

However, type signatures such as that of $\sym{is\_ordered}$ are not
allowed in languages where function and predicate symbols do not take
explicit type arguments, and the concrete instance of the symbol's type
signature is determined by the types of its term arguments and the type
of the result.
One particularly important example of such language is HOL
and its incarnations, HOL Light~\cite{harrison09hollight}
and Isabelle/HOL~\cite{NipkowPaulsonWenzel2002Isabelle}.
The same restriction is adopted in the theorem proving systems
Boogie~\cite{Barnett06boogie}, Why3~\cite{boogie11why3}, and
Alt-Ergo~\cite{conchon08smt}%
\footnote{Other relevant citations, please? What about Mizar?}.
To be able to translate TFF1 problems for these systems, we need
to get rid of {\em ghost type arguments} in polymorphic symbols,
that is, the type variables in the symbol's type signature that
do not appear in the quantifier-free matrix.
This can be done in an easy and lightweight manner.

Let $\Delta$ be a set of sentences.
We construct an equisatisfiable set $\ELI(\Delta)$ as follows.
%
We introduce a new unary type constructor $\ghost$.
We replace every function symbol $f$ of type signature
$\forall \alpha_1\dots\alpha_n \,.\, T_1 \times \dots \times T_m \to T$
with a new function symbol $\hat{f}$ of type signature
$\forall \alpha_1\dots\alpha_n \,.\,
\ghost(\alpha_{i_1}) \times \dots \times \ghost(\alpha_{i_r}) \times
T_1 \times \dots \times T_m \to T$,
where $\alpha_{i_1},\dots,\alpha_{i_r}$ are the type
variables from the quantifier prefix that do not occur in
$T_1,\dots,T_m,T$.
Likewise, every predicate symbol $p \,\ty\,
\forall \alpha_1\dots\alpha_n \,.\, T_1 \times \dots \times T_m \to \omicron$
is replaced with a new predicate symbol $\hat{p} \,\ty\,
\forall \alpha_1\dots\alpha_n \,.\,
\ghost(\alpha_{i_1}) \times \dots \times \ghost(\alpha_{i_r}) \times
T_1 \times \dots \times T_m \to \omicron$,
where $\alpha_{i_1},\dots,\alpha_{i_r}$ are the type
variables from the quantifier prefix that do not occur in
$T_1,\dots,T_m$.
Finally, we add a ``witness'' constant $\wit$
of type signature $\forall \alpha\,.\,\ghost(\alpha)$.
The $\ELI$ transformation translates terms and atomic
formulas according to the following equations:
\begin{align*}
\ELI(u) \eqdef u
\quad\quad & \qquad\qquad\qquad\quad
\ELI(t_1 \eq t_2) \eqdef \ELI(t_1) \eq \ELI(t_2) \\
\ELI(f(S_1,\dots,S_n,t_1,\dots,t_m)) &\eqdef
\hat{f}(S_1,\dots,S_n,\wit(S_{i_1}),\dots,\wit(S_{i_r}),
\ELI(t_1),\dots,\ELI(t_m)) \\
\ELI(p(S_1,\dots,S_n,t_1,\dots,t_m)) &\eqdef
\hat{p}(S_1,\dots,S_n,\wit(S_{i_1}),\dots,\wit(S_{i_r}),
\ELI(t_1),\dots,\ELI(t_m))
\end{align*}
Given a formula $F$ or a set of sentences $\Delta$,
$\ELI(F)$ and $\ELI(\Delta)$ denote the result of
applying $\ELI$ to every atomic formula in $F$ and
$\Delta$, respectively.

\begin{theorem} \label{thm:eli}
Any $\Delta$ is equisatisfiable to $\ELI(\Delta)$.
\end{theorem}
\begin{proof}
A model of $\Delta$ can be converted to a model of $\ELI(\Delta)$
as follows. We choose an arbitrary value $e$ and set the singleton
$\{ e \}$ as the domain of $\ghost(D)$ for any domain $D$.
Accordingly, $\wit$ evaluates to $e$ on any argument.
Interpretations of other function and predicate symbols are
adjusted in an obvious way.

To convert a model of $\ELI(\Delta)$
to a model of $\Delta$, we evaluate any functional symbol
$f$ on $D_1,\dots,D_n,a_1,\dots,a_m$ exactly as
$\hat{f}$ on $D_1,\dots,D_n,e_1,\dots,e_r,a_1,\dots,a_m$,
where each $e_j$ is the evaluation of $\wit$ on $D_{i_j}$;
likewise for predicate symbols.
\qed
\end{proof}

\subsection{Type Skolemization} \label{ssec:skol}

Another feature of TFF1 which is not universally supported
by other systems with polymorphic types is explicit
quantification over type variables.
For example, Coq and Boogie accept quantifiers over types%
\footnote{Mizar? Anybody else?}.
On the other side, the systems based on HOL as well as Why3 and
Alt-Ergo treat all type variables as implicitly universally
quantified at the top of a formula, which is similar to
the treatment of variables in clauses.
A simple and practical solution to translate TFF1 problems
for the systems of the second kind is type skolemization.
Unfortunately, not every TFF1 problem can be treated this way.

We introduce two transformations $\SKO^-$ and $\SKO^+$ to
perform type skolemization in premises and in goals, respectively
(the latter should be rather called ``type herbrandization'').
Let us call a formula $F$ {\em dependently typed} whenever
it contains a quantifier over a type variable in the scope
of a quantifier over an ordinary variable%
\footnote{We might take polarities into account, but then
the proof of Theorem~\ref{thm:sko} wouldn't be so sweet.
Types-under-values are nasty anyway.}.
The transformations $\SKO^-$ and $\SKO^+$ apply to
non-dependently typed TFF1 formulas according to the
following equations:
\begin{align*}
\SKO^-(p(\bar{S},\bar{t})) &\eqdef p(\bar{S},\bar{t}) &
\SKO^+(p(\bar{S},\bar{t})) &\eqdef p(\bar{S},\bar{t}) \\
\SKO^-(t_1 \eq t_2) &\eqdef t_1 \eq t_2 &
\SKO^+(t_1 \eq t_2) &\eqdef t_1 \eq t_2 \\
\SKO^-(\forall u \ty T .\, F) &\eqdef \forall u \ty T .\, F &
\SKO^+(\forall u \ty T .\, F) &\eqdef \forall u \ty T .\, F \\
\SKO^-(F \land G) &\eqdef \SKO^-(F) \land \SKO^-(G) &
\SKO^+(F \land G) &\eqdef \SKO^+(F) \land \SKO^+(G) \\
\SKO^-(\lnot F) &\eqdef \lnot \SKO^+(F) &
\SKO^+(\lnot F) &\eqdef \lnot \SKO^-(F) \\
\SKO^-(\forall \alpha \,.\, F) &\eqdef \forall \alpha \,.\, \SKO^-(F) &
\SKO^+(\forall \alpha \,.\, F) &\eqdef
\SKO^+(F[K(\beta_1,\dots,\beta_n)/\alpha])
\end{align*}
where $K$ is a fresh type constructor and $\beta_1,\dots,\beta_n$ are
the free type variables of $\forall \alpha \,.\, F$.
Given a set $\Delta$ of non-dependently typed sentences,
$\SKO^-(\Delta)$ and $\SKO^+(\Delta)$ denote the result of applying
the corresponding transformations to every formula in $\Delta$.

\begin{theorem} \label{thm:sko}
Any non-dependently typed $\Delta$ is equisatisfiable to $\SKO^-(\Delta)$.
\end{theorem}
\begin{proof}
It is easy to see that
skolemizing the $\typ$-sorted variables in $\MON(\Delta)$ gives
exactly $\MON(\SKO^-(\Delta))$ modulo renaming of Skolem symbols and
permutation of their arguments. Theorems~\ref{thm:mon_sound} and
\ref{thm:mon_compl} conclude the proof.
\qed
\end{proof}

For dependently typed problems, type skolemization would require
Skolem type constructors to take ordinary variables as arguments,
which is beyond the current scope of TFF1. In order to submit such
problems to systems that do not accept quantifiers over types,
one can resort to $\MON$ or some other translation into monomorphic
logic (cf.~Section~\ref{ssec:tff0}). At present, we are not aware
on any less intrusive way to handle dependently typed TFF1 formulas.

