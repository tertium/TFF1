\section{Translations} \label{sec:trans}

\subsection{Encoding into a Many-Sorted Logic} \label{ssec:tff0}

We describe here a simple translation from TFF1 to a traditional
many-sorted first-order logic. More sophisticated encodings which
preserve selected types (by translating them into separate sorts)
or do not change the propositional structure of encoded formulas
are described in \cite{leino10tacas,bobot11frocos}.
Notice that both papers assume preliminary elimination
of ghost type arguments (cf.~Section~\ref{ssec:ghost});
the latter also assumes preliminary type skolemization
(cf.~Section~\ref{ssec:skol}).

Let $\Delta$ be a set of TFF sentences, that is, closed and well-typed
formulas. We construct an equisatisfiable set of monomorphic two-sorted
formulas $\MON(\Delta)$ as follows.
%
We introduce two sorts, $\typ$ and $\und$.
To every type variable $\alpha$ in $\Delta$ we assign
a fresh variable $\hat{\alpha}$ of sort $\typ$.
To every ordinary variable $u$ we assign
a fresh variable $\hat{u}$ of sort $\und$.
To every type constructor $K$ of arity $n$ we assign
a function symbol $\hat{K}$ of sort signature $\typ^n \to \typ$.
To every function symbol $f$ of type signature
$\forall \alpha_1\dots\alpha_n \,.\, T_1 \times \dots \times T_m \to T$
we assign a function symbol $\hat{f}$ of sort signature
$\typ^n \times \und^m \to \und$.
To every predicate symbol $p$ of type signature
$\forall \alpha_1\dots\alpha_n \,.\, T_1 \times \dots \times T_m \to \omicron$
we assign a predicate symbol $\hat{p}$ of sort signature
$\typ^n \times \und^m \to \omicron$.
Finally, we add a new function symbol $\typeof$ of sort signature
$\und \to \typ$.

The $\MON$ transformation translates TFF types, terms, and formulas
according to the following equations ($\bar{S}$ and $\bar{t}$ denote
sequences of types and terms, respectively):
\begin{align*}
\MON(\alpha) &\eqdef \hat{\alpha} &
\MON(K(\bar{S})) &\eqdef \hat{K}(\MON(\bar{S})) \\
\MON(u) &\eqdef \hat{u} &
\MON(f(\bar{S},\bar{t})) &\eqdef \hat{f}(\MON(\bar{S}),\MON(\bar{t})) \\
\MON(t_1 \approx t_2) &\eqdef \MON(t_1) \approx \MON(t_2) &
\MON(p(\bar{S},\bar{t})) &\eqdef \hat{p}(\MON(\bar{S}),\MON(\bar{t})) \\
\MON(\lnot F) &\eqdef \lnot\, \MON(F) &
\MON(F \land G) &\eqdef \MON(F) \land\, \MON(G) \\
\MON(\forall u \ty T .\, F) &\eqdef
\forall \hat{u} \,.\,
\typeof(\hat{u}) \approx \MON(T) \limp \MON(F) &
\MON(\forall \alpha \,.\, F) &\eqdef
\forall \hat{\alpha} \,.\, \MON(F)
\end{align*}

Consider a function symbol $f$ of signature
$\forall \alpha_1\dots\alpha_n \,.\, T_1 \times \dots \times T_m \to T$.
The {\em typing axiom} for $f$ is a formula
$$
\forall \hat{\alpha}_1\dots\hat{\alpha}_n \,.\,
\forall \hat{v}_1\dots\hat{v}_m \,.\,
\typeof(\hat{f}(\hat{\alpha}_1,\dots,\hat{\alpha}_n,
\hat{v}_1,\dots,\hat{v}_m)) \approx \MON(T).
$$
We denote with $\mathrm{Ax}_{\Delta}$ the set of typing axioms
for all function symbols occurring in $\Delta$. Then we define
$$
\MON(\Delta) \eqdef \{ \MON(F) \vb F \in \Delta \} \cup
\mathrm{Ax}_{\Delta}.
$$

It is easy to see that $\MON$ converts TFF types into well-formed
terms of sort $\typ$, TFF terms into well-formed terms of sort $\und$,
and TFF formulas (not necessarily well-typed) into well-formed formulas.

\begin{theorem} \label{thm:mon_sound}
If a set of sentences $\Delta$ is satisfiable in TFF,
then $\MON(\Delta)$ is satisfiable in classical
many-sorted first-order logic with equality.
\end{theorem}

\begin{theorem} \label{thm:mon_compl}
Given a set of TFF sentences $\Delta$, if $\MON(\Delta)$
is satisfiable in classical many-sorted first-order logic
with equality, then $\Delta$ is satisfiable in TFF.
\end{theorem}

\subsection{Elimination of Ghost Type Arguments} \label{ssec:ghost}
\subsection{Type Skolemization} \label{ssec:skol}


