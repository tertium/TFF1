\section{Reduction to TFF0} \label{sec:trans}
% Or: Encoding into a Many-Sorted Logic
% Or: Translation to a Many-Sorted Logic
% Or simply: Translation

We describe a translation from TFF1 to the classical many-sorted first-order logic TFF0
with equality.%
\footnote{The TFF0 specification \cite{TFF0} prefers the appellation ``type'' to
``sort,'' but since only nullary type constructors are supported, TFF0 types are
strictly speaking sorts.}
The translation is included here for illustrative purposes; more suitable
encoding schemes for applications are discussed in
Section~\ref{sec_related_work}.

Our strategy for translating polymorphic types is to encode them as terms and
use a special binary predicate to encode type information.
%
To avoid mixing types
and terms in the encoded problem, we introduce two sorts, $\typ$ and $\und$,
corresponding to the set of domains $\sorts$ and the universe $\univ$,
respectively.

Let $\DEL$ be a set of TFF1 sentences.
%, that is, closed and well-typed formulas.
We construct an equisatisfiable set of monomorphic two-sorted
formulas $\MON(\DEL)$ as follows.
%
To every type variable $\alpha$ in $\DEL$ or in the type signature of a function
symbol, we assign a fresh variable $\hat{\alpha}$ of sort $\typ$.
To every term variable $u$, we assign
a fresh variable $\hat{u}$ of sort $\und$.
To every $n$-ary type constructor $\kappa$, we assign
a function symbol $\hat{\kappa} : \typ^n \to \typ$.
To every function symbol $f : \forall \alpha_1\dots\alpha_m .\; \tau_1 \times \dots \times \tau_n \to \tau$,
we assign a function symbol $\hat{f} : \typ^m \times \und^n \to \und$.
To every predicate symbol $p : \forall \alpha_1\dots\alpha_m .\; \tau_1 \times \dots \times \tau_n \to \omicron$,
we assign a predicate symbol $\hat{p} : \typ^m \times \und^n \to \omicron$.
Finally, we introduce the special predicate symbol $\typeof : \und \times \typ \to \omicron$.

The $\MON$ transformation translates TFF1 types, terms, and formulas
according to the following equations:
\begin{align*}
\MON(\alpha) &\eqdef \hat{\alpha} &
  \MON(t_1 \eq t_2) &\eqdef \MON(t_1) \eq \MON(t_2) \\
\MON(\kappa(\bar{\tau})) &\eqdef \hat{\kappa}(\MON(\bar{\tau})) &
  \MON(\lnot\, \varphi) &\eqdef \lnot\, \MON(\varphi) \\
\MON(u) &\eqdef \hat{u} &
  \MON(\varphi \rand \psi) &\eqdef \MON(\varphi) \rand \MON(\psi) \\
\MON(f(\bar{\tau},\,\bar{t}\,)) &\eqdef \hat{f}(\MON(\bar{\tau}),\,\MON(\bar{t}\,)) &
  \MON(\forall u \ty \tau .\; \varphi) &\eqdef
  \forall \hat{u} .\; \typeof(\hat{u},\, \MON(\tau)) \limp \MON(\varphi) \\
\MON(p(\bar{\tau},\,\bar{t}\,)) &\eqdef \hat{p}(\MON(\bar{\tau}),\,\MON(\bar{t}\,)) &
  \MON(\forall \alpha .\; \varphi) &\eqdef
  \forall \hat{\alpha} .\; \MON(\varphi)
\end{align*}
%
\newcommand{\AxD}{\text{\sc Ax}_{\DEL}}%
\newcommand{\Inh}{\text{\sc Inh}}%
\newcommand{\Dom}{\mathrm{Dom}}%
%
The $\MON$ transformation is lifted to lists of types or terms in the evident,
pointwise way.

The {\em inhabitation axiom}, $\Inh$, is the formula
$\forall \hat{\alpha}.\> \exists \hat{u} .\;
\typeof(\hat{u},\,\hat{\alpha}).$
For each function symbol $f :
\forall \alpha_1\dots\alpha_m .\; \tau_1 \times \dots \times \tau_n \to \tau$,
the associated {\em typing axiom\/} is the formula
$$
\forall \hat{\alpha}_1\dots\hat{\alpha}_m .\;
\forall \hat{u}_1\dots\hat{u}_n .\;
\typeof(\hat{f}(\hat{\alpha}_1,\dots,\hat{\alpha}_m,
\hat{u}_1,\dots,\hat{u}_n),\, \MON(\tau))%.
$$
We let $\AxD$ denote the set of typing axioms associated with all function
symbols occurring in $\DEL$.
%
Finally, we define
$$
\MON(\DEL) \eqdef \{ \MON(\varphi) \vb \varphi \in \DEL \} \mathrel\cup
\AxD \mathrel\cup \{ \Inh \}%.
$$
%
It is easy to see that $\MON$ converts TFF1 types into well-formed TFF0 terms of
sort $\typ$, TFF1 terms into well-formed TFF0 terms of sort $\und$, and
well-formed %(but not necessarily well-typed)
TFF1 formulas into well-formed TFF0 formulas.

\begin{theorem}[Soundness of $\BOLDMON$]%
\label{thm:mon_sound}%
\afterDot
If a set of sentences $\DEL$ is satisfiable in TFF1,
then $\MON(\DEL)$ is satisfiable in TFF0.
%classical many-sorted first-order logic with equality
\end{theorem}
\begin{proof}
Let $\cM$ be a model of $\DEL$.
We construct an interpretation $\cI$ of $\MON(\DEL)$ as follows.
Let $\sorts$ and $\univ$ stand for
the set of domains and the universe in $\cM$, respectively.
In $\cI$, we define the domain of sort $\typ$ to be $\sorts$
and the domain of sort $\und$ to be $\univ$.
Each function symbol~$\hat{\kappa}$, function symbol $\hat{f}$,
or predicate symbol $\hat{p}$ is interpreted in $\cI$
exactly as the type constructor $\kappa$, function symbol $f$,
or predicate symbol $p$ in $\cM$.
The predicate symbol $\typeof$ is interpreted as the membership relation.
%This concludes the definition of interpretation $\cI$.

We show that for every type $\sigma$, term $t$,
and formula $\varphi$ occurring in $\DEL$, and
for every variable valuation $\nu$ in $\cI$, we have
\begin{align*}
\llb \MON(\sigma) \rrb^\cI_{\nu} &= \llb \sigma \rrb^\cM_{\theta} &
\llb \MON(t) \rrb^\cI_{\nu} &= \llb t \rrb^\cM_{\theta,\xi} &
\llb \MON(\varphi) \rrb^\cI_{\nu} &= \llb \varphi \rrb^\cM_{\theta,\xi}
\end{align*}
where $\theta(\alpha) \inlineeqdef \nu(\hat{\alpha})$ and
$\xi(u) \inlineeqdef \nu(\hat{u})$ for every
type variable $\alpha$ and every variable $u$.
%
We prove these equalities by induction on the structure
of types, terms, and formulas. The only nontrivial case is
that of a quantified formula, $\forall u \ty \tau .\; \varphi$.
It is easy to see that any $a \in \univ$ belongs
to $\llb \tau \rrb^\cM_\theta$ if and only if
$\llb \typeof(\hat{u},\MON(\tau)) \rrb^\cI_{\nu'}$,
where $\nu' = \nu[\hat{u} \mapsto a]$, holds.
Indeed, the latter is equivalent to
$a \in \llb \MON(\tau) \rrb^\cI_{\nu'}$.
Since $\nu$ and $\nu'$ produce the same
$\theta$ in $\cM$,
$\llb \MON(\tau) \rrb^\cI_{\nu'} = \llb \tau \rrb^\cM_\theta$.
Then
$\llb \forall \hat{u} .\; \typeof(\hat{u},\MON(\tau))
\limp \MON(\varphi) \rrb^\cI_{\nu}$ is exactly
$\llb \forall u \ty \tau .\; \varphi \rrb^\cM_{\theta,\xi}$
by induction hypothesis.

Now we only have to check that the typing axioms
$\AxD$ and the inhabitation axiom $\Inh$ hold in $\cI$.
This immediately follows from the definition of
interpretation in TFF1 and the first equality above.
\qed
\end{proof}

\begin{theorem}[Completeness of $\BOLDMON$] \label{thm:mon_compl}
Given a set of sentences $\DEL$, if $\MON(\DEL)$
is satisfiable in classical many-sorted first-order logic
with equality, then $\DEL$ is satisfiable in TFF1.
\end{theorem}
\begin{proof}
Let $\cM_0$ be a model of $\MON(\DEL)$. We first construct a model $\cM$ of $\MON(\DEL)$
from $\cM_0$ by replacing every element $d$ in
the domain of $\typ$ by the set $D = \{\la a,\,d \ra \vb \exists a.\; \typeof^{\cM_0\!}(a,\,d)\}$
and updating the interpretations of type constructors
and function and predicate symbols in $\cM_0$ accordingly.
We can safely perform this substitution: Since the inhabitation
axiom $\Inh$ holds in $\cM_0$, every set~$D$ is nonempty, and clearly distinct
elements are mapped to distinct sets.
The predicate $\typeof^{\cM\!}(a,\,D)$ holds if and only if
$D$ contains a pair of the form $\la a,\, d \ra$ for some $d$.
%
Below, $\pi_1$ and $\pi_2$ stand for the first and the second projection
of a pair, respectively.

We construct an interpretation $\cI$ of $\DEL$ as follows.
The set of domains $\sorts$ is the new domain
of $\typ$ in $\cM$. As usual, $\univ$ denotes the union
of all domains in $\sorts$. Note that some elements of
the domain of $\und$ may not appear in any pair in $\univ$.
%
The symbols of $\DEL$ are interpreted in $\cI$
according to the equations
\begin{align*}
\kappa^\cI(D_1,\dots,D_m) &\eqdef \hat{\kappa}{\,}^\cM(D_1,\dots,D_m) \\
f^\cI(D_1,\dots,D_m,\la a_1,d_1 \ra,\dots,\la a_n,d_n \ra) &\eqdef
\bigl\la \hat{f}{\,}^\cM(D_1,\dots,D_m,a_1,\dots,a_n),\, d \bigr\ra \\
p^\cI(D_1,\dots,D_m,\la a_1,d_1 \ra,\dots,\la a_n,d_n \ra) &\eqdef
\hat{p}{\,}^\cM(D_1,\dots,D_m,a_1,\dots,a_n)
\end{align*}
where
$\kappa$ is of arity $m$,
$p : \forall \alpha_1\dots\alpha_m .\; \tau_1 \times \dots \times \tau_n \to
\omicron$,
$f : \forall \alpha_1\dots\alpha_m .\; \tau_1 \times \dots \times \tau_n \to \tau$,
and $d$ is the fixed second coordinate of pairs in the domain
%
\[D = \llb \MON(\tau) \rrb^\cM_{[\hat{\alpha}_1\,\mapsto\,D_1,\dots,
\hat{\alpha}_m\,\mapsto\,D_m]} =
\llb \tau \rrb^\cI_{[\alpha_1\,\mapsto\,D_1,\dots,\alpha_m\,\mapsto\,D_m]}\]%.
%
Since the typing axioms $\AxD$ hold in $\cM$, we have
$\typeof^\cM(f^\cM(D_1,\dots,D_n,a_1,\dots,a_m), D)$,
and therefore, the result of $f^\cI$ belongs indeed to $D$.
%This concludes the definition of $\cI$.

Given a type context $\GAM$, a type valuation $\theta$,
and a variable valuation $\xi$, we say that they are \emph{admissible}
for a TFF1 formula $\varphi$ when the following conditions are satisfied:
\begin{itemize}
\item $\varphi$ is well-typed in context $\GAM$;
\item for every variable $v$ free in $\varphi$, no type variable
occurring in $\GAM(v)$ is bound in $\varphi$;
\item for every variable $v$ free in $\varphi$, we have $\xi(v) \in \evalT{\GAM(v)}$.
\end{itemize}
Obviously, any triple is admissible for a sentence.
We must show that for every type $\sigma$, term $t$, and formula $\varphi$,
and for all $\GAM$, $\theta$, $\xi$ admissible for $\varphi$, we have
\begin{align*}
\evalT{\sigma} &= \llb \MON(\sigma) \rrb^\cM_{\nu} &
\pi_1(\eval{t}) &= \llb \MON(t) \rrb^\cM_{\nu} &
\eval{\varphi} &= \llb \MON(\varphi) \rrb^\cM_{\nu}
\end{align*}
where $\nu(\hat{\alpha}) \inlineeqdef \theta(\alpha)$ and
$\nu(\hat{u}) \inlineeqdef \pi_1(\xi(u))$
for every type variable $\alpha$ and variable $u$.

The proof goes by induction on the structure of types, terms, and formulas.
There are three nontrivial cases:\ equality, quantification over a term
variable, and quantification over a type variable.

Let $\varphi$ be an equality $t_1 \eq t_2$.
Let $\la a_1,\, d_1 \ra$ be $\eval{t_1}$ and $\la a_2,\, d_2 \ra$ be $\eval{t_2}$.
By induction hypothesis, $\llb \MON(t_1) \rrb^\cM_{\nu} = a_1$ and
$\llb \MON(t_2) \rrb^\cM_{\nu} = a_2$. We must show $d_1 = d_2$.
By assumption, $\varphi$ is well-typed in $\GAM$.
Thus, $\GAM\,\vdash t_1 \ty \sigma$ and $\GAM\,\vdash t_2 \ty \sigma$
for some type~$\sigma$.
If $t_1$ is a variable $v$, then
$\eval{t_1} = \xi(v) \in \eval{\GAM(v)} = \evalT{\sigma}$. Otherwise,
$t_1$ is a function application $f(\sigma_1,\dots,\sigma_m,s_1,\dots,s_n)$,
for $f \ty
\forall \alpha_1\dots\alpha_m .\; \tau_1 \times \dots \times \tau_n \to \tau$.
Let $D_i = \evalT{\sigma_i}$ for every $i \in [1,m]$.
Then $\sigma = \tau[\sigma_1/\alpha_1,\dots,\sigma_n/\alpha_n]$ and
$\evalT{\sigma} = \llb \tau \rrb^\cI_{[\alpha_1 \,\mapsto\, D_1,\dots,
\alpha_n \,\mapsto\, D_n]}$.
By construction of $\cI$, $\eval{t_1} \in \evalT{\sigma}$.
By the same argument, $\eval{t_2} \in \evalT{\sigma}$.
Hence, $d_1 = d_2$ and $\eval{\varphi} = \llb \MON(\varphi) \rrb^\cM_{\nu}$.

Let $\varphi$ be a quantified formula $\forall u \ty \tau .\; \psi$.
We must show that for every pair $\la a,\, d \ra$ in $\evalT{\tau}$, we have
$\typeof^\cM(a, \llb \MON(\tau) \rrb^\cM_{\nu'})$,
where $\nu' = \nu[\hat{u} \mapsto a]$,
and vice versa, for every $a$ in the domain of $\und$, if
$\typeof^\cM(a, \llb \MON(\tau) \rrb^\cM_{\nu'})$
holds, then there exists some $d$ such that $\la a,\, d \ra \in
\evalT{\tau}$.
%
Since term variables do not occur in types,
$\llb \MON(\tau) \rrb^\cM_{\nu'} =
\llb \tau \rrb^\cI_\theta$.
By construction of $\cM$,
$\typeof^\cM(a, \llb \tau \rrb^\cI_\theta)$
holds if and only if there exists some $d'$ such that
$\la a,\, d' \ra \in \llb \tau \rrb^\cI_\theta$.
%
Now, notice that the triple $\GAM[u \mapsto \tau],
\theta,\xi[u \mapsto \la a,\, d \ra]$ is admissible for $\psi$.
Indeed, $\psi$ is well-typed in $\GAM[u \mapsto \tau]$,
and for every $v \in \FV(\psi)$, we have
\[\xi[u \mapsto \la a,\, d \ra](v) \in \evalT{\GAM[u \mapsto \tau](v)}\]%.
Also, if $\tau$ contains a type variable $\beta$ bound in $\psi$,
then $\beta$ is both free and bound in $\varphi$, which violates
the no-clash assumption.
%
Then, by induction hypothesis, $\eval{\forall u \ty \tau .\; \psi}$
is exactly $\llb \forall \hat{u} .\; \typeof(\hat{u},\MON(\tau))
\limp \MON(\psi) \rrb^\cM_{\nu}$.

Let $\varphi$ be a quantified formula $\forall \alpha .\; \psi$.
Let $D$ be an element of $\sorts$. Under the no-clash assumption,
formula $\psi$ is well-typed in $\GAM$, and for all
$v \in \FV(\psi) = \FV(\varphi)$, we have
$\evalG{\GAM(v)}_{\theta[\alpha\,\mapsto\,D]} = \evalT{\GAM(v)}$.
Then $\eval{\varphi} = \llb \MON(\varphi) \rrb^\cM_{\nu}$ by induction hypothesis.
\qed
\end{proof}
