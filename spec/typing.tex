\section{Typing Rules}
\label{sec_typing_rules}

Let $\GAM$ be a {\em type context}, a function that maps
every variable symbol to a type.
A type judgment $\GAM \,\vdash t \ty \tau$ expresses that that the term $t$
is {\em well-typed\/} and has type $\tau$ in context~$\GAM$.
A type judgment $\GAM \,\vdash \varphi \ty \omicron$ expresses that the formula
$\varphi$ is {\em well-typed\/} in~$\GAM$.
We write $f \ty \forall \alpha_1\dots\alpha_m .\;
\tau_1 \times \dots \times \tau_n \to \tau$ and
$p \ty \forall \alpha_1\dots\alpha_m .\;
\tau_1 \times \dots \times \tau_n \to \omicron$ to specify
type signatures of function and predicate symbols,
where both $m$ and $n$ can be 0.

The typing rules of TFF1 are given below:
%in Figure~\ref{fig:typing}.

\vskip\abovedisplayskip

\centerline{$\displaystyle \ourfrac{}{\GAM \,\vdash\, v \ty \GAM(v)}$}

\vskip3\smallskipamount

\centerline{$\displaystyle
\ourfrac{f \ty \forall \alpha_1\dots\alpha_m .\;
    \tau_1 \times \dots \times \tau_n \to \tau
\qquad
\GAM \,\vdash\, t_1 \ty \tau_1\,\rho
\quad\cdots\quad
\GAM \,\vdash\, t_n \ty \tau_n\,\rho
}{\GAM \,\vdash\,
f(\alpha_1\,\rho,\>\dots,\>\alpha_m\,\rho,\>t_1,\>\dots,\>t_n) \ty \tau\rho}
$}

\vskip3\smallskipamount

\centerline{$\displaystyle
\ourfrac{p \ty \forall \alpha_1\dots\alpha_m .\;
    \tau_1 \times \dots \times \tau_n \to \omicron
\qquad
\GAM \,\vdash\, t_1 \ty \tau_1\,\rho
\quad\cdots\quad
\GAM \,\vdash\, t_n \ty \tau_n\,\rho
}{\GAM \,\vdash\,
p(\alpha_1\,\rho,\>\dots,\>\alpha_m\,\>\rho,\>t_1,\>\dots,\>t_n) \ty \omicron}
$}

\vskip3\smallskipamount

\centerline{$\displaystyle
\ourfrac{\GAM \,\vdash\, t_1 \ty \tau \qquad \GAM \,\vdash\, t_2 \ty \tau}{
\GAM \,\vdash\, t_1 \eq t_2 \ty \omicron}$\qquad$\displaystyle
\ourfrac{\GAM \,\vdash\, \varphi \ty \omicron \qquad
\GAM \,\vdash\, \psi \ty \omicron}{
\GAM \,\vdash\, \varphi \rand \psi \ty \omicron}$}

\vskip3\smallskipamount

\centerline{$\displaystyle
\ourfrac{\GAM \,\vdash\, \varphi \ty \omicron}{
\GAM \,\vdash\, \lnot\, \varphi \ty \omicron}\qquad\qquad
\ourfrac{\GAM[v \mapsto \tau] \,\vdash\, \varphi \ty \omicron}{
\GAM \,\vdash\, \forall v \mathbin{\ty} \tau .\; \varphi \ty \omicron}\qquad\qquad
\ourfrac{\GAM \,\vdash\, \varphi[\alpha'\!/\alpha] \ty \omicron}{
\GAM \,\vdash\, \forall \alpha .\; \varphi \ty \omicron}$}

\vskip\belowdisplayskip

%\caption{Typing rules of TFF1}
%\label{fig:typing}
%\end{figure}

We write $\GAM[u \mapsto \tau]$ to denote a type context that maps
variable $u$ to type $\tau$ and every other variable $v$ to $\GAM(v)$.
In the last rule, $\alpha'$ is an arbitrary type variable that
occurs neither in $\varphi$ nor in the values of~$\GAM$.
The renaming is necessary to reject formulas such as
$\forall\alpha.\:\forall u\ty\alpha.\:\forall\alpha.\:\forall v\ty\alpha.\;
u \eq v$, where the types of $u$ and $v$ are actually different.
In order to simplify our subsequent definitions, we assume from now on
that no type variable can be both free and bound in the same formula;
we call this the {\em no-clash assumption}.
%Consequently, nested quantifiers on the same type
%variable are not allowed, either.
Then we can do without explicit renaming of type variables,
and the last typing rule's antecedent  is simply
$\GAM \,\vdash\, \varphi \ty \omicron$.

A closed TFF1 formula $\varphi$ is {\em well-typed\/} if and only if
the judgment $\GAM \vdash \varphi \ty \omicron$ is derivable
for any $\GAM$.
Obviously, if a closed formula is
well-typed in one type context, it is well-typed in any other one;
hence, we omit $\GAM$ and simply write
${} \vdash \varphi \ty \omicron$.
Closed well-typed formulas are called {\em sentences}.
