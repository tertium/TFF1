\section{Applications}
\label{sec_apps}

A number of applications already support TFF1. Geoff
Sutcliffe has extended the TPTP World infrastructure to process TFF1 problems
and solutions. This involved adapting the Backus--Naur form specification of the
TPTP syntaxes, from which parsers are generated.%
\footnote{\url{http://www.cs.miami.edu/~tptp/TPTP/SyntaxBNF.html}}
Some TPTP tools still need to be ported to TFF1; this is ongoing work.

%%% TODO: update: grep "pure"
The Why3 \cite{bobot-et-al-2011} environment, which defines its own ML-like
polymorphic specification language, can parse pure TFF1. Why3 translates
between TFF1 and a wide range of
formats, including FOF, SMT-LIB, and
Alt-Ergo's native syntax.
In addition, Why3's TFF1 parser is being ported to
Alt-Ergo \cite{bobot-et-al-2008}, so that it can directly process TFF1. % problems.
%without taking a detour through Why3.
%Code sharing is facilitated here by the use of
%a common implementation language (OCaml).

HOL(y)Hammer \cite{kaliszyk-urban-2013}
and Sledgehammer \cite{paulson-blanchette-2010}
integrate various automatic provers
in the proof assistants HOL~Light and Isabelle\slash HOL. They have been
extended to output pure TFF1 problems for Alt-Ergo and Why3
(in addition to FOF, TFF0, and THF0).
For the first time, Sledgehammer
exploits the polymorphic potential of these tools---without having
to implement their (incompatible) native file formats.
Moreover, using Sledgehammer, we produced 987 problems to populate the TPTP
library.%
\footnote{\url{http://www.cs.miami.edu/~tptp/TPTP/Proposals/TFF1.html}}
%PFF (``\relax{P}olymorphic T\relax{FF}'')
%domain of the TPTP library.
By extending the tool with a TFF1 parser,
we hope to transform it into a versatile translator to FOF and
TFF0.

%Other tools, notably an LF-based type checker similar to TwHelFTC for THF0, are
%in development.

%  * ToFoF, Monotonox, other -ox tools?

%  * TPTP Library:
%     * accepts all problems
%     * populated it with problems generated by Sledgehammer and Why3
%       (and Alt-Ergo?)
