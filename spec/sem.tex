\section{Type Checking and Semantics} \label{sec_semantics}

\ourparagraph{Notation.} In the remaining sections, we use standard
mathematical notation to write types, terms, and formulas. Our conventions
for metavariables are summarized below.
%
\begin{center}
\begin{tabular}{l@{\enskip}l@{\qquad}l@{\enskip}l}
Type variables: & $\alpha,$ $\beta$ &
  Function symbols: & $f,$ $g$ \\
Type constructors: & $\kappa$ &
  Predicate symbols: & $p,$ $q$ \\
Types: & $\sigma$, $\tau$ &
  Terms: & $s$, $t$ \\
Term variables: & $u,$ $v$ &
  Formulas: & $\varphi,$ $\psi$
\end{tabular}
\end{center}
%
%We write the names of type constructors, function, and predicate symbols using a
%fixed-width font:\ {\tt list}, {\tt map}, {\tt cons}, {\tt set}.
%% @ANDREI: There were very few examples of this in the rest of the document (lessEq and isOrdered) and
%% it was obvious from the context what they were. Also, I'm trying italics for those, to mix
%% better with the maths; but I'm using the macro \sym, so it's easy to change.
Possibly empty lists of types and terms are denoted by $\bar{\tau}$ and
$\bar{t}$, respectively.

We use the symbols ${\times}$, ${\to}$, and ${\ALL}$
%\footnote{We can also use ${\prod}$, but should we,
%given that there are no dependent types in TFF1?}
to write type signatures, and write $\omicron$ (lowercase omicron) for the
Boolean type {\tt \$o}. It is convenient to treat
equality ($\eq$), negation ($\lnot$), conjunction ($\rand$), and universal
quantification ($\forall$) as logical symbols and regard disequality
($\not\eq$), disjunction ($\lor$), implication ($\limp$), reverse implication
($\revlimp$), equivalence ($\lequ$), inequivalence ($\inlequ$), and existential
quantification ($\exists$) as abbreviations.
%The symbols $\top$ and $\bot$ stand for ``true'' and ``false'',
%respectively.

Equality can be seen as a polymorphic predicate with the signature
$\forall\alpha.\; \alpha\times\alpha\to\omicron$, but the type instance is
implicitly specified by the type of the arguments, instead of explicitly via a
type argument. Hence, it is preferable to consider it a logical
symbol.

The set of type variables occurring freely in a formula $\varphi$
(in the type arguments of polymorphic symbols or
in the types of bound variables) is denoted by $\FVT(\varphi)$.
The set of free term variables of $\varphi$
is denoted by $\FV(\varphi)$. The formula $\varphi$ is {\em closed\/}
if both $\FVT(\varphi)$ and $\FV(\varphi)$ are empty.

A {\em type substitution\/} $\rho$ is a mapping of type variables
to types. A {\em monomorphic\/} type substitution
maps every type variable either to itself or to a monomorphic type.
%
%A type $\tau$ is said to {\em match\/} another type $\tau'$ whenever there exists a
%type substitution~$\rho$ such that $\tau\rho = \tau'$.
%We reserve $\rho$ for type substitutions.
%
%The operator $\circ$ composes two substitutions:\
%$\tau(\rho \circ \rho') \eqdef (\tau\rho)\rho' = \tau\rho\rho'$.

\ourparagraph{Typing Rules.}
Let $\GAM$ be a {\em type context}, a function that maps
every variable symbol to a type.
A type judgment $\GAM \,\vdash t \ty \tau$ expresses that that the term $t$
is {\em well-typed\/} and has type $\tau$ in context~$\GAM$.
A type judgment $\GAM \,\vdash \varphi \ty \omicron$ expresses that the formula
$\varphi$ is {\em well-typed\/} in~$\GAM$.
We write $f \ty \forall \alpha_1\dots\alpha_m .\;
\tau_1 \times \dots \times \tau_n \to \tau$ and
$p \ty \forall \alpha_1\dots\alpha_m .\;
\tau_1 \times \dots \times \tau_n \to \omicron$ to specify
type signatures of function and predicate symbols,
where $m$ and $n$ can be 0.

The typing rules of TFF1 are given below:
%in Figure~\ref{fig:typing}.

\vskip\abovedisplayskip

\centerline{$\displaystyle \ourfrac{}{\GAM \,\vdash\, v \ty \GAM(v)}$}

\vskip3\smallskipamount

\centerline{$\displaystyle
\ourfrac{f \ty \forall \alpha_1\dots\alpha_m .\;
    \tau_1 \times \dots \times \tau_n \to \tau
\qquad
\GAM \,\vdash\, t_1 \ty \tau_1\,\rho
\quad\cdots\quad
\GAM \,\vdash\, t_n \ty \tau_n\,\rho
}{\GAM \,\vdash\,
f(\alpha_1\,\rho,\>\dots,\>\alpha_m\,\rho,\>t_1,\>\dots,\>t_n) \ty \tau\rho}
$}

\vskip3\smallskipamount

\centerline{$\displaystyle
\ourfrac{p \ty \forall \alpha_1\dots\alpha_m .\;
    \tau_1 \times \dots \times \tau_n \to \omicron
\qquad
\GAM \,\vdash\, t_1 \ty \tau_1\,\rho
\quad\cdots\quad
\GAM \,\vdash\, t_n \ty \tau_n\,\rho
}{\GAM \,\vdash\,
p(\alpha_1\,\rho,\>\dots,\>\alpha_m\,\>\rho,\>t_1,\>\dots,\>t_n) \ty \omicron}
$}

\vskip3\smallskipamount

\centerline{$\displaystyle
\ourfrac{\GAM \,\vdash\, t_1 \ty \tau \qquad \GAM \,\vdash\, t_2 \ty \tau}{
\GAM \,\vdash\, t_1 \eq t_2 \ty \omicron}$\qquad$\displaystyle
\ourfrac{\GAM \,\vdash\, \varphi \ty \omicron \qquad
\GAM \,\vdash\, \psi \ty \omicron}{
\GAM \,\vdash\, \varphi \rand \psi \ty \omicron}$}

\vskip3\smallskipamount

\centerline{$\displaystyle
\ourfrac{\GAM \,\vdash\, \varphi \ty \omicron}{
\GAM \,\vdash\, \lnot\, \varphi \ty \omicron}\qquad\qquad
\ourfrac{\GAM[v \mapsto \tau] \,\vdash\, \varphi \ty \omicron}{
\GAM \,\vdash\, \forall v \mathbin{\ty} \tau .\; \varphi \ty \omicron}\qquad\qquad
\ourfrac{\GAM \,\vdash\, \varphi[\alpha'\!/\alpha] \ty \omicron}{
\GAM \,\vdash\, \forall \alpha .\; \varphi \ty \omicron}$}

\vskip\belowdisplayskip

%\caption{Typing rules of TFF1}
%\label{fig:typing}
%\end{figure}

We write $\GAM[u \mapsto \tau]$ to denote a type context that maps
variable $u$ to type $\tau$ and every other variable $v$ to $\GAM(v)$.
In the last rule, $\alpha'$ is an arbitrary type variable that
occurs neither in $\varphi$ nor in the values of~$\GAM$.
The renaming is necessary to reject formulas such as
$\forall\alpha.\:\forall u\ty\alpha.\:\forall\alpha.\:\forall v\ty\alpha.\;
u \eq v$, where the types of $u$ and $v$ are actually different.
In order to simplify our subsequent definitions, we assume from now on
that no type variable can be both free and bound in the same formula;
we call this the {\em no-clash assumption}.
%Consequently, nested quantifiers on the same type
%variable are not allowed, either.
Then we can do without explicit renaming of type variables,
and the last typing rule's premise is simply
$\GAM \,\vdash\, \varphi \ty \omicron$.

A closed TFF1 formula $\varphi$ is {\em well-typed\/} if and only if
the judgment $\GAM \vdash \varphi \ty \omicron$ is derivable
for any $\GAM$.
Obviously, if a closed formula is
well-typed in one type context, it is well-typed in any other one;
hence, we omit $\GAM$ and simply write
${} \vdash \varphi \ty \omicron$.
Closed well-typed formulas are called {\em sentences}.

\ourparagraph{Interpretation.}
%The semantics of a polymorphic formula coincides with that of its
%monomorphic instances.
%%% @JASMIN: A "monomorphic instance" is a nontrivial notion in
%%% presence of quantifiers over types.
%
The semantics of TFF1 postulates a nonempty collection $\sorts$ of
%(not necessarily mutually disjoint)
%%% @JASMIN: Trop lourd sans trop d'info
nonempty sets, the {\em domains}. The union of all domains is the
{\em universe}, %denoted by
$\univ$.
An interpretation $\cI$ for a given set of type constructors,
function symbols, and predicate symbols is constructed as follows.

An $n$-ary type constructor $\kappa$ is interpreted as a function
$\kappa^{\cI} : \sorts^n \to \sorts$.
Let $\theta$ be a {\em type valuation}, a function that maps every
type variable to a domain. Types are evaluated according to the following
equations:
\begin{align*}
\evalT{\alpha} &\eqdef \theta(\alpha) &
\evalT{\kappa(\tau_1,\dots,\tau_n)} &\eqdef \kappa^{\cI}(\evalT{\tau_1},\,\dots,\,\evalT{\tau_n})
\end{align*}
Since type evaluation depends only on the values of $\theta$
on the type variables occurring in a type, we write $\evalG{\tau}$
to denote the domain of a monomorphic type $\tau$.
%% NEEDED?:
%%% @JASMIN: yes, it is used in the proof of Theorem 2
%We also write
%$\evalG{\tau}_{[\alpha_1\,\mapsto\,D_1,\,\dots,\,\alpha_m\,\mapsto\,D_m]}$
%to evaluate a type whose free type variables
%are among $\alpha_1,\dots,\alpha_m$.

A predicate symbol $p : \forall \alpha_1\dots\alpha_m .\; \tau_1 \times \dots \times \tau_n
\to \omicron$ is interpreted as a relation
$\smash{p^{\cI}} \subseteq \sorts^m \times \univ^n$.
A function symbol $f : \forall \alpha_1\dots\alpha_m .\; \tau_1 \times \dots \times \tau_n \to \tau$
is interpreted as a function
$f^{\cI}$ on $\sorts^m \times \univ^n$ that
maps any $m$ domains $D_1,\dots,D_m$ and
any $n$ universe elements to an element of
$\evalG{\tau}_\theta$ where $\theta$ maps each $\alpha_i$ to $D_i$.
%$\evalG{\tau}_{[\alpha_1\,\mapsto\,D_1,\,\dots,\,\alpha_m\,\mapsto\,D_m]}$.
%Recall that every type variable occurring in a type signature
%must be bound, and therefore every type variable in $\tau$ belongs
%to $\{ \alpha_1,\dots,\alpha_m \}$.

Let $\xi$ be a {\em variable valuation}, a function that assigns
to every variable an element of $\univ$. TFF1 terms
and formulas are evaluated according to the following equations:
\begin{align*}
\eval{u} &\eqdef \xi(u) &
\eval{f(\bar{\tau},\,\bar{t}\,)} &\eqdef f^\cI(\evalT{\bar{\tau}},\eval{\bar{t}\,}) \\
\eval{t_1 \eq t_2} &\eqdef (\eval{t_1} = \eval{t_2}) &
\eval{p(\bar{\tau},\,\bar{t}\,)} &\eqdef p^\cI(\evalT{\bar{\tau}},\eval{\bar{t}\,}) \\
\eval{\lnot\, \varphi} &\eqdef \lnot\, \eval{\varphi} &
\eval{\varphi \rand \psi} &\eqdef \eval{\varphi} \rand \eval{\psi} \\
\eval{\forall u \ty \tau .\; \varphi} &\eqdef \forall {a \mathbin\in \evalT{\tau}} .\;
\evalG{\varphi}_{\theta,\,\xi[u\,\mapsto\,a]} &
\eval{\forall \alpha .\; \varphi} &\eqdef \forall {D \in \sorts} .\;
\evalG{\varphi}_{\theta[\alpha\,\mapsto\,D],\,\xi} % \\
%\eval{\forall u \ty \tau .\; \varphi} &\eqdef \!\!\!\!
%\bigwedge_{a \in \evalT{\tau}} \!
%\evalG{\varphi}_{\theta,\,\xi[u\,\mapsto\,a]} &
%\eval{\forall \alpha .\; \varphi} &\eqdef
%\bigwedge_{D \in \sorts}\;
%\evalG{\varphi}_{\theta[\alpha\,\mapsto\,D],\,\xi}
\end{align*}
The expression $\xi[u \mapsto a]$ stands for the function
that maps $u$ to $a$ and every other variable $v$ to $\xi(v)$, and likewise for
$\theta[\alpha \mapsto D]$.
%Likewise, $\theta[\alpha \mapsto D]$ is the function that maps
%type variable $\alpha$ to domain $D$ and every other type
%variable $\beta$ to $\theta(\beta)$.
We omit irrelevant subscripts and write
$\evalG{\varphi}$ to denote the evaluation of a closed formula.

A sentence $\varphi$ is {\em true} in an interpretation $\cI$,
written $\cI \models \varphi$, if and only if $\evalG{\varphi}$ is true.
The interpretation $\cI$ is then a {\em model\/} of $\varphi$.
A sentence that has a model is {\em satisfiable}.
A sentence that is true in every interpretation is {\em valid}.
These notions are extended to sets and sequents of TFF1 sentences in
the usual way.
