\section{Preprocessing of Type Quantifiers}
\label{sec_preproc}

We describe two preprocessing steps that preserve both satisfiability and
unsatisfiability of TFF1 problems. They eliminate ghost type variables in type
signatures and alternating $\forall\raise.15ex\hbox{\ensuremath{\slash}}\vthinspace\exists$ type quantifier prefixes in
formulas, two features that are not directly supported by ML-style formalisms
(which otherwise are a good match for TFF1).

\subsection{Elimination of Ghost Type Arguments} \label{ssec:ghost}

%\footnote{Other relevant citations, please? What about Mizar?}.
% @ANDREI: I'd rather stick to "ML-style formalisms" here. Mizar is based on
% set theory and has at most soft types; it would be better treated in the
% Related Work section.

ML-style formalisms, as implemented in Alt-Ergo~\cite{bobot-et-al-2008},
Boogie~\cite{leino-ruemmer-2010}, HOL \cite{gordon-melham-1993}, HOL
Light~\cite{harrison-1996}, Isabelle\slash HOL~\cite{nipkow-et-al-2002},
Why3~\cite{bobot-et-al-2011}, and several other systems, allow type variables
in type signatures, but without explicit $\forall$-binders.
Function and predicate symbols do not take explicit type
arguments; instead, the concrete instance of the symbol's type signature is
determined by the types of its term arguments, the type of the result, and
optional type annotations inside terms.

The natural translation from TFF1 to such a formalism would map $\forall
\alpha\beta.\; \alpha\times\beta\to\omicron$ (and $\forall \beta\alpha.\;
\alpha\times\beta\to\omicron$) to $\alpha\times\beta\to\omicron$, simply
omitting the $\forall$-binders. To compensate for the missing type arguments,
type annotations are sometimes needed to guide the Hindley--Milner type
inference.

The difficulties arise in conjunction with ghost type variables: If
$\forall\alpha.\;\omicron$ collapses to $\omicron$, the dependency on the type
is lost.
%
To ease the adoption of TFF1 in such systems, we suggest a preprocessing step
that eliminates ghost type variables. This step is lightweight, as it requires
only the introduction of one term argument per ghost type. In particular,
it is the identity for formulas that do not rely on ghost type variables.

%that is, the type variables in the symbol's type signature that
%do not appear in the quantifier-free matrix.

Let $\DEL$ be a set of sentences.
We construct an equisatisfiable set $\ELI(\DEL)$ as follows.
%
We introduce a special unary type constructor $\ghost$.
We replace every function symbol $f : \forall \alpha_1\dots\alpha_m .\; \tau_1 \times \dots \times \tau_n \to \tau$
with a new function symbol $\hat{f} : \forall \alpha_1\dots\alpha_m .\;
\ghost(\alpha_{i_1}) \times \dots \times \ghost(\alpha_{i_r}) \times
\tau_1 \times \dots \times \tau_n \to \tau$,
where $\alpha_{i_1},\dots,\alpha_{i_r}$ are the type
variables from the quantifier prefix that do not occur in
$\tau_1,\dots,\tau_n$ or $\tau$.
Similarly, every predicate symbol $p \,\ty\,
\forall \alpha_1\dots\alpha_m .\;\allowbreak \tau_1 \times \dots \times \tau_n \to \omicron$
is replaced with a new predicate symbol $\hat{p} \,\ty\,
\forall \alpha_1\dots\alpha_m .\;\allowbreak
\ghost(\alpha_{i_1}) \times \dots \times \ghost(\alpha_{i_r}) \times
\tau_1 \times \dots \times \tau_n \to \omicron$,
where $\alpha_{i_1},\dots,\alpha_{i_r}$ are the type
variables from the quantifier prefix that do not occur in
$\tau_1,\dots,\tau_n$.
Finally, we add a ``witness'' constant $\wit : \forall \alpha.\;\ghost(\alpha)$.
The $\ELI$ transformation translates terms and atomic
formulas according to the following equations:
%\begin{align*}
%\ELI(u) & \eqdef u &
%\ELI(s \eq t) & \eqdef \ELI(s) \eq \ELI(t)
%\end{align*}
\begin{align*}
\ELI(u) \eqdef u
\kern1.4em & \kern5.55em %% TYPESETTING
\ELI(s \eq t) \eqdef \ELI(s) \eq \ELI(t) \\
\ELI(f(\sigma_1,\dots,\sigma_m,\bar t\,)) &\eqdef
\hat{f}(\sigma_1,\dots,\sigma_m,\wit(\sigma_{i_1}),\dots,\wit(\sigma_{i_r}),
\ELI(\bar t\,)) \\
\ELI(p(\sigma_1,\dots,\sigma_m,\bar t\,)) &\eqdef
\hat{p}(\sigma_1,\dots,\sigma_m,\wit(\sigma_{i_1}),\dots,\wit(\sigma_{i_r}),
\ELI(\bar t\,))
\end{align*}
Given a formula $\varphi$ or a set of sentences $\DEL$,
$\ELI(\varphi)$ and $\ELI(\DEL)$ denote the result of
applying $\ELI$ to every atomic formula in $\varphi$ and
$\DEL$, respectively.

\begin{theorem} \label{thm:eli}
Any $\DEL$ is equisatisfiable to $\ELI(\DEL)$.
\end{theorem}
\begin{proof}
A model of $\DEL$ can be converted to a model of $\ELI(\DEL)$
as follows. We choose an arbitrary value $e$ and take the singleton
$\{ e \}$ as the domain of $\ghost(D)$ for any domain $D$.
Accordingly, $\wit$ evaluates to $e$ on any argument.
Interpretations of other function and predicate symbols are
adjusted in the obvious way.

\pagebreak[2] %% TYPESETTING

To convert a model of $\ELI(\DEL)$
to a model of $\DEL$, we evaluate any functional symbol
$f$ on $D_1,\dots,D_m,a_1,\dots,a_n$ exactly as
$\hat{f}$ on $D_1,\dots,D_m,e_1,\dots,e_r,a_1,\dots,a_n$,
where each $e_k$ is the evaluation of $\wit$ on $D_{i_k}$, and similarly
for predicate symbols.
\qed
\end{proof}

\subsection{Type Skolemization} \label{ssec:skol}

Another feature of TFF1 that is not universally supported
by systems with polymorphic types is explicit
quantification over type variables.
For example, while Boogie \cite{leino-ruemmer-2010} and Coq \cite{bertot-casteran-2004}
allow quantifiers over types, % in their logics,
HOL and other ML-style logics provide no such syntax:
They consider all type variables
implicitly universally quantified at the top of a formula, which
recalls the treatment of term variables in TPTP CNF clauses.

A simple and practical solution for preparing TFF1 problems
for ML-style logics is type skolemization. This is slightly involved because
type quantifiers may appear under equivalence, where they are unpolarized.
%
We define two transformations, $\SKOM$ and $\SKOP$, to perform type
skolemization in axioms and in conjectures, respectively:
%(The latter is more correctly called ``type Herbrandization.'') -- no space
%
%Let us call a formula $\varphi$ {\em dependently typed} whenever
%it contains a quantifier over a type variable in the scope
%of a quantifier over a term variable%
%\footnote{We might take polarities into account, but then
%the proof of Theorem~\ref{thm:sko} wouldn't be so sweet.
%Types-under-values are nasty anyway.}.
\begin{align*}
\SKOM(p(\bar{\sigma},\,\bar{t}\,)) &\eqdef p(\bar{\sigma},\,\bar{t}\,) &
\SKOP(p(\bar{\sigma},\,\bar{t}\,)) &\eqdef p(\bar{\sigma},\,\bar{t}\,) \\
\SKOM(t_1 \eq t_2) &\eqdef t_1 \eq t_2 &
\SKOP(t_1 \eq t_2) &\eqdef t_1 \eq t_2 \\
\SKOM(\lnot\, \varphi) &\eqdef \lnot\, \SKOP(\varphi) &
\SKOP(\lnot\, \varphi) &\eqdef \lnot\, \SKOM(\varphi) \\
\SKOM(\varphi \rand \psi) &\eqdef \SKOM(\varphi) \rand \SKOM(\psi) &
\SKOP(\varphi \rand \psi) &\eqdef \SKOP(\varphi) \rand \SKOP(\psi) \\
\SKOM(\forall u \ty \tau .\; \varphi) &\eqdef \forall u \ty \tau .\; \varphi &
\SKOP(\forall u \ty \tau .\; \varphi) &\eqdef \forall u \ty \tau .\; \varphi \\
\SKOM(\forall \alpha .\; \varphi) &\eqdef \forall \alpha .\; \SKOM(\varphi) &
\SKOP(\forall \alpha .\; \varphi) &\eqdef
\SKOP(\varphi[\kappa(\bar\beta)/\alpha])
\end{align*}
where $\kappa$ is a fresh type constructor and $\bar\beta$ is the list of free
type variables of $\forall \alpha .\; \varphi$.
Recall that TFF1 forbids quantifiers over type variables in the scope
of a quantifier over a term variable; otherwise, Skolem type constructors
would have to take term variables as arguments. This explains why $\SKOM$ and
$\SKOP$ simply stop at the outermost term quantifier.

Given a set of TFF1 sentences $\DEL$,
$\SKOM(\DEL)$ and $\SKOP(\DEL)$ denote the result of applying
the corresponding transformations to every formula in $\DEL$.

\begin{theorem} \label{thm:sko}
Any\/ $\DEL$ is equisatisfiable to $\SKOM(\DEL)$.
\end{theorem}
\begin{proof}
It is easy to see that
skolemizing the $\typ$-sorted variables in $\MON(\DEL)$ gives
exactly $\MON(\SKOM(\DEL))$ modulo renaming of Skolem symbols and
permutation of their arguments. Theorems~\ref{thm:mon_sound} and
\ref{thm:mon_compl} conclude the proof.
\qed
\end{proof}

%%% @ANDREI: Which restriction are you talking about? I'm not following you here:
%
%Given a TFF1 problem that does not satisfy the aforementioned restriction,
%one can submit it to a system like Isabelle/HOL or Alt-Ergo by encoding it
%with $\MON$ or any other sound translation into monomorphic logic.
